% only displays equation numbers if they are referenced
\mathtoolsset{showonlyrefs,showmanualtags}

\DeclareEmphSequence{\scshape}

% BEGIN custom geometry ************************************
\setlength{\abovedisplayskip}{0.1em}
\setlength{\belowdisplayskip}{0.2em}
\setlength{\abovedisplayshortskip}{0pt}
\setlength{\belowdisplayshortskip}{0pt}
\setlength{\jot}{0pt}

\titlespacing{\chapter}{0pt}{0em}{0.4em}
\titlespacing{\section}{0pt}{1.3em}{0.4em} %15pt, 3pt
\titlespacing{\subsection}{0pt}{1em}{0.4em} %15pt, 3pt
%\titlespacing{\subsubsection}{0pt}{6pt}{3pt}
%\titlespacing*{\chapter}{0pt}{-19pt}{0pt}

%For item, enumerate, description, lists
\setitemize{noitemsep,topsep=0pt,parsep=0pt,partopsep=0pt}
\setenumerate{noitemsep,topsep=0pt,parsep=0pt,partopsep=0pt}
\setdescription{noitemsep,topsep=0pt,parsep=0pt,partopsep=0pt}
\setlist{noitemsep,topsep=0pt,parsep=0pt,partopsep=0pt}

% END custom geometry **************************************

% new mdframed style that places the edges at the corners (.675em):
\mdfdefinestyle{proof-style}{
  skipabove         = 0.2em,% .5\baselineskip ,
  skipbelow         = 0,%.5\baselineskip ,
  leftmargin        = 0.4em ,
  rightmargin       = 0.4em ,
  innermargin       = 0pt ,
  innertopmargin    = 0.4em, %0.6 %.675em ,
  innerleftmargin   = 0.4em, %.675em ,
  innerrightmargin  = 0.4em,
  innerbottommargin = 0.2em, %.675em +3pt,
  hidealllines      = true,
  singleextra       = {
    \draw (O) -- ++(0,.7em) (O) -- ++(.7em,0) ;
    \draw (P-|O) -- ++(0,-.7em) (P-|O) -- ++(.7em,0) ;
  },
  firstextra        = {
    \draw (P-|O) -- ++(0,-.7em) (P-|O) -- ++(.7em,0) ;
  },
  secondextra       = {
    \draw (O) -- ++(0,.7em) (O) -- ++(.7em,0) ;
  },
}

\surroundwithmdframed[style=proof-style]{prf}
\surroundwithmdframed[style=proof-style]{proof}
\newtheorem*{prf}{Proof}

% custom QED Symbol
%\newcommand*\closedbox{%
  %    \leavevmode\hbox to.77778em{\rule{.675em}{.675em}}}
%\let\qedsymbol\closedbox

\renewenvironment{proof}{{\textbf{\scshape \slshape Beweis:}}}{\hfill $\qedsymbol$}
\renewenvironment{prf}{{\textbf{\scshape \slshape Beweis:}}}{\hfill $\qedsymbol$}

% put the new mdframed style around the 'proof' and 'xmpl0 environment:
%\surroundwithmdframed[style=proof]{xmpl}

% BEGIN custom therem and proof environments ***************
\newtheoremstyle{mytheoremstyle} % name
{0.4em}                       % Space above {\topsep} 
{.2em}                        % Space below
{}                            % Body font
{0em}                         % Indent amount
{}                            % Theorem head font {\ttfamily\fontseries{b}\selectfont}
{\textbf{:\,}}       % Punctuation after theorem head
{.2em}                        % Space after theorem head
{{\textbf{\scshape{\thmname{#1}\thmnumber{ #2}}}{\normalfont{\;}\thmnote{({\itshape#3})}}}} 
% Theorem head spec (can be left empty, meaning ‘normal’)

% Define 'thm', 'thm-non', 'mydef', 'mydef-non', 'example', 'example-non'
% environments

\theoremstyle{mytheoremstyle}
\newtheorem{thm}{Thm}[chapter]
\newtheorem*{thm-non}{Thm}

%\theoremstyle{mytheoremstyle}
\newtheorem{mydef}[thm]{Def}
\newtheorem*{mydef-non}{Def}

%\theoremstyle{mytheoremstyle}
\newtheorem{example}[thm]{Example}
\newtheorem*{example-non}{Example}

%\theoremstyle{mytheoremstyle}
\newtheorem{lemma}[thm]{Lemma}
\newtheorem*{lemma-non}{Lemma}

\newtheorem{imp-ex}[thm]{Wichtige Übung}
\newtheorem*{imp-ex-non}{Wichtige Übung}

\newtheorem{xrcs}{Exercise}
\newtheorem*{xrcs-non}{Exercise}

% quotes
\newcommand\qt[1]{\textit{``#1''}}

\newcommand{\gt}{>}
\newcommand{\lt}{<}

\newcommand{\mytext}[1]{\;\; \text{#1}\;\;}
\newcommand\myand{\;\;\, \text{and}\;\;}
\newcommand\myor{\;\;\, \text{or}\;\;}
\newcommand\where{\;\;\, \text{where}\;\;}
\newcommand\whereEach{\;\;\, \text{where each}\;\;}
\newcommand\und{\;\; \text{und}\;\;}
\newcommand\oder{\;\;\, \text{oder}\;\;}
\newcommand\fuer{\;\;\, \text{für}\;\;}
\newcommand\fueralle{\;\;\, \text{für alle}\;\;}
\newcommand\dx{\, dx}
\newcommand\dy{\, dy}
\newcommand\dz{\, dz}
\newcommand\du{\, du}

% SUMS 
\newcommand\sumkinfty{\sum_{k=1}^{\infty}}
\newcommand\sumkn{\sum_{k=1}^{n}}
\newcommand\sumknplone{\sum_{k=1}^{n+1}}
\newcommand\sumkzerotoinfty{\sum_{k=0}^{\infty}}
\newcommand\sumkzeroton{\sum_{k=0}^{n}}

\newcommand\sumkonetilll{\sum_{k=1}^{l}}

\newcommand\sumlinfty{\sum_{l=1}^{\infty}}
\newcommand\sumln{\sum_{l=1}^{n}}
\newcommand\sumninfty{\sum_{n=1}^{\infty}}
\newcommand\sumnM{\sum_{n=1}^{M}}

%LIMITS
\newcommand\limninfty{\lim_{n \to \infty}}

\def\degree{{\operatorname{deg} \,}}

\newcommand{\myspan}[1]{\operatorname{span} (#1)}

%Fast way to write v_1 ... v_n
\newcommand{\oneTillN}[1]{#1_1, \dots, #1_n}
\newcommand{\onetilln}[1]{#1_1, \dots, #1_n}

%Fast way to write v_1 ... v_m
\newcommand{\oneTillM}[1]{#1_1, \dots, #1_m}
\newcommand{\onetillm}[1]{#1_1, \dots, #1_m}

% fast way to write v_1 ... v_{#2}
% usage \onetill{v}{k-1} yields v_1 \dots v_{k-1}
\newcommand{\oneTill}[2]{#1_1, \dots, #1_{#2}}
\newcommand{\onetill}[2]{#1_1, \dots, #1_{#2}}

\newcommand{\kInOneTillM}{k \in \{1, \dots, m \}}
\newcommand{\kinonetillm}{k \in \{1, \dots, m \}}
\newcommand{\kInOneTillN}{k \in \{1, \dots, n \}}
\newcommand{\kinonetilln}{k \in \{1, \dots, n \}}
\newcommand{\kInOneTillP}{k \in \{1, \dots, p \}}
\newcommand{\kinonetillp}{k \in \{1, \dots, p \}}

% abreviation for finite-dimensional vector space
\newcommand{\findimvecpac}{finite-dimensional vector space }
\newcommand{\findimvs}{finite-dimensional vector space }
\newcommand{\fdvs}{finite-dimensional vector space }

%abreviation for linearly independent
\newcommand{\lid}{linearly independent}

%abreviation for linearly independent
\newcommand{\ld}{linearly dependent }

%abreviation for linearly independent
\newcommand{\vs}{vector space }

%abreviation for finite-dimensional
\newcommand{\fd}{{finite-dimensional }}

%abreviation for linear map
\newcommand{\lm}{{linear map }}

%abbreaviation for L(V,W)
\newcommand{\lvw}{{\mathcal{L}(V,W)}}

\newcommand{\linmap}{\mathcal{L}}
\newcommand{\lin}[2]{{\mathcal{L}(#1, #2)}}

\newcommand{\mynull}{\operatorname{null}}

\newcommand{\myrange}{\operatorname{range}}

\newcommand{\even}{\operatorname{even}}
\newcommand{\odd}{\operatorname{odd}}

%\newcommand{\mmatrix}{\mathcal{M}}

% Natural numbers, integers, real numbers, complex numbers:
\newcommand{\nat}{\mathbb{N}}
\newcommand{\integer}{\mathbb{N}}
\newcommand{\real}{\mathbb{R}}
\newcommand{\compl}{\mathbb{C}}
\newcommand{\myF}{\mathbb{F}}

% Polynomial symbol:
\newcommand{\polyn}{\mathcal{P}}

% Matrix symbol:
\newcommand{\mmatrix}{\mathcal{M}}

%\newcommand{\bfemph}[1]{{\ttfamily\fontseries{b}\selectfont #1}}
\newcommand{\bfemph}[1]{{\scshape\relscale{1.1} #1}}

%\newcommand{\basis}[2]{\overbrace{ \myspan{#1_1, \dots #1_{#2}}}^{\text{linearly independent}} }}

\def\myimpl{{ \{black}{\implies}}


\def\bold#1{{\bf #1}}

%\newtheoremstyle{mytheoremstyle} % name
%%{\topsep}                    % Space above
%{0.8em}                    % Space above
%{0em}                        % Space below
%{}                           % Body font
%{0em}                           % Indent amount
%%{\ttfamily\fontseries{b}\selectfont}                   % Theorem head font
%{\bfseries\scshape}                   % Theorem head font
%{:\newline}                          % Punctuation after theorem head
%{.3em}                       % Space after theorem head
%{}  					     % Theorem head spec (can be left empty, meaning ‘normal’)
%
%\theoremstyle{mytheoremstyle}
%\newtheorem{thm}{Theorem}[chapter]
%
%\theoremstyle{mytheoremstyle}
%\newtheorem{mydef}[thm]{Definition}
%\newtheorem*{mydef-non}{Definition}
%
%\theoremstyle{mytheoremstyle}
%\newtheorem{example}[thm]{Example}
%
%
%
%\newtheoremstyle{indented}
%{3pt}% space before
%{3pt}% space after
%{\addtolength{\@totalleftmargin}{3.5em}
%  \addtolength{\linewidth}{-3.5em}
%  \parshape 1 3.5em \linewidth}% body font
%{}% indent
%{\bfseries}% header font
%{.}% punctuation
%{.5em}% after theorem header
%{}% header specification (empty for default)
%\makeatother
%
%\renewenvironment{proof}
%{
%	{
%		\bfseries
%		\scshape
%		\itshape
%		\selectfont
%		Beweis:}
%}
%{
%	\hfill $\Box$ \\
%}

% make counter equal
\newcommand{\mce}[1]{\setcounter{thm}{#1-1}}

\setlength{\abovedisplayskip}{4pt}
\setlength{\belowdisplayskip}{3pt}

\setlength{\abovedisplayskip}{0.1em}
\setlength{\belowdisplayskip}{0.2em}
\setlength{\abovedisplayshortskip}{0pt}
\setlength{\belowdisplayshortskip}{0pt}
\setlength{\jot}{0pt}