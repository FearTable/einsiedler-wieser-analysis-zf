\section{Mengenlehre und Abbildungen}
\subsection{Naive Mengenlehre}

% 1.22 Disjunktheit
\begin{mydef}[Disjunktheit]
  
  $A,B$ heissen \qt{disjunkt}, falls $A\cup B = \{ \}$. 
  In diesem Fall wird $A\cup B$ \qt{disjunkte Vereinigung} genannt und aus als $A \sqcup B$ geschrieben. Für eine Kollektion $\mathcal{A}$ von Mengen, sagen wir, dass die Mengen in $A$ \qt{paarweise disjunkt} sind, falls
  \begin{equation}
    \forall A_1, A_2 \in \mathcal{A} $ mit $ A_1\neq A_2 $ gilt $ A_1 \cup A_2 = \{ \}.
  \end{equation}
  
  Die Vereinigung der Mengen in $\mathcal{A}$ wird dann auch \qt{disjunkte Vereinigung} genannt und wird in diesem fall auch als 
  \begin{equation}
    \bigsqcup_{A\in\mathcal{A}} A :\equiv \bigcup_{A\in\mathcal{A}} A
  \end{equation}
  
  geschrieben.
\end{mydef}

% 1.23 Kartesisches Produkt
\begin{mydef}[Kartesisches Produkt]
  Das \qt{kartesische Produkt} $X\times Y$ ist die Menge aller geordneter Paare $(x,y)$, wobei $x\in X$ und $y \in Y$. In Symbolen,
  \begin{equation}
    X \times Y :\equiv \{(x,y) \mid x \in X $ und $ y \in Y \}. $ Allgemeiner definiert: $
  \end{equation}
  \begin{equation}
    X_1 \times \cdots \times X_m :\equiv \{ (x_1, \ldots, x_m) \mid x_1 \in X_1, \ldots, x_m \in X_m \}.
  \end{equation}
  % TODO: allgemeiner
\end{mydef}

% 1.27 Potenzmenge
\begin{mydef}[Potenzmenge $\mathcal{P}$] Für eine Menge $X$ wird ihre \qt{Potenzmenge} wie folgt definiert
  \begin{equation}
    \mathcal{P}(X) :\equiv \{Q \mid Q \subseteq X \}
  \end{equation}
\end{mydef}

% 1.4.2
\subsection{Abbildungen}

% 1.31
\begin{mydef}[Funktionen und erste dazugehörige Begriffe]
  Falls $f:X \to Y$ eine Funktion und $A\subseteq X$ ist, dann definieren wird die \qt{Einschränkung} $\left. f \right | _{A} : A \to Y$ durch
  \begin{equation}
    \left. f \right | _{A} (x) :\equiv f(x)  \quad \forall x \in A
  \end{equation}
  
  Des Weiteren definieren wir das \qt{Bild der Teilmenge $A$} unter $f$ als
  \begin{equation}
    f(A) = \left. f \right |_{A}(A) :\equiv \{y \in Y \mid \exists x \in A : y=f(x) \}
  \end{equation}
\end{mydef}

Für eine Funktion schreibt man $f: X \to Y, x \mapsto f(x)$ oder $x \in X \mapsto f(x) \in Y$, also zum Beispiel $x \in \real \mapsto x^2 \in \real$. Wir sprechen \qt{$\mapsto$} unter Anderem als \qt{wird abgebildet auf} aus.

% 1.33
\begin{mydef}[Drei Eigenschaften von Funktionen]
  Sei $f: X \to Y$ eine Funktion.
  \begin{itemize}
    \item $f$ heisst \qt{injektiv} oder eine \qt{Injektion} falls $\forall x_1, x_2 \in X$ gilt, dass 
    \begin{equation}
      f(x_1) = f(x_2) \implies x_1 = x_2 $ oder$
    \end{equation}
    \begin{equation}
      x_1 \neq x_2 \implies f(x_1) \neq f(x_2).
    \end{equation}
    
    \item $f$ heisst \qt{surjektiv},eine \qt{Surjektion} oder eine Funktion von \qt{X auf Y}, falls 
    \begin{equation}
      \forall y \in Y \; \exists x \in X: f(x) = y
    \end{equation}
    
    Also falls $f(X) = Y$.
    
    \item $f$ heisst \qt{bijektiv}, eine \qt{Bijektion} oder eine \qt{eineindeutige Abbildung}, falls $f$ surjektiv und injektiv ist.
  \end{itemize}
\end{mydef}

% 1.40
\begin{lemma} [Eigenschaften von Verknüpfungen]
  Seien $f: X \to Y$ und $g: Y \to Z$ Funktionen.
  \begin{enumerate}[label=(\roman*)]
    \item Falls $f$ und $g$ injektiv sind, dann ist auch $g \circ f : X \to Z$ injektiv.
    \item Falls $f$ und $g$ surjektiv sind, dann ist auch $g \circ f : X \to Z$ surjektiv.
    \item Falls $f$ und $g$ bijektiv sind, dann ist auch $g \circ f : X \to Z$ bijektiv und es gilt 
    \begin{equation}
      (g \circ f)^{-1} = f^{-1} \circ g^{-1}
    \end{equation}
  \end{enumerate}
  
\end{lemma}
\begin{prf}
  %\phantom{.}
  \begin{enumerate}[label=(\roman*)]
    \item
    { 
      Seien $f: X\to Y$ und $g:Y \to Z$ injektiv und $ x_1, x_2 \in X$ beliebig gewählt, sodass
      \begin{equation}
        g \circ f(x_1) = g \circ f(x_2).
      \end{equation}
      
      Aus der Definition der Verknüpfung $g(f(x_1)) = g(f(x_2))$ und weil $g$ injektiv ist folgt, dass \begin{equation}
        f(x_1) = f(x_2). $ Dies impliziert aufgrund der Injektivität von $f$ ebenso, dass $
      \end{equation}
      \begin{equation}
        x_1 = x_2. $ Damit ist auch $g \circ f$ injektiv.$
      \end{equation}
    }
    \item
    {
      Angenommen $f: X\to Y$ und $g:Y \to Z$ sind surjektiv und $z \in Z$ ein beliebiges Element. Aus der Surjektivität von g folgt
      \begin{equation}
        \exists y \in Y: g(y) = z. $ Ebenso gilt aufgrund der Surjektivität von $f
      \end{equation}
      \begin{equation}
        \exists x \in X: f(x) = y$ und damit $
      \end{equation}
      \begin{equation}
        g \circ f(x) = g(f(x)) = g(y) = z. $\phantom{A}$
      \end{equation}
      
      Also $\forall z\in Z \; \exists x \in X: g \circ f (x) = z$ und daher ist $g \circ f$ surjektiv.
    }
    \item
    {
      Wenn $f: X\to Y$ und $g:Y \to Z$ bijektiv sind dann ist $g\circ f: X \to Z$ injektv und surjektiv und somit auch bijektiv wegen Teil (i) und (ii). Wir wollen zeigen, dass $(g \circ f)^{-1} = f^{-1} \circ g^{-1}$.
      
      Sei $z \in Z$. Die Inversen $f^{-1}: Y \to X$ und $g^{-1}: Z \to Y$ erfüllen die Aussagen
      \begin{equation}
        \begin{aligned}
          f(f^{-1}(y)) = y \quad \forall y \in Y \\
          g(g^{-1}(z)) = z \quad \forall z \in Z
        \end{aligned}
      \end{equation}
      
      Weswegen auch $f^{-1}(g^{-1}(z)) \in X$ wohldefiniert ist und 
      \begin{equation}
        g \circ f (f^{-1} (g^{-1}(z))) = g(f(f^{-1}(g^{-1}(z))))=g(g^{-1}(z))=z
      \end{equation}
      
      für alle $z \in Z$ gilt. Damit ist (iii) bewiesen.
    }
  \end{enumerate}
  \vspace{-1em}
\end{prf}

% 1.41
\mce{41}
\begin{imp-ex}[Weitere Eigenschaften von Verknüpfungen]
  Seien $f: X \to Y$ und $g: Y \to Z$ Funktionen.
  
  \begin{enumerate}[label=(\roman*)]
    \item {
      Zeigen Sie, dass $g$ surjektiv ist, falls $g \circ f: X \to Z$ surjektiv ist. Überzeugen Sie sich davon, dass in diesem Fall $f$ nicht unbedingt surjektiv sein muss.
    
      \textbf{Lösung:} Sei $z \in Z$. Falls $g \circ f$ surjektiv, existiert ein $x \in X$ so dass $g \circ f(x) = z = g(f(x))$. Da $f(x) \in Y$ gilt folgendes:
      \begin{equation}
        \forall z \in Z \; \exists y \in Y: g(y) = z
      \end{equation}
      
      Womit wir gezeigt haben, dass auch $g$ surjektiv ist. $f$ muss in diesem Fall nicht zwingend surjektiv sein, da die Menge $Y$ von $f$ nicht unbedingt ausgeschöpft wird. Man muss aber erwähnen, dass die Menge $Y$ von $g$ selbst komplett verwendet wird, auch wenn sie womöglich von der Verknüpfung selbst nicht vollständig beansprucht wird. Funktionen sind auf ihrem Definitionsbereich immer definiert.
    }
    
    \item {
      Zeigen Sie, dass $f$ surjektiv ist, falls $g \circ f$ surjektiv und $g$ injektiv ist.
      
      \textbf{Lösung:} Wir arbeiten mit folgender Voraussetzung, da $g\circ f$ surjektiv ist:
      \begin{equation}
        \forall z \in Z \; \exists x \in X : g \circ f(x) = g(\underbrace{f(x)}_{\in Y}) = z.
      \end{equation}
      
      Die obere Gleichung ist gleichbeutend mit 
      \begin{equation}
        \forall z \in Z \; \exists x \in X \; \exists y \in Y : g \circ f(x) = g(y) = z \und f(x) = y 
      \end{equation}
      
      Da aber $g$ injektiv ist und wir aus Teil (i) wissen, dass $g$ auch zwingend surjektiv sein muss, ist $g$ auch bijektiv. Das heisst, für jedes $z$ aus $Z$ existiert genau ein $y$ aus $Y$ welches 
      \begin{equation}
        g(y) = z \und f(x) = y 
      \end{equation}
      erfüllt. Somit gilt
      \begin{equation}
        \forall y \in Y \; \exists x \in X: f(x) = y.
      \end{equation}  
       
       Somit ist $f$ surjektiv.
    }
    
    \item{ 
      Zeigen Sie, dass $f$ injektiv ist, falls $g \circ f$ injektiv ist. Überzeugen Sie sich davon, dass in diesem Fall $g$ nicht unbedingt injektiv sein muss.
    
      \textbf{Lösung:} Da $g \circ f$ injektiv ist, wissen wir
      \begin{equation}
        \forall x_1, x_2 \in X: x_1 \neq x_2 \implies g (\underbrace{f (x_1)}_{\in Y}) \neq g ( \underbrace{f (x_2)}_{\in Y}).
      \end{equation}
      
      Daraus folgt aber auch, dass $f(x_1) \neq f(x_2)$, womit wir gezeigt haben, dass $f$ auch injektiv ist.
      
      $g$ muss nicht unbedingt injektiv sein, da $Y$ grösser sein kann als $X$.
    }
    
    \item {
      Zeigen Sie, dass $g$ injektiv ist, falls $g \circ f$ injektiv ist und $f$ surjektiv ist.
      
      \textbf{Lösung:} Die Injektivität von $g\circ f$ bedeutet, dass
      \begin{equation}
        \forall x_1, x_2 \in X:
          g(f(x_1)) = g(f(x_2)) \implies x_1 = x_2. $ (Das bedeutet auch, dass $f(x_1) = f(x_2)$ ist)$
      \end{equation}
      
      Respektive mit Zwischenschritten
      \begin{equation}
        \begin{aligned}
          &\forall x_1, x_2 \in X \; \exists y_1, y_2 \in Y: \\
          &\quad ((y_1 = f(x_1) \und y_2 = f(x_2) \und g(y_1) = g(y_2)) \\
          &\quad \implies x_1 = x_2) \implies y_1=y_2
        \end{aligned}
      \end{equation}
      
      Aus der Surjektivität von $f$ wissen wir auch 
      \begin{equation}
        \forall y_1,y_2 \in Y \; \exists x_1, x_2 \in X : y_1 = f(x_1) $ und $ y_2 = f(x_2).
      \end{equation}
      
      Und somit können wir die Werte für $y_1$  und $y_2$ auch direkt verwenden, da sowieso die gesammte Menge $Y$ verwendet wird.
      \begin{equation}
        \begin{aligned}
          &\forall y_1,y_2 \in Y \; \exists x_1, x_2 \in X \\
          &\quad (y_1 = f(x_1) \und y_2 = f(x_2) \und g(y_1) = g(y_2)) \\
          &\quad \implies x_1 = x_2 \implies y_1=y_2
        \end{aligned}
      \end{equation}
      
      vereinfacht sich zu 
      \begin{equation}
        \begin{aligned}
          &\forall y_1,y_2: g(y_1) = g(y_2) \implies y_1=y_2,
        \end{aligned}
      \end{equation}
      
      womit wir gezeigt haben, dass $g$ injektiv ist.
      
      Eine direktere Formulierung wäre wie folgt: Die Injektivität von $g \circ f$ beduetet, dass  
      \begin{equation}
        \forall x_1, x_2 \in X:
        g \circ f (x_1) = g (f(x_1)) = g(f(x_2)) = g \circ f(x_2) \implies x_1 = x_2.
      \end{equation}
      
      Dies bedeutet aber auch, dass im Falle einer Implikation, $f(x_1) = f(x_2)$ ist. Weil $f$ surjektiv ist, gilt. \begin{equation}
        \forall y \in Y \; \exists x \in X: y = f(x).
      \end{equation}
      
      Das heisst, wir können die Menge $Y$ direkt verwenden und kommen zum Schluss dass
      \begin{equation}
        \forall y_1, y_2 \in Y: g(y_1) = g(y_2) \implies y_1 = y_2
      \end{equation} 
      
      Somit haben wir gezeigt, dass $g$ injektiv ist.
    } 
  \end{enumerate}
\end{imp-ex}

% 1.4.4
\subsection{Bild- und Urbildmenge}
Sei $f: X \to Y$ und $A \subseteq X$. Man erinnert sich and die Notation $f(A) :\equiv \{f(x) \mid x \in A\}.$ für das Bild von $A$. Manchmal verwenden wir auch für eine Aussage $P(x)$ die Notation:
\begin{equation}
  \{ f(x) \mid x \in X \und P(x) \} : \equiv
  \left \{ y \mid y \in Y \und \exists x \in X : \left(f(x) = y \und P(x) \right) \right \}
\end{equation}

\mce{48}
\begin{mydef} [Urbilder bezüglich einer Funktion]
  Für $f:X \to Y$ und $B \subseteq Y$ definieren wird das \qt{Urbild} $f^{-1} (B)$ von $B$ unter $f$ als 
  \begin{equation}
    f^{-1} (B) :\equiv \{ x \in X \mid f(x) \in B \}
  \end{equation}
  
  Es gilt beispielsweise $f^{-1} (Y) = X$ und $f^{-1}( \{\})= \{\}$.
\end{mydef}

\begin{imp-ex}[Verhalten von Bildern und Urbilderln unter Mengenoperationen]
  Sei $f:X \to Y$ und $A, A' \subseteq X$ und $B,B' \subseteq Y$.
  \begin{enumerate}[label=(\roman*)]
    \item Zeigen Sie, dass $ f(f^{-1}(B)) \subseteq B$ gilt. Unter welcher Bedingung an $f$ gilt auf jeden Fall Gleichheit?
    
    \textbf{Lösung:} 
    \begin{equation}
      \begin{aligned}
        f(f^{-1}(B)) 
        &= \{ y \in Y \mid \exists x \in \{ x \in X \mid f(x) \in B\} : y= f(x) \} \\ 
        &= \{ y \in Y \mid \exists x \in X \und f(x) \in B \und y = f(x) \} \\ 
        &= \{ y \in Y \mid \exists x \in X \und y = f(x) \in B  \} \\ 
        &= \{ f(x) \mid x \in X \und f(x) \in B  \} \\
        & \subseteq B
      \end{aligned}
    \end{equation}
    
    Es herrscht auf jeden Fall Gleicheit, falls $f$ surjektiv ist, weil $B \subseteq Y$. Das bedeutet, $\forall y \in B \; \exists x \in X: y = f(x)$.
  \end{enumerate}
\end{imp-ex}