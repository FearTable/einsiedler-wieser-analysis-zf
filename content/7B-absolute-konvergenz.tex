\section{Absolute Konvergenz}

% Proposition 7.28
\setcounter{thm}{27}
\begin{thm}[Absolute Konvergenz]
	\label{thm:absolute-konvergenz}
	Eine absolut konvergente Reihe $\sumninfty a_n$ ist auch konvergent und es gilt die verallgemeinerte Dreiecksungleichung
	\begin{equation}
		\left | \sumninfty a_n \right | = \sumninfty | a_n |
	\end{equation}
\end{thm}

\subsection{Hinreichende Kriterien für absolute Konvergenz}

\begin{thm}[Majorantenkriterium von Weierstrass]
	Sei $(a_n)_n \in \nat \times \compl$ und $(b_n)_n \in \nat \times \real$ mit $|a_n| < b_n$ für alle hinreichend grossen $n \in \nat$. Falls $\sumkinfty b_m$ konvergiert, dann ist ist $\sumkinfty a_k$ absolut konvergent.
\end{thm}

\begin{thm}[Cauchy-Wurzelkriterium]
	
	Sei $(a_n)_n$ eine Folge komplexer Zahlen und
	\begin{equation}
		\alpha = \limsup_{n \to \infty} \sqrt[n]{|a_n|} \in \real \cup \infty
	\end{equation}
	
	Dann gilt
	\begin{equation}
		\begin{aligned}
    	\alpha < 1 &\implies \sumkinfty a_n \mytext{ist absolut konvergent} \\
    	\alpha > 1 &\implies \sumkinfty a_n \mytext{ist divergent und $(a_n)_n$ ist keine Nullfolge}
		\end{aligned}
	\end{equation}
\end{thm}

\begin{thm}[D'Alemberts Quotientenkriterium]
Sei $(a_n)_n \in \nat \times \compl$ mit $a_n \neq 0 \; \forall n \in \nat$, so dass
\begin{equation}
	\alpha = \limninfty \frac{|a_{n+1}|}{|a_n|}
\end{equation}

existiert. Dann gilt
\begin{equation}
	\begin{aligned}
		\alpha < 1 &\implies \sumkinfty a_n \mytext{ist absolut konvergent} \\
		\alpha > 1 &\implies \sumkinfty a_n \mytext{ist divergent und $\limninfty a_n \neq 0$}
	\end{aligned}
\end{equation}
\end{thm}

% Wichtige Übung 7.33. Beweisen Sie Korollar 7.32

\subsection{Umordnen von Reihen}

% Satz 7.35 
\setcounter{thm}{34}
\begin{thm}
	Sei $\sumninfty a_n \in \nat \times \compl$ absolut konvergent. Sei $\varphi: \nat \to \nat$ eine Bijektion. Dann ist $\sumninfty a_{\varphi(k)}$ ebenso absolut konvergent und es gilt
	\begin{equation}
		\sumninfty a_n = \sumninfty a_{\varphi(n)}
	\end{equation}
\end{thm}

\subsection{Produkte} 

\begin{thm}
	
	Seien $\sumninfty a_n$ mit $a_n \in \nat \times \compl$ und $\sumninfty b_n$ mit $b_n \in \nat \times \compl$ zwei absolut konvergente Reihen. Sei $\varphi \to \nat \times \nat$ eine biijektive Abbildung. Dann ist
	\begin{equation}
		\sumninfty a_{\varphi(n)_1} b_{\varphi(n)_2} 
	\end{equation}
	
	eine absolut konvergente Reihe, wobei $\varphi (n) = (\varphi(n)_1, \varphi(n)_2)$. Weiters gilt
	\begin{equation}
		\sumninfty a_{\varphi(n)_1} b_{\varphi(n)_2} = \left( \sumninfty a_n \right) \left( \sumninfty b_n \right)
	\end{equation}
	
	Informell ausgedrückt kann man schreiben
	\begin{equation}
		\left( \sumninfty a_n \right) \left( \sumninfty b_n \right) =
	  \sumninfty \left( \sumninfty b_n \right) a_m =
	  \sum_{(m,n) \in \nat^2} a_m b_n
	\end{equation}
\end{thm}


% Korollar 7.37 (Cauchy-Produkt)
\setcounter{thm}{36}
\begin{thm}
	Falls $\sumninfty a_n$ und $\sumninfty b_n$ absolut konvergente Reihen mit komlexen Gliedern sind, dann gilt
	\begin{equation}
		\sum_{n=0}^{\infty} \left ( \sum_{k=0}^{n} a_{n-k} b_k \right )
		=\left( \sumninfty a_n \right) \left( \sumninfty b_n \right)
	\end{equation}
\end{thm}

