\section{Folgen und Konvergenz}

% Definition 5.20
\setcounter{thm}{19}
\begin{mydef}[Folge]
	Eine \qt{Folge} in $X$ ist eine Abbildung $a: \nat \to X$. Man schreibt
	\begin{equation}
    a_n :\equiv a(n) $ als das $n$-te \qt{Folgenglied}, und verwendet folgende Abkürzungen$
	\end{equation}
  \begin{equation}
  	a: \nat \to X \equiv (a_1, a_2, \ldots) \equiv (a_n)_{n\in\nat} \equiv(a_n)_{n=1}^{\infty} \equiv (a_n)_n.
  \end{equation}
  
	Die Menge der Folgen wird auch als $X^\nat$ bezeichnet.
	
	Eine Folge heisst \qt{konstant}, falls $\forall m,n \in \nat: a_n=a_m$ und \qt{schliesslich konstant}, falls 
	\begin{equation}
		\exists M\in \nat: \; \forall m,n \in \nat \; : m,n \geq \nat \implies a_n = a_m
	\end{equation}
\end{mydef}

% Definition 5.21
\begin{mydef}[Konvergenz]
	Sei $(X,d)$ ein metrischer Raum und $(a_n)_n$ eine Folge in $X$.
	
	$(a_n)_n$ \qt{konvergiert} oder \qt{strebt} gegen einen Punkt $A\in X$, falls 
	\begin{equation}
		\forall \epsilon > 0 \; \exists N \in \nat \; \forall n \geq N: (a_n, A) < \epsilon.
	\end{equation}
	
	In diesem Fall nenne wir den Punkt $A$ den \qt{Grenzwert} der Folge und schreiben auch
	\begin{equation}
		\limninfty a_n = A
	\end{equation}
	
	Weiter ist eine Folge in $X$ \qt{konvergent}, falls sie einen Grenzwert besitzt, und \qt{divegent}, falls nicht.
\end{mydef}


\begin{thm}
	Sei $(X,d)$ ein metrischer Raum. Jede konvergente Folge in $X$ besitzt einen eindeutigen Grenzwert.
\end{thm}
\begin{proof} Beweis durch Widerspruch. Seien $A_1, A_2$ zwei verschiedene Grenzwerte einer konvergenten Folge $(a_n)_n$. Sei $\epsilon :\equiv \frac{d(A_1, A_2)}{2} > 0$. Aus der Annahme folgt
	\begin{equation}
		\begin{aligned}
			\exists N_1 \in \nat: d(a_n, A_1) &< \epsilon \quad \forall n \geq N_1 \und \\
			\exists N_2 \in \nat: d(a_n, A_2) &< \epsilon \quad \forall n \geq N_2.
		\end{aligned}
	\end{equation}
	
	Daraus folgt
	\begin{equation}
		 N :\equiv \max \{ N_1, N_2 \} : d(a_n, A_1) < \epsilon \und d(a_n, A_2) < \epsilon \quad \forall n \geq N.
	\end{equation}
	
	Nach der Dreiecksungleichung gilt
	\begin{equation}
		d(A_1, A_2) \leq d(A_1, a_N) + d(a_N, A_2) \; \boxed{<} \;  2\epsilon = d(A_1, A_2)
	\end{equation}
	
	Was ein Widerspruch darstellt. Es kann nicht zwei Grenzwerte geben.
\end{proof}

\emph{Erinnerung:} Für einen metrischen Raum $(X,d)$ und 
$\epsilon > 0$ ist der \qt{$\epsilon$-Ball} oder auch die \qt{$\epsilon$-Umgebung} um $x_0\in X$ durch
\begin{equation}
	B_{\epsilon} (x_0) :\equiv \{ x\in X \mid d(x, x_0) <\epsilon \}
\end{equation}

gegeben. Eine allgemeine Umgebung wird wie folgt definiert.

% Definition 5.24
\setcounter{thm}{23}
\begin{mydef}[Umgebung]
	Eine \qt{Umgebung} $U$ von $x_0 \in X$ ist Teilmenge $U \subseteq X$, die eine $\epsilon$-Umgebung von $x_0$ für ein $\epsilon > 0$ enthält.
\end{mydef}

% Lemma 5.25
\begin{thm}[Indexverschiebung]
	$\forall l \in \nat_0$: $(a_n)_n$ ist konvergent $\iff$ $(a_{n+l})_n$ ist konvergent.
	
	In diesem Fall gilt
	\begin{equation}
		\limninfty a_n = \lim_{n \to \infty} a_{n+l}.
	\end{equation}
	
	Dies gilt für jede Folge $(a_n)_n$ und $(a_{n+l})_n$ in einem metrischen Raum.
\end{thm}


