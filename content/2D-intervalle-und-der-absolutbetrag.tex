\section{Intervalle und der Absolutbetrag}

\subsection{Intervalle}

\mce{41}
\begin{mydef}[Intervalle]
  Seien $a,b \in \real$. Dann wird das
  \begin{itemize}
    \item \emph{abgeschlossene Intervall} $[a,b]$ durch
    $[a,b] :\equiv \{ x \in \real \mid a \leq x \leq b \}$, das
    \item \emph{offene Intervall} $[a,b)$ durch
    $[a,b) :\equiv \{x \in \real \mid a \leq x < b \}$, und das
    \item \emph{(rechts) halboffne Intervall} $(a,b]$ durch
    $(a,b] :\equiv \{ x \in \real \mid a < x \leq b \}$
  \end{itemize}
  definiert. Wenn das Intervall nicht-leer ist, dann wird $a$ der \qt{linke Endpunkt}, $b$ der \qt{rechte Endpunkt}, und $b-a$ die \qt{Länge des Intervalls} genannt. Die Intervalle $(a,b]$, $[a,b)$, $(a,b)$ sind für genau dann nicht-leer, wenn $a<b$ ist. Das Intervall $[a,b]$ ist genau dann nicht-leer, wenn $a\leq b$ ist. Die genannten Intervalle werden auch \qt{beschränkte Intervalle} genannt.
\end{mydef}

\begin{mydef}[Unbeschränkte Intervalle]
  Für $a,b \in \real$ definieren wir die \qt{unbeschränkten abgeschossenen Intervalle}
  \[
    \begin{aligned}
      [a, \infty) &:\equiv \real_{\geq a} :\equiv \{x \in \real \mid a \leq x \} \\
      (-\infty, b) &:\equiv \real_{\leq b} :\equiv \{ x \in \real \mid x \leq b \}
    \end{aligned}
  \]

  und die \qt{unbeschränkten offenen Intervalle}
  \[
    \begin{aligned}
      (a, \infty) &:\equiv \real_{>a} :\equiv \{ x \in \real \mid a < x \} \\
      (-\infty, b) &:\equiv \real_{<b} :\equiv \{ x \in \real \mid x < b \} \\
      (-\infty, \infty) &:\equiv \real
    \end{aligned}
  \]
\end{mydef}

\begin{mydef}[Umgebung eines Punktes] Sei $x \in \real$. Eine Menge, die ein offenes Intervall enthält, in dem $x$ liegt, wird auch eine \qt{Umgebung} oder \qt{Nachbarschaft} von $x$ genannt. Für ein $\delta > 0$ wird das offene Intervall $(x-\delta, x + \delta)$ die \qt{$\delta$-Umgebung} von $x$ genannt.

Beispielsweise wäre also $\rat \cup [-1,1]$ eine Umgebung von $0 \in \real$, weil sie das offene Intervall $(-1,1)$ und den Punkt $0$ enthält. Die Umgebung selbst kann demnach auch abgeschlossen sein.

Falls ein $y \in \real$ in einer $\delta$-Umgebung von $x$ ist, so sagt man auch, dass $y$ \qt{$\delta$-nahe} an $x$ ist.

Definition durch Absolutbetrag: Für ein $\delta > 0$ und ein $x \in \real$ ist die $\delta$-Umgebung von $x$ durch $\{y \in \real \mid \abs{x-y} < \delta \}$ gegeben. $\abs{x-y} = \abs{y-x}$ kann als \qt{Abstand} von $x$ zu $y$ interpretiert werden.

\end{mydef}

\begin{ex}[Verhalten von Intervallen unter Durchscnitt und Vereinigung]
  \phantom{.}
  \begin{enumerate}
    \item Zeigen Sie, dass ein endlicher Schnitt $\cap _{k=1}^{n}I_k$ von Intervallen $I_1, \dots, I_n$ wieder ein Intervall ist, wenn die leere Menge auch als Intervall zugelassen ist. Kann man die Endpunkte eines nicht-leeren Durchschnitts mittels der Endpunkte der ursprünglichen Intervalle beschreiben?
    \item Wann ist die Vereinigung von zwei Intervallen wieder ein Intervall? Was geschieht in diesem Fall, wenn man zwei Intervalle des selben Typs (offen, abgeschlossen, links halboffen, rechs halboffen) vereinigt?
  \end{enumerate}
\end{ex}

\subsection{Der Absolutbetrag auf den reellen Zahlen}
\begin{mydef}[Absolutbetrag] Der \qt{Absolutbetrag} ist due Funktion
  \[| \cdot | : \real \to \real, \quad x \mapsto |x| :\equiv
  \begin{cases}
    x & \text{ falls } x \geq 0 \\
    -x & \text{ falls } x \leq 0
  \end{cases}\]
\end{mydef}

\textbf{Folgerungen.} $\forall x, y \in \real$ gilt
\begin{enumerate}
  \item Es ist $|x| \geq 0$ und $|x| = 0 \iff x=0$. Dies folgt aus der Trichtomie der Reellen Zahlen:\\
  Für $x=0$ gilt $|x| = 0$, für $x>0$ gilt $|x|=x > 0$, und für $x<0$ folgt $|x| = -x > 0$.
  \item $|-x| = |x|$
  \item $|xy| = |x||y|$. Beweis durch Fallunterscheidung.
  \item Für $x\neq 0$ gilt $\left| \tfrac{1}{x} \right | = \tfrac{1}{|x|}$. Beweis: Aus (c) folgt $\left| \tfrac{1}{x} \right | {|x|} = \left| \tfrac{1}{x} x \right |  = 1$.
  \item $\abs{x} \leq y$ $\iff$ $-y \leq x  \leq y$.
  \begin{prf}
    \qt{$\Rightarrow$ Richtung}: Angenommen $\abs{x} \leq y$. Falls $x \geq 0$, dann gilt $-y \leq 0 \leq x = \abs{x} \leq y$. Falls $x < 0$, dann ist $-y \leq -\abs{x} = x < 0 \leq y$ und damit wiederum $-y \leq x \leq y$. \\
    \qt{$\Leftarrow$ Rightung:} Wir bemerken, dass $-y \leq x \leq y$ auch $-y \leq -x \leq y$ und somit auf jeden Fall $\abs{x} \leq y$ impliziert.
  \end{prf}
  \item $\abs{x} < y \iff -y < x < y$.
  \item \emph{Dreiecksungleichung:} $\abs{x+y} \leq \abs{x} + \abs{y}.$
  \begin{prf}
    Wir addieren die Ungleichungen $-\abs{x} \leq x \leq \abs{x}$ und $-\abs{y} \leq y \leq \abs{y}$ und erhalten
    \[
      -(\abs{x} + \abs{y}) \leq x+y \leq \abs{x} + \abs{y}
    \]

    Woraus nach Eigenschaft (e) die Gleichung $\abs{x+y} \leq \abs{\abs{x} + \abs{y}} = \abs{x} + \abs{y}$ folgt.
  \end{prf}
  \item \emph{Umgekehrte Dreiecksungleichung:} $\abs{\abs{x} - \abs{y}} \leq \abs{x-y}$.
\end{enumerate}

\begin{ex}
  Wann gilt Gleichheit in der Dreicecksungleichung oder umgekehrten Dreiecksungleichung?
\end{ex}

\begin{mydef-non}[Vorzeichen/Signum]
  $\forall x \in \real:$ $x = \operatorname{sgn}(x) \abs{x}$, weobei $\operatorname{sgn}(x)$ das \qt{Vorzeichen} (oder \qt{Signum}) von $x$ ist, welches durch
\[
  \operatorname{sgn}: \real \to \{-1,0,1\}, \quad x \mapsto \begin{cases}
    1  & \text{ falls } x > 0 \\
    0  & \text{ falls } x = 0 \\
    -1 & \text{ falls } x < 0
  \end{cases}
\]

definiert ist.
\end{mydef-non}

\begin{ex}[Absolutbetrag und Quadratwurzel] $\forall x \in \real: x^2 = \abs{x}^2$ und $\sqrt{x^2} = \abs{x}$.
\end{ex}

\begin{mydef}[Offene und abgeschlossene Teilmengen]
  Eine Teilmenge $U \subseteq \real$ heisst \qt{offen} (in $\real$), wenn für jedes $x \in U$ ein $\epsilon >0$ exisitiert mit
  \[
    \{ y \in \real \mid \abs{y-x} < \epsilon \} = (x-\epsilon, x + \epsilon) \subseteq U.
  \]

  Eine Teilmenge $A \subseteq \real$ heisst \qt{abgeschlossen} (in $\real$), wenn ihr Komplement $\real \backslash A$ offen ist.
\end{mydef}

\begin{ex}[Offene Intervalle]
  Zeigen Sie, dass eine Teilmene $U \subseteq \real$ genau dann offen ist, wenn $\forall x \in U$ ein offenenes Intervall $I$ mit $x \in I$ und $I \subseteq U$ existiert. Schliessen Sie, dass die offenen (respektive abgeschlossenen) Intervalle auch im Sinne der obigen Definition offen (respektive abgeschlossen) sind.
\end{ex}

\begin{ex}
  Welche der folgenden Teilmengen von $\real$ sind jeweils offen, abgeschlossen oder weder noch?
  \begin{itemize}
    \item Die Teilmengen $\varnothing, \nat, \integer, \real$.
    \item Die Teilmengen $[0,1), (0,1]$ und $(0,1) \cup (2,3)$
  \end{itemize}
\end{ex}

\subsection{Der Absolutbetrag auf den komplexen Zahlen}

\begin{mydef}[Absolutbetrag]
  Der \qt{Absolutbetrag} auf $\compl$ ist gegeben durch
  \[
    \abs{z} = \abs{x + yi} :\equiv \sqrt{x^2 + y^2} = \sqrt{z \overline z} $ für $z=x+yi \in \compl.
  \]

  und stimmt mit dem Absolutbetrag auf $\real$ überein, da für $x \in \real: \sqrt{x \overline x} = \sqrt{x^2} = \abs{x}$.
\end{mydef}

\textbf{Eigenschaften des Absolutbetrags auf $\compl$.}
$\forall z,w \in \compl$ gilt
\begin{enumerate}
  \item \emph{Definitheit:} $\abs{z} > 0$ und $\abs{z}=0$ nur dann, wenn $z = 0$.
  \item \emph{Multiplikativität:} $\abs{zw} = \abs{z}\abs{w}$.
  \item \emph{Dreiecksungleichung:} $\abs{z+w} \leq \abs{z} + \abs{w}$.
  \item \emph{Umgekehrte Dreiecksungleichung:} $\abs{\abs{z}-\abs{w}} \leq \abs{z-w}$.
\end{enumerate}

\mce{52}
\begin{mydef}[Offene Bälle]
  Der \qt{offene Ball} mit Radius $r > 0$ um einen Punkt $z \in \compl$ ist die Menge
  \[
    B_r :\equiv \{ w \in \compl \mid \abs{z-w} <r \}.
  \]
\end{mydef}

\begin{imp-ex}[Durchschnitt von offenen Bällen] Seien $z_1, z_2 \in \compl, r_1 > 0$ und $r_2 > 0$. Für jeden Punkt $z \in B_{r_1}(z_1) \cup B_{r_2}(z_2)$ existiert ein Radius $r>0$, so dass
\[
  B_r(z) \subseteq B_{r_1}(z_1) \cup B_{r_2}(z_2).
\]
\end{imp-ex}

\begin{mydef}[Offene und abgeschlossene Teilmenge von $\compl$] Eine Teilmenge $U \subseteq \compl$ heisst \qt{offen} (in $\compl$), wenn zu jedem Punkt in $U$ ein offener Ball um diesen Punkt existiert, der in $U$ enthalten ist.
\[
  \forall z \in U \subseteq \compl \; \exists r>0: B_r(z) \subseteq U.
\]

Eine Teilmenge $A \subseteq \compl$ heisst \qt{abgeschlossen} (in $\compl$), falls ihr Komplement $\compl \backslash A$ offen ist.
\end{mydef}

