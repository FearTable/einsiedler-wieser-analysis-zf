\section{Erste Konsequenzen der Vollständigkeit}

\subsection{Das Archimedische Prinzip}

\begin{thm}[Das Archimedische Prinzip]
  \label{thm: das archimedische prinzip}
  \phantom{.} \\
  Es gelten folgende Aussagen:
  \begin{enumerate}
    \item Jede nicht-leere, von oben beschränkte Teilmenge von $\integer$ hat ein Maximum.
    \item $\forall x \in \real \; \exists! n \in \integer$: $n \leq x < n+1$. Dies wird als der eindeutig bestimmte \qt{ganzzahlige Anteil} $\lfloor x \rfloor$ einer reellen Zahl $x \in \real$ definiert. Wir erhalten also die Funktion $x \in \real \mapsto \lfloor x \rfloor \in \integer$, die auch \qt{Abrundungsfunktion} genannt wird.
    \item $\forall \epsilon > 0 \; \exists n \in \nat: \frac{1}{n} < \epsilon$.
  \end{enumerate}
\end{thm}

Der \qt{gebrochene Anteil} (oder auch \qt{Nachkommaanteil}) ist $\{ x\} :\equiv x - \lfloor x \rfloor \in [0,1)$. Wir erhalten eine Funktion $x \in \real \mapsto \{ x\} \in [0, 1)$ mit $x=\lfloor x \rfloor + \{ x\} \quad \forall x \in \real$.

\begin{prf}
  Sei $E \subset \integer$, $E \neq \varnothing$ eine von oben beschränkte Teilmenge. Da auch $E \subseteq \real$, folgt aus \ref{thm: supremum} $\exists s_0 = \sup (E)$. Nach \ref{thm: supremum} (c) $\exists n_0 \in E: s_0 \geq n_0 > s-1$. Wobei hier $1$ die Rolle von $\epsilon$ übernommen hat. Es folgt $s_0 < n_0+1$. Somit gilt $\forall m \in E: m \leq s_0 < n_0+1$ und $m \leq n_0$. Daher ist $n_0$ das Maximum von $E$ und Aussage (a) ist bewiesen.

  \textbf{Zu (b):} Sei $x \in \real, x \geq 0$. Dann ist $E :\equiv \{n \in \integer \mid n\leq x\}$ eine von oben beschränkte, nicht leere Teilmenge, weil auf jeden Falls $0 \in E$. Nach (a) hat $E$ ein Maximum, das heisst, es gibt ein maximales $n_0 \in \integer$ mit $n_0 \leq x$. Daraus folgt $x < n_0+1$, weil falls $x \geq n_0+1$ gelten würde, $n_0$ nicht das maximale Elemennt wäre. Somit gilt $n_0 \leq x < n_0+1$ wie in (b).

  Falls $x<0$ ist, dann können wir obigen Fall auf $-x$ anwenden und finden ein $l \in \integer$ mit
  \[
    l \leq -x < l +1
  \]

  Falls $x=-l$ wählen wir $k :\equiv l+1$ weil dann die Ungleichung $l = -x < l+1$ galt. Ansonsten wählen wir $k :\equiv l$ weil dann die Ungleichung $l < -x < l+1$ gegolten hat und wir erhalten somit für dieses gewählte $k \in \{ l, l+1\} \subseteq \integer$ die Gleichung
  \[
    k-1 < -x \leq k
  \]
  Für $n :\equiv -k \in \integer$ erhalten wir
  \[
  \begin{aligned}
    -n-1 < -x \leq -n \\
    n+1 > x \geq n
  \end{aligned}
  \]

  Somit ist die Existenz einer solchen natürlichen Zahl $n$ auch für diesen Fall bewiesen.

  \textbf{Eindeutigkeit:} Für den Beweis der Eindeutigkeit nehmen wir an, dass für $n_1, n_2 \in \integer$ die Ungleichungen $n_1 \leq x < n_1 + 1$ und $n_2 \leq x < n_2+1$ gelten. Daraus folgt $n_1 \leq x < n_2 + 1$ und auch $n_2 \leq x < n_1 + 1$ und damit $n_1 \leq n_2$ und $n_2 \leq n_1$. Desshalb ist $n_1 = n_2$.

  \textbf{Zu (c):} Sei $\epsilon > 0$. Dann gilt auch $\frac{1}{\epsilon} > 0$. Nach Teil (b) gibt es ein $n \in \nat$ mit $n \leq \frac{1}{\epsilon} < n$. Für dieses $n$ gilt aber auch $\frac{1}{n} < \epsilon$, wie in (c) behauptet.
\end{prf}

\begin{ex}[Supremum von Bildmengen]
  Sei $A\subseteq\real$ eine nichtleere Teilmenge von $\real$. Zeigen Sie, dass im Allgemeinen $\sup(\lfloor A \rfloor ) = \lfloor \sup(A) \rfloor$ nicht gilt. Hierbei ist $\lfloor A \rfloor$ das Bild von $A$ unter der Abrundungsfunktion.
\end{ex}

\mce{70}
\begin{thm}[Dichtheit von $\rat$ in $\real$]
  \label{thm: dichtheit von Q in R}
  Zwischen zwei reellen Zahlen $a,b \in \real$ mit $a < b$ gibt es  ein $r \in \rat$ mit $a < r < b$.
\end{thm}
\begin{prf}
  Nach \ref{thm: das archimedische prinzip} (c) $\exists m \in \integer$:
  \[
    \frac{1}{m} < b-a.
  \]

  Ebenso gibt es nach \ref{thm: das archimedische prinzip} (b) ein $n \in \integer$:
  \[
  \begin{aligned}
    &n-1 \leq ma < n \mytext{oder äquivalenterweise }\\
    &\tfrac{n}{m} - \tfrac{1}{m} = \tfrac{n-1}{m} \leq a < \tfrac{n}{m}.
  \end{aligned}
  \]

  Dies führt zu
  \[
  a < \underbrace{\tfrac{n}{m}}_{r} \leq a + \tfrac{1}{m} < a + b- a = b
  \]

  womit die Behauptung bewiesen ist, wenn wir $r:\equiv \frac{n}{m}$ wählen.
\end{prf}

Dies bedeutet auch für jede Umgebung $I$, dass $I \cap \rat \neq \varnothing$

\mce{71}
\begin{ex}[Jede reelle Zahl ist ein Supremum einer Menge von rationalen Zahlen] Zeigen Sie, dass für jedes $x \in \real$ das Supremum von $\{ r \in \rat \mid r < x\}$ gerade $x$ ist.
\end{ex}

\subsection{Häufungspunkte einer Menge}
\mce{73}
\begin{mydef}[Häufungspunkte von Mengen]
  Sei $A \subseteq \real$ und $x_0 \in \real$. Man nennt $x_0$ einen \qt{Häufungspunkt der Menge} $A$, falls $\forall \epsilon > 0 \; \exists a \in A: 0 < \abs{a-x_0} < \epsilon$.

  In andern Worten: Es gibt für einen Häufungspunkt $x_0$ in jeder Umgebung abgesehen von $x_0$ Punkte in $A$. Hierbei muss $x_0$ nicht unbedingt in $A$ liegen. Die Menge $A$ kommt also ihren Häufungspunkten \qt{von aussen} beliebig nahe.
\end{mydef}

\begin{ex}[Endliche Mengen haben keine Häufungspunkte] Zeigen Sie, dass eine endliche Teilmenge $A \subseteq\real$ keine Häufungspunkte hat.
\end{ex}

\begin{thm}[Existenz von Häufungspunkten]
  Sei $A \subseteq \real$ eine beschränkte unendliche Teilmenge. Dann existiert ein Häufungspunkt von $A$ in $\real$.
\end{thm}
\begin{prf}
  Angenommen $m, M \in \real$ sind die beiden Schranken von $A$, so dass $A \subseteq [m,M]$. Wir definieren
  \[
    X :\equiv \{ x \in \real \mid \abs{A \cap (-\infty, x] } < \infty \}
  \]

  Dann ist $m \in X$ da $\abs{A \cap (-\infty,m]} \leq 1$.

  Des Weiteren gilt $x < M \quad \forall x \in X$ also $m \notin X$, denn für ein hypothetisches $x \geq M$ wäre $A \cap (-\infty,x] = A \cap (-\infty, M] = A$ eine unendliche Menge mit $\abs{A \cap (-\infty,x]} = \abs{A \cap (-\infty, M]} = \abs{A}=\infty$. Wäre also zum Beispiel $A$ von der Gestalt $A=[b,c]$ oder $A = (a,b)$ für $b,c \in \real$ dann wäre $X=\{b \}$ eine Menge mit nur einem Element $b$.
  Daher ist $X$ eine beschränkte, nicht-leere Teilmenge von $\real$, womit das Supremum $x_0 :\equiv \sup(X)$ nach \ref{thm: supremum} existiert. Also $X$ kann auch aus Elementen bestehn, die nicht in $A$ enthalten sind.

  Sei nun $\epsilon>0$. Dann existiert ein $x\in X$ mit $x > x_0 - \epsilon$, was zeigt, dass $A\cap (-\infty, x_0 -\epsilon]$ eine endliche Menge ist, da
  \[
    A\cap (-\infty, x_0 -\epsilon] \subseteq A \cap (-\infty, x] \subseteq X.
  \]

  Des Weitern gilt $x_0 + \epsilon \notin X$. Damit gilt
  $
    \abs{A \cap [-\infty, x_0 + \epsilon]} = \infty.
  $

  Es folgt, dass
  \[
    A \cap (x_0 - \epsilon, x_0 + \epsilon] =
    \underbrace{(A \cap (-\infty, x_0 + \epsilon])}_{\text{unendlich}} \, \backslash \,  \underbrace{(A \cap (-\infty, x_0 - \epsilon])}_{\text{endlich}}
  \]

  eine unendliche Menge ist und abgesehen von möglicherweise $x_0$ und
  $x_0 + \epsilon$ noch weitere Punkte besitzen muss. Da $\epsilon > 0$ beliebig war, sehen wir, dass $x_0$ ein Häufungspunkt der Menge $A$ ist.
\end{prf}

\begin{ex}[Alternative Charakterisierung von Häufungspunkten]
  Sei $A \subseteq \real$ und $x_0 \in \real$. Zeigen Sie, dass $x_0$ genau dann ein Häufungspunkt der Menge $A$ ist, wenn $\forall \epsilon > 0$ der Durchschnitt von $A$ mit der $\epsilon$-Umgebung $(x_0 - \epsilon, x_0 + \epsilon)$ unendlich viele Punkte enthält.
\end{ex}

\subsection{Intervallschachtelungsprinzip}

Der durchschnitt von nicht-leereren, ineainder geschachtelten Intervallen $I_1 \subseteq I_2 \cdots $ mit $I_k \subseteq \real$, die kleiner werden, kann durchaus leer sein. Zum Beispiel gilt
\[
  \bigcap_{n=1}^{\infty} [n, \infty) = \varnothing,
  \qquad
  \bigcap_{n=1}^{\infty} \left(0, \frac{1}{n}\right) = \varnothing
\]

\begin{thm}[Intervallschachtelungsprinzip]
  \label{thm: Intervallschachtelungsprinzip}
  Sei $\forall n \in \nat$ ein (nicht-leeres, abgeschlossenes, beschränktes) Intervall
  \[
    I_n :\equiv [a_n, b_n]
  \]

  gegeben, so dass $\forall n,m \in \nat, m \leq n$ die Inklusion $I_m \subseteq I_n$ oder äquivalenterweise die Ungleichungen
  \[
  a_m \leq a_n \leq b_n \leq b_m
  \]

  gelten.
  Dann ist der Durchschnitt
  \[
    \bigcap_{n=1}^{\infty}I_n :\equiv \left[ \sup\{ a_n \mid n \in \nat \}, \inf \{ b_n \mid n \in \nat \}\right]
  \]

  nicht-leer.
\end{thm}\mce{77}
\begin{prf}
  Es gilt $\forall l,m,n \in \nat$, $l \leq m \leq n$:
  \[
    a_l \leq a_m \leq a_n \leq b_n \leq b_m \leq b_l.
  \]

  Dies zeigt, dass $b_m$ eine obere Schranke von $\{a_k \mid k \in \nat \}$ ist, woraus
  \[
    \widetilde{a} :\equiv \sup \{ a_k \mid k \in \nat \} \leq b_m \in \{ b_k \mid k \in \nat \}
  \]

  folgt. Somit ist auch $\{ b_k \mid k \in \nat \}$ von unten beschränkt. Da $m \in \nat$ beliebig ist, können wir die Gleichung oben für $\widetilde{b} :\equiv \inf \{ b_k \mid k \in \nat \}$ etwas umgestalten zu
  \[
  \widetilde{a} = \sup \{ a_k \mid k \in \nat \} \leq \inf \{ b_k \mid k \in \nat \} = \widetilde{b}.
  \]

  Für $x \in \real$ gilt nun die Abfolge von Äquivalenzen
  \[
  \begin{aligned}
    x \in \bigcap_{n=1}^{\infty}[a_n, b_n]
    & \iff \forall n \in \nat: a_n \leq x \leq b_n \\
    & \iff (\forall n \in \nat: a_n \leq x) \wedge (\forall n \in \nat: x \leq b_n) \\
    & \iff \widetilde{a} \leq x \wedge x \leq \widetilde{b},
  \end{aligned}
  \]

  womit $\bigcap_{n=1}^{\infty}[a_n, b_n]=[\widetilde{a}, \widetilde{b}]=\left[ \sup\{ a_k \mid k \in \nat \}, \inf \{ b_k \mid k \in \nat \}\right]$ gilt, was zu beweisen war.
\end{prf}

\mce{79}
\begin{ex}[Charakterisierung von Intervallen]
  \phantom{.}
  \begin{enumerate}
    \item Eine Teilmenge $I \subseteq \real$ ist ein Intervall $\iff$ $\forall x,y \in I, z \in \real: (x \leq z \leq z) \implies z \in I$.
    \item Daraus folgt, dass ein beliebiger Schnitt $\bigcap_{I \in \mathcal{I}} I$ von Intervallen $I \in \mathcal{I}$ ein Intervall ist.
  \end{enumerate}
\end{ex}


\begin{ex}[Zusammenziehende Intervalle]
  Sei zusätzlich wie in \ref{thm: Intervallschachtelungsprinzip} gegeben dass $\inf\{b_n-a_n \mid n \in \nat \} = 0$. Das heisst, dass die Intervalle immer kürzer werden. Dann besteht $\bigcap_{n=1}^{\infty}[a_n, b_n]$ aus nur einem Punkt.
\end{ex}

