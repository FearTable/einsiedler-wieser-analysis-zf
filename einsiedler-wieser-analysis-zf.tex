\documentclass[11pt]{report}

%\usepackage[fleqn]{amsmath}
\usepackage[fleqn]{amsmath}
\usepackage{amssymb}
\usepackage{xcolor}
\usepackage[dvipsnames]{xcolor}
\usepackage[skip = 2pt, parfill]{parskip}
\usepackage{blkarray}
\usepackage{blindtext}
\usepackage{geometry}
\usepackage[explicit]{titlesec} %"big, medium, small, tiny".
\usepackage{xfakebold}
\usepackage{amsthm}
\usepackage{lmodern}
\usepackage{enumitem}
\usepackage{ragged2e}
%\usepackage{kpfonts}
\usepackage{lipsum}
\usepackage{xfrac}
\usepackage{breqn}
\usepackage[colorlinks=false]{hyperref}
\usepackage{xr-hyper}
%\usepackage[fontsize=15pt]{scrextend}
\usepackage{relsize}
\usepackage{enumitem}
\setlist{nosep}


\definecolor{darkteal}{RGB}{0, 41, 59}

%o\everymath{\color{darkteal}}
%\everydisplay{\color{darkteal}}
%\everydisplay{\color{Melon}}

%\hypersetup{
%	colorlinks,
%	linkcolor={red!50!black},
%	citecolor={blue!50!black},
%	urlcolor={blue!80!black}
%}

\hypersetup{
	colorlinks,
	linkcolor={black},
	citecolor={black},
	urlcolor={black}
}

\geometry{
	a4paper,
	total={170mm,257mm},
	left=17mm,
	right=17mm,
	top=15mm,
}

%\setcounter{secnumdepth}{0}
%\renewenvironment{proof}{\underline{\ttfamily Proof:}}{\hfill $\Box$ \\}


%\titlespacing{\chapter}{0pt}{6pt}{3pt}
%\titlespacing{\section}{0pt}{10pt}{3pt}
%\titlespacing{\subsection}{0pt}{13pt}{3pt}
%\titlespacing{\subsubsection}{0pt}{6pt}{3pt}

\titlespacing{\chapter}{0pt}{6pt}{3pt}
\titlespacing{\section}{0pt}{25pt}{3pt}
\titlespacing{\subsection}{0pt}{25pt}{3pt}
\titlespacing{\subsubsection}{0pt}{6pt}{3pt}


%\renewcommand*{\thechapter}{Chapter \arabic{chapter}}
\renewcommand*{\thechapter}{\arabic{chapter}}
%\renewcommand*{\thesection}{\Roman{section}}
\renewcommand*{\thesection}{\arabic{chapter}\Alph{section}}
\renewcommand*{\thesubsection}{\arabic{chapter}\Alph{section}\arabic{subsection}}

%\titleformat*{\chapter}{\ttfamily\fontseries{b}\selectfont\large}
\titleformat{\chapter}[block]
{\bfseries\relscale{1.3}\scshape\selectfont}{\thechapter { } { } #1}{0.5em}{\selectfont\large}


\titlespacing*{\chapter}{0pt}{-19pt}{0pt}


% Old titleformat:
%\titleformat*{\section}{\fontseries{b}\selectfont\large}
%\titleformat*{\subsection}{\fontseries{b}\selectfont\normalsize}
%\titleformat*{\subsubsection}{\fontseries{b}\selectfont\normalsize}
%\titleformat*{\subsubsection}{\fontseries{b}\selectfont\normalsize}

\titleformat{\section}{\bfseries\relscale{1.2}\scshape\selectfont}{}{0em}{\thesection { } #1} %Large

\titleformat{\subsection}{\bfseries\relscale{1.1}\scshape\itshape\selectfont}{}{0em}{#1\  { } (\thesubsection)} %large

\titleformat{\subsubsection}{\relscale{1}\selectfont}{}{0em}{#1\  \thesection}


%\titlespacing{\chapter}{0pt}{\parskip}{-\parskip}
%\titlespacing{\section}{0pt}{\parskip}{-\parskip}
%\titlespacing{\subsection}{0pt}{\parskip}{-\parskip}
%\titlespacing{\subsubsection}{0pt}{\parskip}{-\parskip}

%\usepackage[skip=2pt plus1pt, indent=0pt]{parskip}
% \usepackage[skip=10pt plus1pt, indent=40pt]{parskip}
%\usepackage[small]{titlesec} % "big, medium, small, tiny".

%\setlength{\abovedisplayskip}{0pt}
%\setlength{\belowdisplayskip}{0pt}

\begin{document}

	  
  \linespread{1.2}

  % only displays equation numbers if they are referenced
\mathtoolsset{showonlyrefs,showmanualtags}

\DeclareEmphSequence{\scshape}

% BEGIN custom geometry ************************************
\setlength{\abovedisplayskip}{0.1em}
\setlength{\belowdisplayskip}{0.2em}
\setlength{\abovedisplayshortskip}{0pt}
\setlength{\belowdisplayshortskip}{0pt}
\setlength{\jot}{0pt}

\titlespacing{\chapter}{0pt}{0em}{0.4em}
\titlespacing{\section}{0pt}{1.3em}{0.4em} %15pt, 3pt
\titlespacing{\subsection}{0pt}{1em}{0.4em} %15pt, 3pt
%\titlespacing{\subsubsection}{0pt}{6pt}{3pt}
%\titlespacing*{\chapter}{0pt}{-19pt}{0pt}

%For item, enumerate, description, lists
\setitemize{noitemsep,topsep=0pt,parsep=0pt,partopsep=0pt}
\setenumerate{noitemsep,topsep=0pt,parsep=0pt,partopsep=0pt}
\setdescription{noitemsep,topsep=0pt,parsep=0pt,partopsep=0pt}
\setlist{noitemsep,topsep=0pt,parsep=0pt,partopsep=0pt}

% END custom geometry **************************************

% new mdframed style that places the edges at the corners (.675em):
\mdfdefinestyle{proof-style}{
  skipabove         = 0.2em,% .5\baselineskip ,
  skipbelow         = 0,%.5\baselineskip ,
  leftmargin        = 0.4em ,
  rightmargin       = 0.4em ,
  innermargin       = 0pt ,
  innertopmargin    = 0.4em, %0.6 %.675em ,
  innerleftmargin   = 0.4em, %.675em ,
  innerrightmargin  = 0.4em,
  innerbottommargin = 0.2em, %.675em +3pt,
  hidealllines      = true,
  singleextra       = {
    \draw (O) -- ++(0,.7em) (O) -- ++(.7em,0) ;
    \draw (P-|O) -- ++(0,-.7em) (P-|O) -- ++(.7em,0) ;
  },
  firstextra        = {
    \draw (P-|O) -- ++(0,-.7em) (P-|O) -- ++(.7em,0) ;
  },
  secondextra       = {
    \draw (O) -- ++(0,.7em) (O) -- ++(.7em,0) ;
  },
}

\surroundwithmdframed[style=proof-style]{prf}
\surroundwithmdframed[style=proof-style]{proof}
\newtheorem*{prf}{Proof}

% custom QED Symbol
%\newcommand*\closedbox{%
  %    \leavevmode\hbox to.77778em{\rule{.675em}{.675em}}}
%\let\qedsymbol\closedbox

\renewenvironment{proof}{{\textbf{\scshape \slshape Beweis:}}}{\hfill $\qedsymbol$}
\renewenvironment{prf}{{\textbf{\scshape \slshape Beweis:}}}{\hfill $\qedsymbol$}

% put the new mdframed style around the 'proof' and 'xmpl0 environment:
%\surroundwithmdframed[style=proof]{xmpl}

% BEGIN custom therem and proof environments ***************
\newtheoremstyle{mytheoremstyle} % name
{0.4em}                       % Space above {\topsep} 
{.2em}                        % Space below
{}                            % Body font
{0em}                         % Indent amount
{}                            % Theorem head font {\ttfamily\fontseries{b}\selectfont}
{\textbf{:\,}}       % Punctuation after theorem head
{.2em}                        % Space after theorem head
{{\textbf{\scshape{\thmname{#1}\thmnumber{ #2}}}{\normalfont{\;}\thmnote{({\itshape#3})}}}} 
% Theorem head spec (can be left empty, meaning ‘normal’)

% Define 'thm', 'thm-non', 'mydef', 'mydef-non', 'example', 'example-non'
% environments

\theoremstyle{mytheoremstyle}
\newtheorem{thm}{Thm}[chapter]
\newtheorem*{thm-non}{Thm}

%\theoremstyle{mytheoremstyle}
\newtheorem{mydef}[thm]{Def}
\newtheorem*{mydef-non}{Def}

%\theoremstyle{mytheoremstyle}
\newtheorem{example}[thm]{Example}
\newtheorem*{example-non}{Example}

%\theoremstyle{mytheoremstyle}
\newtheorem{lemma}[thm]{Lemma}
\newtheorem*{lemma-non}{Lemma}

\newtheorem{imp-ex}[thm]{Wichtige Übung}
\newtheorem*{imp-ex-non}{Wichtige Übung}

\newtheorem{xrcs}{Exercise}
\newtheorem*{xrcs-non}{Exercise}

% quotes
\newcommand\qt[1]{\textit{``#1''}}

\newcommand{\gt}{>}
\newcommand{\lt}{<}

\newcommand{\mytext}[1]{\;\; \text{#1}\;\;}
\newcommand\myand{\;\;\, \text{and}\;\;}
\newcommand\myor{\;\;\, \text{or}\;\;}
\newcommand\where{\;\;\, \text{where}\;\;}
\newcommand\whereEach{\;\;\, \text{where each}\;\;}
\newcommand\und{\;\; \text{und}\;\;}
\newcommand\oder{\;\;\, \text{oder}\;\;}
\newcommand\fuer{\;\;\, \text{für}\;\;}
\newcommand\fueralle{\;\;\, \text{für alle}\;\;}
\newcommand\dx{\, dx}
\newcommand\dy{\, dy}
\newcommand\dz{\, dz}
\newcommand\du{\, du}

% SUMS 
\newcommand\sumkinfty{\sum_{k=1}^{\infty}}
\newcommand\sumkn{\sum_{k=1}^{n}}
\newcommand\sumknplone{\sum_{k=1}^{n+1}}
\newcommand\sumkzerotoinfty{\sum_{k=0}^{\infty}}
\newcommand\sumkzeroton{\sum_{k=0}^{n}}

\newcommand\sumkonetilll{\sum_{k=1}^{l}}

\newcommand\sumlinfty{\sum_{l=1}^{\infty}}
\newcommand\sumln{\sum_{l=1}^{n}}
\newcommand\sumninfty{\sum_{n=1}^{\infty}}
\newcommand\sumnM{\sum_{n=1}^{M}}

%LIMITS
\newcommand\limninfty{\lim_{n \to \infty}}

\def\degree{{\operatorname{deg} \,}}

\newcommand{\myspan}[1]{\operatorname{span} (#1)}

%Fast way to write v_1 ... v_n
\newcommand{\oneTillN}[1]{#1_1, \dots, #1_n}
\newcommand{\onetilln}[1]{#1_1, \dots, #1_n}

%Fast way to write v_1 ... v_m
\newcommand{\oneTillM}[1]{#1_1, \dots, #1_m}
\newcommand{\onetillm}[1]{#1_1, \dots, #1_m}

% fast way to write v_1 ... v_{#2}
% usage \onetill{v}{k-1} yields v_1 \dots v_{k-1}
\newcommand{\oneTill}[2]{#1_1, \dots, #1_{#2}}
\newcommand{\onetill}[2]{#1_1, \dots, #1_{#2}}

\newcommand{\kInOneTillM}{k \in \{1, \dots, m \}}
\newcommand{\kinonetillm}{k \in \{1, \dots, m \}}
\newcommand{\kInOneTillN}{k \in \{1, \dots, n \}}
\newcommand{\kinonetilln}{k \in \{1, \dots, n \}}
\newcommand{\kInOneTillP}{k \in \{1, \dots, p \}}
\newcommand{\kinonetillp}{k \in \{1, \dots, p \}}

% abreviation for finite-dimensional vector space
\newcommand{\findimvecpac}{finite-dimensional vector space }
\newcommand{\findimvs}{finite-dimensional vector space }
\newcommand{\fdvs}{finite-dimensional vector space }

%abreviation for linearly independent
\newcommand{\lid}{linearly independent}

%abreviation for linearly independent
\newcommand{\ld}{linearly dependent }

%abreviation for linearly independent
\newcommand{\vs}{vector space }

%abreviation for finite-dimensional
\newcommand{\fd}{{finite-dimensional }}

%abreviation for linear map
\newcommand{\lm}{{linear map }}

%abbreaviation for L(V,W)
\newcommand{\lvw}{{\mathcal{L}(V,W)}}

\newcommand{\linmap}{\mathcal{L}}
\newcommand{\lin}[2]{{\mathcal{L}(#1, #2)}}

\newcommand{\mynull}{\operatorname{null}}

\newcommand{\myrange}{\operatorname{range}}

\newcommand{\even}{\operatorname{even}}
\newcommand{\odd}{\operatorname{odd}}

%\newcommand{\mmatrix}{\mathcal{M}}

% Natural numbers, integers, real numbers, complex numbers:
\newcommand{\nat}{\mathbb{N}}
\newcommand{\integer}{\mathbb{N}}
\newcommand{\real}{\mathbb{R}}
\newcommand{\compl}{\mathbb{C}}
\newcommand{\myF}{\mathbb{F}}

% Polynomial symbol:
\newcommand{\polyn}{\mathcal{P}}

% Matrix symbol:
\newcommand{\mmatrix}{\mathcal{M}}

%\newcommand{\bfemph}[1]{{\ttfamily\fontseries{b}\selectfont #1}}
\newcommand{\bfemph}[1]{{\scshape\relscale{1.1} #1}}

%\newcommand{\basis}[2]{\overbrace{ \myspan{#1_1, \dots #1_{#2}}}^{\text{linearly independent}} }}

\def\myimpl{{ \{black}{\implies}}


\def\bold#1{{\bf #1}}

%\newtheoremstyle{mytheoremstyle} % name
%%{\topsep}                    % Space above
%{0.8em}                    % Space above
%{0em}                        % Space below
%{}                           % Body font
%{0em}                           % Indent amount
%%{\ttfamily\fontseries{b}\selectfont}                   % Theorem head font
%{\bfseries\scshape}                   % Theorem head font
%{:\newline}                          % Punctuation after theorem head
%{.3em}                       % Space after theorem head
%{}  					     % Theorem head spec (can be left empty, meaning ‘normal’)
%
%\theoremstyle{mytheoremstyle}
%\newtheorem{thm}{Theorem}[chapter]
%
%\theoremstyle{mytheoremstyle}
%\newtheorem{mydef}[thm]{Definition}
%\newtheorem*{mydef-non}{Definition}
%
%\theoremstyle{mytheoremstyle}
%\newtheorem{example}[thm]{Example}
%
%
%
%\newtheoremstyle{indented}
%{3pt}% space before
%{3pt}% space after
%{\addtolength{\@totalleftmargin}{3.5em}
%  \addtolength{\linewidth}{-3.5em}
%  \parshape 1 3.5em \linewidth}% body font
%{}% indent
%{\bfseries}% header font
%{.}% punctuation
%{.5em}% after theorem header
%{}% header specification (empty for default)
%\makeatother
%
%\renewenvironment{proof}
%{
%	{
%		\bfseries
%		\scshape
%		\itshape
%		\selectfont
%		Beweis:}
%}
%{
%	\hfill $\Box$ \\
%}

% make counter equal
\newcommand{\mce}[1]{\setcounter{thm}{#1-1}}

\setlength{\abovedisplayskip}{4pt}
\setlength{\belowdisplayskip}{3pt}

\setlength{\abovedisplayskip}{0.1em}
\setlength{\belowdisplayskip}{0.2em}
\setlength{\abovedisplayshortskip}{0pt}
\setlength{\belowdisplayshortskip}{0pt}
\setlength{\jot}{0pt}

  \chapter{Einführung}


  \chapter{Die reellen Zahlen}
	.
	
	\chapter{Funktionen und die reellen Zahlen}
	
  \chapter{Das Riemann-Integral}
	.
	
  \chapter{Metrische Räume, Folgen und Stetigkeit}
	\input{content/5A-normierte-vektorraeume}
	\input{content/5B-metrische-raeume}
	\section{Folgen und Konvergenz}

% Definition 5.20
\setcounter{thm}{19}
\begin{mydef}[Folge]
	Eine \qt{Folge} in $X$ ist eine Abbildung $a: \nat \to X$. Man schreibt
	\begin{equation}
    a_n :\equiv a(n) $ als das $n$-te \qt{Folgenglied}, und verwendet folgende Abkürzungen$
	\end{equation}
  \begin{equation}
  	a: \nat \to X \equiv (a_1, a_2, \ldots) \equiv (a_n)_{n\in\nat} \equiv(a_n)_{n=1}^{\infty} \equiv (a_n)_n.
  \end{equation}
  
	Die Menge der Folgen wird auch als $X^\nat$ bezeichnet.
	
	Eine Folge heisst \qt{konstant}, falls $\forall m,n \in \nat: a_n=a_m$ und \qt{schliesslich konstant}, falls 
	\begin{equation}
		\exists M\in \nat: \; \forall m,n \in \nat \; : m,n \geq \nat \implies a_n = a_m
	\end{equation}
\end{mydef}

% Definition 5.21
\begin{mydef}[Konvergenz]
	Sei $(X,d)$ ein metrischer Raum und $(a_n)_n$ eine Folge in $X$.
	
	$(a_n)_n$ \qt{konvergiert} oder \qt{strebt} gegen einen Punkt $A\in X$, falls 
	\begin{equation}
		\forall \epsilon > 0 \; \exists N \in \nat \; \forall n \geq N: (a_n, A) < \epsilon.
	\end{equation}
	
	In diesem Fall nenne wir den Punkt $A$ den \qt{Grenzwert} der Folge und schreiben auch
	\begin{equation}
		\limninfty a_n = A
	\end{equation}
	
	Weiter ist eine Folge in $X$ \qt{konvergent}, falls sie einen Grenzwert besitzt, und \qt{divegent}, falls nicht.
\end{mydef}


\begin{thm}
	Sei $(X,d)$ ein metrischer Raum. Jede konvergente Folge in $X$ besitzt einen eindeutigen Grenzwert.
\end{thm}
\begin{proof} Beweis durch Widerspruch. Seien $A_1, A_2$ zwei verschiedene Grenzwerte einer konvergenten Folge $(a_n)_n$. Sei $\epsilon :\equiv \frac{d(A_1, A_2)}{2} > 0$. Aus der Annahme folgt
	\begin{equation}
		\begin{aligned}
			\exists N_1 \in \nat: d(a_n, A_1) &< \epsilon \quad \forall n \geq N_1 \und \\
			\exists N_2 \in \nat: d(a_n, A_2) &< \epsilon \quad \forall n \geq N_2.
		\end{aligned}
	\end{equation}
	
	Daraus folgt
	\begin{equation}
		 N :\equiv \max \{ N_1, N_2 \} : d(a_n, A_1) < \epsilon \und d(a_n, A_2) < \epsilon \quad \forall n \geq N.
	\end{equation}
	
	Nach der Dreiecksungleichung gilt
	\begin{equation}
		d(A_1, A_2) \leq d(A_1, a_N) + d(a_N, A_2) \; \boxed{<} \;  2\epsilon = d(A_1, A_2)
	\end{equation}
	
	Was ein Widerspruch darstellt. Es kann nicht zwei Grenzwerte geben.
\end{proof}

\emph{Erinnerung:} Für einen metrischen Raum $(X,d)$ und 
$\epsilon > 0$ ist der \qt{$\epsilon$-Ball} oder auch die \qt{$\epsilon$-Umgebung} um $x_0\in X$ durch
\begin{equation}
	B_{\epsilon} (x_0) :\equiv \{ x\in X \mid d(x, x_0) <\epsilon \}
\end{equation}

gegeben. Eine allgemeine Umgebung wird wie folgt definiert.

% Definition 5.24
\setcounter{thm}{23}
\begin{mydef}[Umgebung]
	Eine \qt{Umgebung} $U$ von $x_0 \in X$ ist Teilmenge $U \subseteq X$, die eine $\epsilon$-Umgebung von $x_0$ für ein $\epsilon > 0$ enthält.
\end{mydef}

% Lemma 5.25
\begin{thm}[Indexverschiebung]
	$\forall l \in \nat_0$: $(a_n)_n$ ist konvergent $\iff$ $(a_{n+l})_n$ ist konvergent.
	
	In diesem Fall gilt
	\begin{equation}
		\limninfty a_n = \lim_{n \to \infty} a_{n+l}.
	\end{equation}
	
	Dies gilt für jede Folge $(a_n)_n$ und $(a_{n+l})_n$ in einem metrischen Raum.
\end{thm}



	
  \chapter{Grenzwerte reeller Folgen und Funktionen}
 	.
 	
	
  \chapter{Reihen, Funktionsfolgen und Potenzreihen}
  \section{Reihen}

\begin{mydef}
	Sei $(a_k)_k \in \nat \times \compl$. Man nennt $a_k$ das \qt{$k$-te Glied} oder den \qt{$k$-ten Summanden} der \qt{(unendlichen) Reihe} 
  \begin{equation}
		\sum_{k=1}^{\infty} a_k. $ Man nennt $
	\end{equation}
	\begin{equation}
		s_n=\sum_{k=1}^{n} a_k 
	\end{equation}
	
	die \qt{$n$-te Partialsumme} der obigen Reihe. Wir nennen die Reihe \qt{konvergent}, falls der Grenzwert
	\begin{equation}
		\sum_{k=1}^\infty a_k :\equiv \lim_{n \to \infty} \sum_{k=1}^{n} a_k = \lim_{n \to \infty} s_n \in \compl
	\end{equation}
	
	existiert. Wir nennen dies den \qt{Wert der Reihe}. Ansonsten nennen wir die Reihe \qt{divergent}.
\end{mydef}

\begin{thm}
	Falls $\sum_{k=1}^{\infty} a_k$ konvergiert, dann ist $(a_n)_n$ eine Nullfolge.
\end{thm}

\setcounter{thm}{5}
\begin{thm}[Linearität]
	Seinen $\sum_{k=1}^{\infty} a_k$ und $\sum_{k=1}^{\infty} b_k$ konvergente Reihen und $\alpha \in \compl$. Dann sind die Reihen  $\sum_{k=1}^{\infty} (a_k + b_k)$ und $\sum_{k=1}^{\infty} \alpha a_k$ konvergent und es gilt
	\begin{equation}
		\sum_{k=1}^{\infty} (a_k + b_k) = \sum_{k=1}^{\infty} a_k + \sum_{k=1}^{\infty} b_k \und \sum_{k=1}^{\infty} (\alpha a_k) = \alpha \sum_{k=1}^{\infty} a_k
	\end{equation}
\end{thm}

\setcounter{thm}{7}
\begin{thm}[Indexverschiebung für Reihen]
	Sei $\sumkinfty a_k$ eine Reihe. Es gilt $\forall M \in \nat$:
	
	Die Reihe $\sum_{k=M}^{\infty} a_k = \sumlinfty a_{l+M-1}$ ist konvergent $\iff$ die Reihe $\sumkinfty a_k$ ist konvergent.
	
	In diesem Fall gilt
	\begin{equation}
		\sumkinfty a_k = \sum_{k=1}^{M-1} a_k + \sum_{k=M}^{\infty} a_k.
	\end{equation}
\end{thm}

\begin{thm}[Zusammenfassen von benachbarten Gliedern]
	Sei $\sumninfty a_n$ eine konvergente Reihe und $(n_k)_k \in \nat \times \nat$ eine streng monoton wachsende Folge (natürlicher Zahlen).
	Sei 
	\begin{equation}
		\begin{aligned}
			A_1 &: \equiv a_1 + \cdots + a_{n_1} \\
			A_k &: \equiv a_{n_{k-1}} + \cdots + a_{n_k} \fuer k \geq 2
		\end{aligned}
	\end{equation}
	Dann gilt
	\begin{equation}
		\sumkinfty A_k = \sumkinfty a_n
	\end{equation}
\end{thm}

\subsection{Reihen mit nicht-negativen Gliedern}

\begin{thm}[Monotone Partialsummen]
	Für eine Reihe $\sumkinfty a_k$ mit $a_k \geq 0$ bilden die Partialsummen $s_n=\sumkn a_k$ eine monoton wachsende Folge. Falls diese Folge $(s_n)_n$ beschränkt ist, dann Konvergiert die Reihe. Ansonsten gilt
	\begin{equation}
		\sumkinfty a_k = \limninfty s_n = \infty.
	\end{equation}	
\end{thm}

\begin{thm}[Vergleichssatz]
	
	Seien $\sumkinfty a_k$ und $\sumkinfty b_k$ zwei Reihen mit der Eigenschaft 
	\begin{equation}
		0\leq a_k \leq b_k. $ Dann gilt $
	\end{equation}
	\begin{equation}
	 	\underbrace{\sumkinfty a_k}_{\text{Minorante von $\sumkinfty b_k$}}  \leq \underbrace{\sumkinfty b_k}_{\text{Majorante von $\sumkinfty a_k$}} 
	\end{equation} 
	
	und insbesondere gelten die Implikationen
	\begin{equation}
		\begin{aligned}
			&\sumkinfty b_k
			&\text{konvergent} 
			&\implies 
			&\sumkinfty a_k 
			&\mytext{konvergent} 
			&\text{(Majorantenkriterium)} \\
			&\sumkinfty a_k
			&\text{divergent} 
			&\implies 
			&\sumkinfty b_k 
			&\mytext{divergent} 
			&\text{(Minorantenkriterium)}
		\end{aligned}
	\end{equation}
	
	Diese beiden Implikationen treffen auch dann zu, wenn $0\leq a_n \leq b_n$ nur für alle hinreichend grossen $n \in \nat$ gilt.
\end{thm}

\setcounter{thm}{15}
\begin{thm}[Verdichtung]
	\label{thm:verdichtung}
	Eine Reihe $\sumkinfty a_k$ mit $a_1 \geq a_2 \cdots \geq 0$ (nicht-negativen, monoton abnehmenden Gliedern) ist konvergent $\iff$ $\sumkinfty 2^k a_{2^k}$ ist konvergent.
\end{thm}
\begin{proof}
	Es gilt $\forall n\in \nat$
	\begin{equation}
		\begin{aligned}
			a_2    &\leq a_2 \leq a_1 \\
			2a_4   &\leq a_3+a_4 \leq 2a_2 \\
			2^2a_8 &\leq a_5+a_6+a_7+a_8 \leq 2^2a_4 \\
			       &\quad \vdots \\
			2^{n} a_{2^{n+1}} & \leq a_{(2^n)+1} + \ldots + a_{2^{n+1}} \leq 2^n a_{2^n}
		\end{aligned}
	\end{equation}
	\begin{equation}
		\implies \sumknplone 2^{k-1} a_{2^l} \leq \sum_{l=2}^{2^{n+1}} a_l \leq \sumkzeroton 2^{k}a_{2^k}
	\end{equation}
	
	Der Rest folgt durch \autoref{thm:verdichtung} und durch den Grenzübergang $n \to \infty$.
\end{proof}


\subsection{Bedingte Konvergenz}
\begin{mydef}
	Eine Reihe $\sumninfty a_n$ mit $a_n \in \compl$ \qt{konvergiert absolut}, falls die Reihe $\sumkinfty |a_n|$ konvergiert. \\
Die Reihe $\sumninfty a_n$ ist \qt{bedingt konvergent}, falls sie konvergiert, aber nicht absolut konvergiert.
\end{mydef}

\setcounter{thm}{20}
\begin{thm}[Riemannscher Umordnungssatz]
	Sei $\sumninfty a_n$ eine bedingt konvergente Reihe mit reelen Gliedern. Dann gibt es zu jedem $A\in \real$ eine bijektive Funktion (eine Umordnung) $\varphi: \nat \to \nat$, so dass die Reihe $\sumninfty a_{\varphi(n)}$ bedingt konvergiert und $\sumninfty a_{\varphi(n)} = A$ ist. Weiters gibt es eine Umordnung der Reihe, die divergiert.	
\end{thm}


\subsection{Alternierende Reihen}

Für eine Folge $(a_n)_n$ positiver Zahlen bezeichnen wir die Reihe $\sumkinfty (-1)^{n+1}a_n$ als \qt{alternierende Reihe}.

\begin{thm}[Leibniz-Kriterium]
	Gegeben sei eine monoton fallende Folge $(a_n)_n$ positiver (reeler?) Zahlen, die gegen Null konvergiert. Dann konvergiert die zugehörige alternierende Reihe $\sumkinfty (-1)^{k+1} a_k$ und es gilt, dass
	\begin{equation}
		\label{eq: fehlerabschaetzung}
		\left | \sumkonetilll (-1)^{k+1} a_k - \sumkinfty (-1)^{k+1} a_k \right | \leq a_{l+1}. \fueralle l\in \nat
	\end{equation}
	
	Weiters ist
	\begin{equation}
		\sum_{k=1}^{2n} (-1)^{k+1} a_k \leq \sum_{k=1}^{\infty} (-1)^{k+1} a_k \leq \sum_{k=1}^{2n-1} (-1)^{k+1}a_k
	\end{equation}
	
	Abschätzungen des Typs (\autoref{eq: fehlerabschaetzung}) werden meist auch Fehlerabschätzungen oder Fehlerschranken bezeichnet. (Intuitiv: Approximation durch Summe bis zu einem gewissen Glied anstatt der Wert der Reihe)
\end{thm}


\subsection{Das Cauchy-Kriterium}

\begin{thm}[Cauchy-Kriterium] Die Reihe $\sumkinfty a_k$ konvergiert $\iff$ $\forall \epsilon > 0 \; \exists N \in \nat$, so dass
	\begin{equation}
		\left | \sum_{k=m}^{n} a_k \right | < \epsilon \fuer n \geq m \geq N
	\end{equation}
\end{thm}
\begin{proof}
	Dies folgt aus dem Cauchy-Kriterium für Folgen angewendet auf die Folge der Partialsummen $s_n=\sumkinfty a_k$, da 
	\begin{equation}
				s_n - s_{m-1} = \sum_{k=1}^n a_k - \sum_{k=1}^{m-1} a_k = \sum_{k=m}^n a_k \fuer n \geq m
	\end{equation}
\end{proof}


  \section{Absolute Konvergenz}

% Proposition 7.28
\setcounter{thm}{27}
\begin{thm}[Absolute Konvergenz]
	\label{thm:absolute-konvergenz}
	Eine absolut konvergente Reihe $\sumninfty a_n$ ist auch konvergent und es gilt die verallgemeinerte Dreiecksungleichung
	\begin{equation}
		\left | \sumninfty a_n \right | = \sumninfty | a_n |
	\end{equation}
\end{thm}

\subsection{Hinreichende Kriterien für absolute Konvergenz}

\begin{thm}[Majorantenkriterium von Weierstrass]
	Sei $(a_n)_n \in \nat \times \compl$ und $(b_n)_n \in \nat \times \real$ mit $|a_n| < b_n$ für alle hinreichend grossen $n \in \nat$. Falls $\sumkinfty b_m$ konvergiert, dann ist ist $\sumkinfty a_k$ absolut konvergent.
\end{thm}

\begin{thm}[Cauchy-Wurzelkriterium]
	
	Sei $(a_n)_n$ eine Folge komplexer Zahlen und
	\begin{equation}
		\alpha = \limsup_{n \to \infty} \sqrt[n]{|a_n|} \in \real \cup \infty
	\end{equation}
	
	Dann gilt
	\begin{equation}
		\begin{aligned}
    	\alpha < 1 &\implies \sumkinfty a_n \mytext{ist absolut konvergent} \\
    	\alpha > 1 &\implies \sumkinfty a_n \mytext{ist divergent und $(a_n)_n$ ist keine Nullfolge}
		\end{aligned}
	\end{equation}
\end{thm}

\begin{thm}[D'Alemberts Quotientenkriterium]
Sei $(a_n)_n \in \nat \times \compl$ mit $a_n \neq 0 \; \forall n \in \nat$, so dass
\begin{equation}
	\alpha = \limninfty \frac{|a_{n+1}|}{|a_n|}
\end{equation}

existiert. Dann gilt
\begin{equation}
	\begin{aligned}
		\alpha < 1 &\implies \sumkinfty a_n \mytext{ist absolut konvergent} \\
		\alpha > 1 &\implies \sumkinfty a_n \mytext{ist divergent und $\limninfty a_n \neq 0$}
	\end{aligned}
\end{equation}
\end{thm}

% Wichtige Übung 7.33. Beweisen Sie Korollar 7.32

\subsection{Umordnen von Reihen}

% Satz 7.35 
\setcounter{thm}{34}
\begin{thm}
	Sei $\sumninfty a_n \in \nat \times \compl$ absolut konvergent. Sei $\varphi: \nat \to \nat$ eine Bijektion. Dann ist $\sumninfty a_{\varphi(k)}$ ebenso absolut konvergent und es gilt
	\begin{equation}
		\sumninfty a_n = \sumninfty a_{\varphi(n)}
	\end{equation}
\end{thm}

\subsection{Produkte} 

\begin{thm}
	
	Seien $\sumninfty a_n$ mit $a_n \in \nat \times \compl$ und $\sumninfty b_n$ mit $b_n \in \nat \times \compl$ zwei absolut konvergente Reihen. Sei $\varphi \to \nat \times \nat$ eine biijektive Abbildung. Dann ist
	\begin{equation}
		\sumninfty a_{\varphi(n)_1} b_{\varphi(n)_2} 
	\end{equation}
	
	eine absolut konvergente Reihe, wobei $\varphi (n) = (\varphi(n)_1, \varphi(n)_2)$. Weiters gilt
	\begin{equation}
		\sumninfty a_{\varphi(n)_1} b_{\varphi(n)_2} = \left( \sumninfty a_n \right) \left( \sumninfty b_n \right)
	\end{equation}
	
	Informell ausgedrückt kann man schreiben
	\begin{equation}
		\left( \sumninfty a_n \right) \left( \sumninfty b_n \right) =
	  \sumninfty \left( \sumninfty b_n \right) a_m =
	  \sum_{(m,n) \in \nat^2} a_m b_n
	\end{equation}
\end{thm}


% Korollar 7.37 (Cauchy-Produkt)
\setcounter{thm}{36}
\begin{thm}
	Falls $\sumninfty a_n$ und $\sumninfty b_n$ absolut konvergente Reihen mit komlexen Gliedern sind, dann gilt
	\begin{equation}
		\sum_{n=0}^{\infty} \left ( \sum_{k=0}^{n} a_{n-k} b_k \right )
		=\left( \sumninfty a_n \right) \left( \sumninfty b_n \right)
	\end{equation}
\end{thm}


	\section{Konvergenz von Funktionsfolgen}
\subsection{Punktweise Konvergenz}
\begin{mydef}[Funktionsfolgen und punktweise Kovergenz]
	Eine reellwertige oder komlexwertige ``Funktionsfolge" ist eine Folge $(f_n)_n$ von Funktionen $f_n: X \to \real$ (oder $f_n: X \to \compl$). 
	
	Wir sagen dass eine Funktion ``punktweise" gegen eine Funktion $f: X \to \real$ (oder $f: X \to \compl$) ``konvergiert", falls $f_n(x) \to f(x)$ für $n \to \infty$ $\; \forall x\in X$.
	
	Wir bezeichnen die Funktion $f$ als den ``punktweisen Grenzwert" (oder auch ``Grenzfunktion" oder ``Limes") der Funktionsfolge $(f_n)_n$
	\begin{equation}
		\forall x \in X \; \forall \epsilon > 0 \; \exists M \in \nat \; \forall n \in \nat : (n > M \implies |f_n(x)-f(x)| < \epsilon) 
	\end{equation}
	gegeben.

\end{mydef}

\subsection{Gleichmässige Konvergenz}

\begin{mydef}[Gleichmässige Konvergenz]
	Wir sagen, $f_n$ ``strebt gleichmässig" gegen $f$ für $n \to \infty$, oder dass $f$ der ``gleichmässige Grenzwert" der Funktionenfolge $(f_n)_n$ ist, falles es zu jedem $\epsilon > 0$ ein $M \in \nat$ gibt, so dass $\forall n > M$ und alle $x\in X$ die Abschätzung
	\begin{equation}
		|f_n(x)-f(x)|<\epsilon
	\end{equation}
	
	gilt. In der Prädikatenlogik ist gleichmässige Konvergenz durch
	\begin{equation}
	  \forall \epsilon > 0 \; \exists M \in \nat \; \forall n\in \nat : \big(n\geq M \implies ( \forall x \in X: |f_n(x) - f(x)| < \epsilon ) \big)
	\end{equation}
	gegeben.
	
\end{mydef}

% Satz 7.48
\setcounter{thm}{47}
\begin{thm}[Gleichmässige Konvergenz und Stetigkeit]
	\label{thm:gleichmaessige-konvergenz-und-stetigkeit} 
	Sei $D\subseteq \compl$ und $f_n : D\to \compl$ eine Funktionsfolge stetiger Funktionen. Falls $(f_n)_n$ gleichmässig gegen $f:d \to \compl$ konvergiert, dann ist $f$ ebenso stetig.
\end{thm}

% Satz 7.49
\begin{thm}
	Sei $[a,b]$ ein kompaktes Intervall und $f_n: [a,b] \to \real$ eine Funktionsfolge Riemann-integrierbarer Funktionen. Falls $(f_n)_n$ gleichmässig gegen $f:[a,b] \to \real$ konvergiert, dann ist $f$ Riemann-integrierbar und
	\begin{thm}
		\begin{equation}
			\limninfty \int_{a}^{b} f_n \dx = \int_{a}^{b} \limninfty f_n \dx = \int_{a}^{b} f \dx
		\end{equation}
	\end{thm}
\end{thm}
	\section{Potenzreihen}

% Definition 7.54
\setcounter{thm}{53}
\begin{mydef}[Potenzreihe]
	$\forall n \in \nat_0$ sei $a_n \in \compl$. Dann ist der formale Ausdruck
	\begin{equation}
		\sumninfty a_n z^n
	\end{equation}
	
	eine ``Potenzreihe" in der Variable $z$
\end{mydef}

\subsection{Konvergenzradius}

\begin{mydef}[Konvergenzradius]
	Der entsprechende ``Konvergenzradius" wird durch
	\begin{equation}
		R=\frac{1}{\limsup_{n \to \infty} \sqrt[n]{|a_n|}}
	\end{equation}
	
	definiert. Wir setzen $\frac{1}{+\infty} = 0$ und hier $\frac{1}{0} = +\infty$
\end{mydef}

\begin{thm}[Über den Konvergenzradius]
	Sei $\sumninfty a_n z^n$ eine Potenzreihe und $R$ ihr Konvergenzradius. Dann konvergiert die Reihe für alle $z \in \compl$ mit $|z| < R$ absolut und divergiert für alle $z \in \compl$ mit $|z| > R$. \\
	Weiters konvergiert die Funktionenfolge $\sum_{j=0}^{n} a_j z^j$ gleichmässig gegen $\sumninfty a_n z^n$ auf jeder Kreisscheibe der Form $B_S(0) = \{ z \in \compl \mid |z| < S\}$ für jedes $S\in (0, R)$. Insbesondere definiert die Potenzreihe die stetige Abbildung 
	\begin{equation}
		z \in B_r (0) \mapsto \sum_{n=0}^{\infty} a_n z^n \in \compl
	\end{equation}
	
\end{thm}

\subsection{Addition und Multiplikation}

\begin{thm}[Summe und Produkte]
	Seien $\sumninfty a_n z^n$ und $\sumninfty b_n z^n$ zwei Potenzreihen mit Konvergenzradius $R_a$ respektive $R_b$. Dann gilt $\forall z \in \compl$ mit $|z| < \min \{R_a, R_b \}$
	\begin{equation}
		\begin{aligned}
			\sumninfty a_n z^n  \sumninfty b_n z^n  &= \sumninfty (a_n + b_n) z^n \\
			\left( \sumninfty a_n z^n \right) \left( \sumninfty b_n z^n \right)
			 & = \sumninfty \left( \sum_{k=0}^{n} a_{n-k} b_k z^n \right) 
		\end{aligned}
	\end{equation}
	
	Insbesondere ist der Konvergenzradius der Potenzreihen auf der recheten Seite mindestens $\min \{ R_a, R_b \}$
\end{thm}

\end{document}
