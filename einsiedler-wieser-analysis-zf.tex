% !TeX program = xelatex
\documentclass[10pt]{report}
\usepackage{listings}
\usepackage{matlab-prettifier}
\usepackage[fleqn]{amsmath}
\usepackage{amssymb}
\usepackage{xcolor}
\usepackage[ngerman]{babel}
\usepackage[dvipsnames]{xcolor}
\usepackage[skip = 2pt, parfill]{parskip}
\usepackage{blkarray}
\usepackage{blindtext}
\usepackage{geometry}
\usepackage[explicit]{titlesec} %"big, medium, small, tiny".
%\usepackage{xfakebold}
\usepackage{amsthm}
%\usepackage{lmodern}
\usepackage{ragged2e}
%\usepackage[rmx]{kpfonts}
%\usepackage{concmath}
\usepackage{lipsum}
\usepackage{xfrac}
%\usepackage{breqn}
\usepackage[colorlinks=false]{hyperref}
\usepackage{xr-hyper}
%\usepackage[fontsize=10pt]{scrextend}
\usepackage{relsize}
\usepackage{enumitem}
\usepackage{etoolbox}
\usepackage{changepage}
\usepackage{mathtools}
\usepackage[shortcuts]{extdash}
\usepackage[framemethod=tikz]{mdframed}
%\usepackage[T1]{fontenc}
%\usepackage{inconsolata}
\usepackage[most]{tcolorbox}

\usepackage{fontspec}
\usepackage{unicode-math}
%\usepackage{xltxtra}

%Hyphenation rules
%--------------------------------------
\usepackage{hyphenat}

\setlist{nosep}


\definecolor{darkteal}{RGB}{0, 41, 59}

\hypersetup{
	colorlinks,
	linkcolor={black},
	citecolor={black},
	urlcolor={black}
}

\geometry{
	a4paper,
	total={170mm,257mm},
	left=22mm,
	right=22mm,
	top=15mm,
}

\linespread{1.1}

% LaTeX
%\setmainfont{Latin Modern Roman}
%\setmathfont{Latin Modern Math}
%

\setmainfont{TeX Gyre Bonum} % nice, but without bold scshape
\setmathfont{TeX Gyre Bonum Math}
%
%\setmainfont{TeX Gyre Pagella}
%\setmathfont{TeX Gyre Pagella Math}
%
%\setmainfont{TeX Gyre Schola} %nicer than the one above
%\setmathfont{TeX Gyre Schola Math}
%
%\setmainfont{TeX Gyre Termes} %meh, but much better than standard. Supports Bolds scshape!!
%\setmathfont{TeX Gyre Termes Math}
%
%\setmainfont{DejaVu Serif}
%\setmathfont{TeX Gyre DejaVu Math}

%\setmainfont{FiraGO}
%\setmathfont{FiraMath-Regular}

\begin{document}

  % only displays equation numbers if they are referenced
\mathtoolsset{showonlyrefs,showmanualtags}

\DeclareEmphSequence{\scshape}

% BEGIN custom geometry ************************************
\setlength{\abovedisplayskip}{0.1em}
\setlength{\belowdisplayskip}{0.2em}
\setlength{\abovedisplayshortskip}{0pt}
\setlength{\belowdisplayshortskip}{0pt}
\setlength{\jot}{0pt}

\titlespacing{\chapter}{0pt}{0em}{0.4em}
\titlespacing{\section}{0pt}{1.3em}{0.4em} %15pt, 3pt
\titlespacing{\subsection}{0pt}{1em}{0.4em} %15pt, 3pt
%\titlespacing{\subsubsection}{0pt}{6pt}{3pt}
%\titlespacing*{\chapter}{0pt}{-19pt}{0pt}

%For item, enumerate, description, lists
\setitemize{noitemsep,topsep=0pt,parsep=0pt,partopsep=0pt}
\setenumerate{noitemsep,topsep=0pt,parsep=0pt,partopsep=0pt}
\setdescription{noitemsep,topsep=0pt,parsep=0pt,partopsep=0pt}
\setlist{noitemsep,topsep=0pt,parsep=0pt,partopsep=0pt}

% END custom geometry **************************************

% new mdframed style that places the edges at the corners (.675em):
\mdfdefinestyle{proof-style}{
  skipabove         = 0.2em,% .5\baselineskip ,
  skipbelow         = 0,%.5\baselineskip ,
  leftmargin        = 0.4em ,
  rightmargin       = 0.4em ,
  innermargin       = 0pt ,
  innertopmargin    = 0.4em, %0.6 %.675em ,
  innerleftmargin   = 0.4em, %.675em ,
  innerrightmargin  = 0.4em,
  innerbottommargin = 0.2em, %.675em +3pt,
  hidealllines      = true,
  singleextra       = {
    \draw (O) -- ++(0,.7em) (O) -- ++(.7em,0) ;
    \draw (P-|O) -- ++(0,-.7em) (P-|O) -- ++(.7em,0) ;
  },
  firstextra        = {
    \draw (P-|O) -- ++(0,-.7em) (P-|O) -- ++(.7em,0) ;
  },
  secondextra       = {
    \draw (O) -- ++(0,.7em) (O) -- ++(.7em,0) ;
  },
}

\surroundwithmdframed[style=proof-style]{prf}
\surroundwithmdframed[style=proof-style]{proof}
\newtheorem*{prf}{Proof}

% custom QED Symbol
%\newcommand*\closedbox{%
  %    \leavevmode\hbox to.77778em{\rule{.675em}{.675em}}}
%\let\qedsymbol\closedbox

\renewenvironment{proof}{{\textbf{\scshape \slshape Beweis:}}}{\hfill $\qedsymbol$}
\renewenvironment{prf}{{\textbf{\scshape \slshape Beweis:}}}{\hfill $\qedsymbol$}

% put the new mdframed style around the 'proof' and 'xmpl0 environment:
%\surroundwithmdframed[style=proof]{xmpl}

% BEGIN custom therem and proof environments ***************
\newtheoremstyle{mytheoremstyle} % name
{0.4em}                       % Space above {\topsep} 
{.2em}                        % Space below
{}                            % Body font
{0em}                         % Indent amount
{}                            % Theorem head font {\ttfamily\fontseries{b}\selectfont}
{\textbf{:\,}}       % Punctuation after theorem head
{.2em}                        % Space after theorem head
{{\textbf{\scshape{\thmname{#1}\thmnumber{ #2}}}{\normalfont{\;}\thmnote{({\itshape#3})}}}} 
% Theorem head spec (can be left empty, meaning ‘normal’)

% Define 'thm', 'thm-non', 'mydef', 'mydef-non', 'example', 'example-non'
% environments

\theoremstyle{mytheoremstyle}
\newtheorem{thm}{Thm}[chapter]
\newtheorem*{thm-non}{Thm}

%\theoremstyle{mytheoremstyle}
\newtheorem{mydef}[thm]{Def}
\newtheorem*{mydef-non}{Def}

%\theoremstyle{mytheoremstyle}
\newtheorem{example}[thm]{Example}
\newtheorem*{example-non}{Example}

%\theoremstyle{mytheoremstyle}
\newtheorem{lemma}[thm]{Lemma}
\newtheorem*{lemma-non}{Lemma}

\newtheorem{imp-ex}[thm]{Wichtige Übung}
\newtheorem*{imp-ex-non}{Wichtige Übung}

\newtheorem{xrcs}{Exercise}
\newtheorem*{xrcs-non}{Exercise}

% quotes
\newcommand\qt[1]{\textit{``#1''}}

\newcommand{\gt}{>}
\newcommand{\lt}{<}

\newcommand{\mytext}[1]{\;\; \text{#1}\;\;}
\newcommand\myand{\;\;\, \text{and}\;\;}
\newcommand\myor{\;\;\, \text{or}\;\;}
\newcommand\where{\;\;\, \text{where}\;\;}
\newcommand\whereEach{\;\;\, \text{where each}\;\;}
\newcommand\und{\;\; \text{und}\;\;}
\newcommand\oder{\;\;\, \text{oder}\;\;}
\newcommand\fuer{\;\;\, \text{für}\;\;}
\newcommand\fueralle{\;\;\, \text{für alle}\;\;}
\newcommand\dx{\, dx}
\newcommand\dy{\, dy}
\newcommand\dz{\, dz}
\newcommand\du{\, du}

% SUMS 
\newcommand\sumkinfty{\sum_{k=1}^{\infty}}
\newcommand\sumkn{\sum_{k=1}^{n}}
\newcommand\sumknplone{\sum_{k=1}^{n+1}}
\newcommand\sumkzerotoinfty{\sum_{k=0}^{\infty}}
\newcommand\sumkzeroton{\sum_{k=0}^{n}}

\newcommand\sumkonetilll{\sum_{k=1}^{l}}

\newcommand\sumlinfty{\sum_{l=1}^{\infty}}
\newcommand\sumln{\sum_{l=1}^{n}}
\newcommand\sumninfty{\sum_{n=1}^{\infty}}
\newcommand\sumnM{\sum_{n=1}^{M}}

%LIMITS
\newcommand\limninfty{\lim_{n \to \infty}}

\def\degree{{\operatorname{deg} \,}}

\newcommand{\myspan}[1]{\operatorname{span} (#1)}

%Fast way to write v_1 ... v_n
\newcommand{\oneTillN}[1]{#1_1, \dots, #1_n}
\newcommand{\onetilln}[1]{#1_1, \dots, #1_n}

%Fast way to write v_1 ... v_m
\newcommand{\oneTillM}[1]{#1_1, \dots, #1_m}
\newcommand{\onetillm}[1]{#1_1, \dots, #1_m}

% fast way to write v_1 ... v_{#2}
% usage \onetill{v}{k-1} yields v_1 \dots v_{k-1}
\newcommand{\oneTill}[2]{#1_1, \dots, #1_{#2}}
\newcommand{\onetill}[2]{#1_1, \dots, #1_{#2}}

\newcommand{\kInOneTillM}{k \in \{1, \dots, m \}}
\newcommand{\kinonetillm}{k \in \{1, \dots, m \}}
\newcommand{\kInOneTillN}{k \in \{1, \dots, n \}}
\newcommand{\kinonetilln}{k \in \{1, \dots, n \}}
\newcommand{\kInOneTillP}{k \in \{1, \dots, p \}}
\newcommand{\kinonetillp}{k \in \{1, \dots, p \}}

% abreviation for finite-dimensional vector space
\newcommand{\findimvecpac}{finite-dimensional vector space }
\newcommand{\findimvs}{finite-dimensional vector space }
\newcommand{\fdvs}{finite-dimensional vector space }

%abreviation for linearly independent
\newcommand{\lid}{linearly independent}

%abreviation for linearly independent
\newcommand{\ld}{linearly dependent }

%abreviation for linearly independent
\newcommand{\vs}{vector space }

%abreviation for finite-dimensional
\newcommand{\fd}{{finite-dimensional }}

%abreviation for linear map
\newcommand{\lm}{{linear map }}

%abbreaviation for L(V,W)
\newcommand{\lvw}{{\mathcal{L}(V,W)}}

\newcommand{\linmap}{\mathcal{L}}
\newcommand{\lin}[2]{{\mathcal{L}(#1, #2)}}

\newcommand{\mynull}{\operatorname{null}}

\newcommand{\myrange}{\operatorname{range}}

\newcommand{\even}{\operatorname{even}}
\newcommand{\odd}{\operatorname{odd}}

%\newcommand{\mmatrix}{\mathcal{M}}

% Natural numbers, integers, real numbers, complex numbers:
\newcommand{\nat}{\mathbb{N}}
\newcommand{\integer}{\mathbb{N}}
\newcommand{\real}{\mathbb{R}}
\newcommand{\compl}{\mathbb{C}}
\newcommand{\myF}{\mathbb{F}}

% Polynomial symbol:
\newcommand{\polyn}{\mathcal{P}}

% Matrix symbol:
\newcommand{\mmatrix}{\mathcal{M}}

%\newcommand{\bfemph}[1]{{\ttfamily\fontseries{b}\selectfont #1}}
\newcommand{\bfemph}[1]{{\scshape\relscale{1.1} #1}}

%\newcommand{\basis}[2]{\overbrace{ \myspan{#1_1, \dots #1_{#2}}}^{\text{linearly independent}} }}

\def\myimpl{{ \{black}{\implies}}


\def\bold#1{{\bf #1}}

%\newtheoremstyle{mytheoremstyle} % name
%%{\topsep}                    % Space above
%{0.8em}                    % Space above
%{0em}                        % Space below
%{}                           % Body font
%{0em}                           % Indent amount
%%{\ttfamily\fontseries{b}\selectfont}                   % Theorem head font
%{\bfseries\scshape}                   % Theorem head font
%{:\newline}                          % Punctuation after theorem head
%{.3em}                       % Space after theorem head
%{}  					     % Theorem head spec (can be left empty, meaning ‘normal’)
%
%\theoremstyle{mytheoremstyle}
%\newtheorem{thm}{Theorem}[chapter]
%
%\theoremstyle{mytheoremstyle}
%\newtheorem{mydef}[thm]{Definition}
%\newtheorem*{mydef-non}{Definition}
%
%\theoremstyle{mytheoremstyle}
%\newtheorem{example}[thm]{Example}
%
%
%
%\newtheoremstyle{indented}
%{3pt}% space before
%{3pt}% space after
%{\addtolength{\@totalleftmargin}{3.5em}
%  \addtolength{\linewidth}{-3.5em}
%  \parshape 1 3.5em \linewidth}% body font
%{}% indent
%{\bfseries}% header font
%{.}% punctuation
%{.5em}% after theorem header
%{}% header specification (empty for default)
%\makeatother
%
%\renewenvironment{proof}
%{
%	{
%		\bfseries
%		\scshape
%		\itshape
%		\selectfont
%		Beweis:}
%}
%{
%	\hfill $\Box$ \\
%}

% make counter equal
\newcommand{\mce}[1]{\setcounter{thm}{#1-1}}

\setlength{\abovedisplayskip}{4pt}
\setlength{\belowdisplayskip}{3pt}

\setlength{\abovedisplayskip}{0.1em}
\setlength{\belowdisplayskip}{0.2em}
\setlength{\abovedisplayshortskip}{0pt}
\setlength{\belowdisplayshortskip}{0pt}
\setlength{\jot}{0pt}

  \chapter{Einführung}
  \section{Mengenlehre und Abbildungen}
\subsection{Naive Mengenlehre}

% 1.22 Disjunktheit
\begin{mydef}[Disjunktheit]

  $A,B$ heissen \qt{disjunkt}, falls $A\cup B = \{ \}$.
  In diesem Fall wird $A\cup B$ \qt{disjunkte Vereinigung} genannt und aus als $A \sqcup B$ geschrieben. Für eine Kollektion $\mathcal{A}$ von Mengen, sagen wir, dass die Mengen in $A$ \qt{paarweise disjunkt} sind, falls
  \begin{equation}
    \forall A_1, A_2 \in \mathcal{A} $ mit $ A_1\neq A_2 $ gilt $ A_1 \cup A_2 = \{ \}.
  \end{equation}

  Die Vereinigung der Mengen in $\mathcal{A}$ wird dann auch \qt{disjunkte Vereinigung} genannt und wird in diesem fall auch als
  \begin{equation}
    \bigsqcup_{A\in\mathcal{A}} A :\equiv \bigcup_{A\in\mathcal{A}} A
  \end{equation}

  geschrieben.
\end{mydef}

% 1.23 Kartesisches Produkt
\begin{mydef}[Kartesisches Produkt]
  Das \qt{kartesische Produkt} $X\times Y$ ist die Menge aller geordneter Paare $(x,y)$, wobei $x\in X$ und $y \in Y$. In Symbolen,
  \begin{equation}
    X \times Y :\equiv \{(x,y) \mid x \in X $ und $ y \in Y \}. $ Allgemeiner definiert: $
  \end{equation}
  \begin{equation}
    X_1 \times \cdots \times X_m :\equiv \{ (x_1, \ldots, x_m) \mid x_1 \in X_1, \ldots, x_m \in X_m \}.
  \end{equation}
  % TODO: allgemeiner
\end{mydef}

% 1.27 Potenzmenge
\begin{mydef}[Potenzmenge $\mathcal{P}$] Für eine Menge $X$ wird ihre \qt{Potenzmenge} wie folgt definiert
  \begin{equation}
    \mathcal{P}(X) :\equiv \{Q \mid Q \subseteq X \}
  \end{equation}
\end{mydef}

% 1.4.2
\subsection{Abbildungen}

% 1.31
\begin{mydef}[Funktionen und erste dazugehörige Begriffe]
  Falls $f:X \to Y$ eine Funktion und $A\subseteq X$ ist, dann definieren wird die \qt{Einschränkung} $\left. f \right | _{A} : A \to Y$ durch
  \begin{equation}
    \left. f \right | _{A} (x) :\equiv f(x)  \quad \forall x \in A
  \end{equation}

  Des Weiteren definieren wir das \qt{Bild der Teilmenge $A$} unter $f$ als
  \begin{equation}
    f(A) = \left. f \right |_{A}(A) :\equiv \{y \in Y \mid \exists x \in A : y=f(x) \} = \{y \mid y \in Y \und \exists x \in A : y=f(x) \}
  \end{equation}
\end{mydef}

Für eine Funktion schreibt man $f: X \to Y, x \mapsto f(x)$ oder $x \in X \mapsto f(x) \in Y$, also zum Beispiel $x \in \real \mapsto x^2 \in \real$. Wir sprechen \qt{$\mapsto$} unter Anderem als \qt{wird abgebildet auf} aus.

% 1.33
\begin{mydef}[Drei Eigenschaften von Funktionen]
  Sei $f: X \to Y$ eine Funktion.
  \begin{itemize}
    \item $f$ heisst \qt{injektiv} oder eine \qt{Injektion} falls $\forall x_1, x_2 \in X$ gilt, dass
    \begin{equation}
      f(x_1) = f(x_2) \implies x_1 = x_2 $ oder$
    \end{equation}
    \begin{equation}
      x_1 \neq x_2 \implies f(x_1) \neq f(x_2).
    \end{equation}

    \item $f$ heisst \qt{surjektiv},eine \qt{Surjektion} oder eine Funktion von \qt{X auf Y}, falls
    \begin{equation}
      \forall y \in Y \; \exists x \in X: f(x) = y
    \end{equation}

    Also falls $f(X) = Y$.

    \item $f$ heisst \qt{bijektiv}, eine \qt{Bijektion} oder eine \qt{eineindeutige Abbildung}, falls $f$ surjektiv und injektiv ist.
  \end{itemize}
\end{mydef}

% 1.40
\begin{lemma} [Eigenschaften von Verknüpfungen]
  Seien $f: X \to Y$ und $g: Y \to Z$ Funktionen.
  \begin{enumerate}[label=(\roman*)]
    \item Falls $f$ und $g$ injektiv sind, dann ist auch $g \circ f : X \to Z$ injektiv.
    \item Falls $f$ und $g$ surjektiv sind, dann ist auch $g \circ f : X \to Z$ surjektiv.
    \item Falls $f$ und $g$ bijektiv sind, dann ist auch $g \circ f : X \to Z$ bijektiv und es gilt
    \begin{equation}
      (g \circ f)^{-1} = f^{-1} \circ g^{-1}
    \end{equation}
  \end{enumerate}

\end{lemma}
\begin{prf}
  %\phantom{.}
  \begin{enumerate}[label=(\roman*)]
    \item
    {
      Seien $f: X\to Y$ und $g:Y \to Z$ injektiv und $ x_1, x_2 \in X$ beliebig gewählt, sodass
      \begin{equation}
        g \circ f(x_1) = g \circ f(x_2).
      \end{equation}

      Aus der Definition der Verknüpfung $g(f(x_1)) = g(f(x_2))$ und weil $g$ injektiv ist folgt, dass \begin{equation}
        f(x_1) = f(x_2). $ Dies impliziert aufgrund der Injektivität von $f$ ebenso, dass $
      \end{equation}
      \begin{equation}
        x_1 = x_2. $ Damit ist auch $g \circ f$ injektiv.$
      \end{equation}
    }
    \item
    {
      Angenommen $f: X\to Y$ und $g:Y \to Z$ sind surjektiv und $z \in Z$ ein beliebiges Element. Aus der Surjektivität von g folgt
      \begin{equation}
        \exists y \in Y: g(y) = z. $ Ebenso gilt aufgrund der Surjektivität von $f
      \end{equation}
      \begin{equation}
        \exists x \in X: f(x) = y$ und damit $
      \end{equation}
      \begin{equation}
        g \circ f(x) = g(f(x)) = g(y) = z. $\phantom{A}$
      \end{equation}

      Also $\forall z\in Z \; \exists x \in X: g \circ f (x) = z$ und daher ist $g \circ f$ surjektiv.
    }
    \item
    {
      Wenn $f: X\to Y$ und $g:Y \to Z$ bijektiv sind dann ist $g\circ f: X \to Z$ injektv und surjektiv und somit auch bijektiv wegen Teil (i) und (ii). Wir wollen zeigen, dass $(g \circ f)^{-1} = f^{-1} \circ g^{-1}$.

      Sei $z \in Z$. Die Inversen $f^{-1}: Y \to X$ und $g^{-1}: Z \to Y$ erfüllen die Aussagen
      \begin{equation}
        \begin{aligned}
          f(f^{-1}(y)) = y \quad \forall y \in Y \\
          g(g^{-1}(z)) = z \quad \forall z \in Z
        \end{aligned}
      \end{equation}

      Weswegen auch $f^{-1}(g^{-1}(z)) \in X$ wohldefiniert ist und
      \begin{equation}
        g \circ f (f^{-1} (g^{-1}(z))) = g(f(f^{-1}(g^{-1}(z))))=g(g^{-1}(z))=z
      \end{equation}

      für alle $z \in Z$ gilt. Damit ist (iii) bewiesen.
    }
  \end{enumerate}
  \vspace{-1em}
\end{prf}

% 1.41
\mce{41}
\begin{imp-ex}[Weitere Eigenschaften von Verknüpfungen]
  Seien $f: X \to Y$ und $g: Y \to Z$ Funktionen.

  \begin{enumerate}[label=(\roman*)]
    \item {
      Zeigen Sie, dass $g$ surjektiv ist, falls $g \circ f: X \to Z$ surjektiv ist. Überzeugen Sie sich davon, dass in diesem Fall $f$ nicht unbedingt surjektiv sein muss.

      \textbf{Lösung:} Sei $z \in Z$. Falls $g \circ f$ surjektiv, existiert ein $x \in X$ so dass $g \circ f(x) = z = g(f(x))$. Da $f(x) \in Y$ gilt folgendes:
      \begin{equation}
        \forall z \in Z \; \exists y \in Y: g(y) = z
      \end{equation}

      Womit wir gezeigt haben, dass auch $g$ surjektiv ist. $f$ muss in diesem Fall nicht zwingend surjektiv sein, da die Menge $Y$ von $f$ nicht unbedingt ausgeschöpft wird. Man muss aber erwähnen, dass die Menge $Y$ von $g$ selbst komplett verwendet wird, auch wenn sie womöglich von der Verknüpfung selbst nicht vollständig beansprucht wird. Funktionen sind auf ihrem Definitionsbereich immer definiert.
    }

    \item {
      Zeigen Sie, dass $f$ surjektiv ist, falls $g \circ f$ surjektiv und $g$ injektiv ist.

      \textbf{Lösung:} Wir arbeiten mit folgender Voraussetzung, da $g\circ f$ surjektiv ist:
      \begin{equation}
        \forall z \in Z \; \exists x \in X : g \circ f(x) = g(\underbrace{f(x)}_{\in Y}) = z.
      \end{equation}

      Die obere Gleichung ist gleichbeutend mit
      \begin{equation}
        \forall z \in Z \; \exists x \in X \; \exists y \in Y : g \circ f(x) = g(y) = z \und f(x) = y
      \end{equation}

      Da aber $g$ injektiv ist und wir aus Teil (i) wissen, dass $g$ auch zwingend surjektiv sein muss, ist $g$ auch bijektiv. Das heisst, für jedes $z$ aus $Z$ existiert genau ein $y$ aus $Y$ welches
      \begin{equation}
        g(y) = z \und f(x) = y
      \end{equation}
      erfüllt. Somit gilt
      \begin{equation}
        \forall y \in Y \; \exists x \in X: f(x) = y.
      \end{equation}

       Somit ist $f$ surjektiv.
    }

    \item{
      Zeigen Sie, dass $f$ injektiv ist, falls $g \circ f$ injektiv ist. Überzeugen Sie sich davon, dass in diesem Fall $g$ nicht unbedingt injektiv sein muss.

      \textbf{Lösung:} Da $g \circ f$ injektiv ist, wissen wir
      \begin{equation}
        \forall x_1, x_2 \in X: x_1 \neq x_2 \implies g (\underbrace{f (x_1)}_{\in Y}) \neq g ( \underbrace{f (x_2)}_{\in Y}).
      \end{equation}

      Daraus folgt aber auch, dass $f(x_1) \neq f(x_2)$, womit wir gezeigt haben, dass $f$ auch injektiv ist.

      $g$ muss nicht unbedingt injektiv sein, da $Y$ grösser sein kann als $X$.
    }

    \item {
      Zeigen Sie, dass $g$ injektiv ist, falls $g \circ f$ injektiv ist und $f$ surjektiv ist.

      \textbf{Lösung:} Die Injektivität von $g\circ f$ bedeutet, dass
      \begin{equation}
        \forall x_1, x_2 \in X:
          g(f(x_1)) = g(f(x_2)) \implies x_1 = x_2. $ (Das bedeutet auch, dass $f(x_1) = f(x_2)$ ist)$
      \end{equation}

      Respektive mit Zwischenschritten
      \begin{equation}
        \begin{aligned}
          &\forall x_1, x_2 \in X \; \exists y_1, y_2 \in Y: \\
          &\quad ((y_1 = f(x_1) \und y_2 = f(x_2) \und g(y_1) = g(y_2)) \\
          &\quad \implies x_1 = x_2) \implies y_1=y_2
        \end{aligned}
      \end{equation}

      Aus der Surjektivität von $f$ wissen wir auch
      \begin{equation}
        \forall y_1,y_2 \in Y \; \exists x_1, x_2 \in X : y_1 = f(x_1) $ und $ y_2 = f(x_2).
      \end{equation}

      Und somit können wir die Werte für $y_1$  und $y_2$ auch direkt verwenden, da sowieso die gesammte Menge $Y$ verwendet wird.
      \begin{equation}
        \begin{aligned}
          &\forall y_1,y_2 \in Y \; \exists x_1, x_2 \in X \\
          &\quad (y_1 = f(x_1) \und y_2 = f(x_2) \und g(y_1) = g(y_2)) \\
          &\quad \implies x_1 = x_2 \implies y_1=y_2
        \end{aligned}
      \end{equation}

      vereinfacht sich zu
      \begin{equation}
        \begin{aligned}
          &\forall y_1,y_2: g(y_1) = g(y_2) \implies y_1=y_2,
        \end{aligned}
      \end{equation}

      womit wir gezeigt haben, dass $g$ injektiv ist.

      Eine direktere Formulierung wäre wie folgt: Die Injektivität von $g \circ f$ beduetet, dass
      \begin{equation}
        \forall x_1, x_2 \in X:
        g \circ f (x_1) = g (f(x_1)) = g(f(x_2)) = g \circ f(x_2) \implies x_1 = x_2.
      \end{equation}

      Dies bedeutet aber auch, dass im Falle einer Implikation, $f(x_1) = f(x_2)$ ist. Weil $f$ surjektiv ist, gilt. \begin{equation}
        \forall y \in Y \; \exists x \in X: y = f(x).
      \end{equation}

      Das heisst, wir können die Menge $Y$ direkt verwenden und kommen zum Schluss dass
      \begin{equation}
        \forall y_1, y_2 \in Y: g(y_1) = g(y_2) \implies y_1 = y_2
      \end{equation}

      Somit haben wir gezeigt, dass $g$ injektiv ist.
    }
  \end{enumerate}
\end{imp-ex}

% 1.4.4
\subsection{Bild- und Urbildmenge}
Sei $f: X \to Y$ und $A \subseteq X$. Man erinnert sich and die Notation $f(A) :\equiv \{f(x) \mid x \in A\}.$ für das Bild von $A$. Manchmal verwenden wir auch für eine Aussage $P(x)$ die Notation:
\begin{equation}
  \label{eq: vereinfachte schreibweise}
  \begin{aligned}
    \{ f(x) \mid x \in X \und P(x) \}
    : &\equiv
    \left \{ y \mid y \in Y \und \exists x \in X : \left(f(x) = y \und P(x) \right) \right \} \\
    &=   \left \{ y \in Y \mid  \exists x \in X : \left(f(x) = y \und P(x) \right) \right \}
  \end{aligned}
\end{equation}

\mce{48}
\begin{mydef} [Urbilder bezüglich einer Funktion]
  Für $f:X \to Y$ und $B \subseteq Y$ definieren wird das \qt{Urbild} $f^{-1} (B)$ von $B$ unter $f$ als
  \begin{equation}
    f^{-1} (B) :\equiv \{ x \in X \mid f(x) \in B \}
  \end{equation}

  Es gilt beispielsweise $f^{-1} (Y) = X$ und $f^{-1}( \{\})= \{\}$.
\end{mydef}

\begin{imp-ex}[Verhalten von Bildern und Urbilderln unter Mengenoperationen]
  Sei $f:X \to Y$ und $A, A' \subseteq X$ und $B,B' \subseteq Y$.
  \begin{enumerate}[label=(\roman*)]
    \item Zeigen Sie, dass $ f(f^{-1}(B)) \subseteq B$ gilt. Unter welcher Bedingung an $f$ gilt auf jeden Fall Gleichheit?

    \textbf{Lösung:}
    \begin{equation}
      \begin{aligned}
        f(f^{-1}(B))
        &= \{ y \in Y \mid \exists x \in \{ x \in X \mid f(x) \in B\} : y= f(x) \} \\
        &= \{ y \in Y \mid \exists x \in X : f(x) \in B \und y = f(x) \} \\
        &= \{ y \in Y \mid \exists x \in X : y = f(x) \in B  \} \\
        &= \{ f(x) \mid x \in X \und f(x) \in B  \} \qquad \qquad \mytext{Siehe: \eqref{eq: vereinfachte schreibweise}} \\
        & \subseteq B
      \end{aligned}
    \end{equation}

    Es herrscht auf jeden Fall Gleicheit, falls $f$ surjektiv ist, weil $B \subseteq Y$. Das bedeutet, $\forall y \in B \; \exists x \in X: y = f(x)$.
  \end{enumerate}
\end{imp-ex}


  \chapter{Die reellen Zahlen}
  \section{Die Axiome der reele Zahlen}

\begin{mydef}[reelle Zahlen, $\real$]
  Die Menge der reellen Zahlen $\real$ besitzt eine \qt{Addition}
  \[
    +: \real \times \real \to \real, (x,y) \mapsto x+y,
  \]

  und eine \qt{Multiplikation},
  \[
    \cdot: \real \times \real \to \real, (x,y) \mapsto x\cdot y.
  \]

  und eine Relation $\leq$ auf $\real$, die wir \qt{kleiner gleich} nennen.
\end{mydef}

\subsection{Körperaxiome}

Die Addition erfüllt folgende Eigenschaften
\begin{enumerate}[label=\textbf{(\alph*)}]
  \item Nullelement: $\exists 0 \in \real \; \forall x \in \real: x + 0 = 0 + x = x$
  \item Additives Inverses: $\forall x \in \real \; \exists (-x) \in \real: x + (-x) = (-x) + x = 0$
  \item Assoziativgesetz: $\forall x,y,z \in \real: (x+y)+z = x+(y+z)$
  \item Kommutativgesetz: $\forall x,y \in \real: x+y = y + x$
\end{enumerate}
$\real$ wird gemeinsam mit der Verknüpfung $+: \real \times \real \to \real$ auch \qt{kommutative} oder \qt{abelsche Gruppe} genannt.

\emph{Erste Folgerungen für die Addition.}
\begin{enumerate}
  \item Das Nullelement ist eindeutig. Denn gäbe es zwei unterschiedliche $0_1$ und $0_2$, dann folgt daraus $0_1 = 0_1 + 0_2 = 0_2$.
  \item Das Negative $-x$ ist für jedes $x \in \real$ eindeutig bestimmt. Falls für $x,y,z$ die gleichung $0=x+y=x+z$ gilt, dann gilt $y=y+0=y+(x+z)=(y+x)+z=(x+y)+z=0+z=z.$
  \item Es gilt $-(-x) = x \quad \forall x \in \real$. Denn das additive Inverse ist eindeutig und weil $(-x) + x = 0$ gilt, gilt auch $-(-x) = x$.
  \item \qt{Additives Kürzen}: $x + y = x + z \implies y=z \quad \forall x,y,z \in \real$.
  \[
    \begin{aligned}
      y = ((-x)+x)+y &= (-x)+(x+y) \\
      &=(-x)+(x+z) = ((-x)+x)+z = z
    \end{aligned}
  \]
\end{enumerate}

\begin{imp-ex}
  Zeigen Sie die folgenden Regeln.
  \begin{enumerate}
    \item $-0=0$
    \item $-(x+y)=(-x)+(-y)  \quad \forall x,y \in \real$

    Wir bezeichnen $(-x)+(-y)$ als $-x-y$.
    \item $-(x-y)=-x+y \quad \forall x,y \in \real$
  \end{enumerate}
\end{imp-ex}

Die Multiplikation erfüllt folgende Eigenschaften
\begin{enumerate}
  \item Einselemnt: $\exists 1 \in \real - \{0\} \; \forall x \in \real: x \cdot 1 = 1 \cdot x = x$
  \item Multiplikative Inverse: $\forall x \in \real - \{0\} \; \exists (x^{-1}) \in \real: x \cdot (x^{-1}) = (x^{-1})x = 1$
  \item  Assoziativgesetz: $\forall x,y,z \in \real: x \cdot (y \cdot z) = (x \cdot y) \cdot z$
  \item Kommutativgesetz: $\forall x,y \in \real: x \cdot y = y \cdot x$
  \item \emph{Kompatibilität von $+$ und $-$} \\
  Distributivgesetz: $(x+y)\cdot z = (x \cdot z) + (y \cdot z) \quad \forall x,y,z \in \real$
\end{enumerate}

\emph{Folgerungen für die Multiplikation}
\begin{enumerate}
  \item $0 = 0x = x0 \quad \forall x \in \real$
  \item $(-1)x = -x \quad \forall x \in \real$, weil $x + (-1) x = 1x + (-1)x = (1+(-1))x = 0x = 0$. Man sieht, dass $(-1)x$ das additive Inverse von $x$ ist. Desshalb gilt $(-1)x = -x$.
  \item \qt{Multiplikatives Kürzen} ist erlaubt: Seit $x \in \real - \{0\}, yz \in \real$ und es gilt $xy=xz$. Dann gilt $y=z$. Weil
  \[
  \begin{aligned}
   y=1y=((x^{-1})x)y&=(x^{-1})(xy)\\
   &=(x^{-1})(xz)=((x^{-1})x)z=1z = z
  \end{aligned}
  \]
  \item Es gibt keine \qt{Nullteiler}: Falls $xy=0$ für zwei Elemente $x,y \in \real$, dann ist $x=0$ oder $y=0$. Denn falls $x \neq 0$, dann gilt $x \cdot 0 = 0$ und $x \cdot y = 0$. Deswegen ist $y=0$.
  \item Das Einselement ist durch die Eigenschaft in Axiom (e) eindeutig bestimmt.
  \item Das (mulötiplikatives) Inverse $x^{-1} \in \real - {0}$  ist für jedes Element $x \in \real - \{0\}$ eindeutig durch $x \cdot x^{-1} = 1$ bestimmt.
  \item $(x^{-1})^{-1} = x$.
\end{enumerate}

\begin{imp-ex}
  Seien $x,y,z \in \real$.
  \begin{enumerate}
    \item Zeigen Sie, dass die Identität $(-x)(-y)=xy$ gilt. Überprüfen Sie auch, dass $-x \in \real - \{0\}$ und $(-x)^{-1} = -(x^{-1})$ gilt, falls $x \in \real-{0}$ ist.
    \item Zeigen Sie, dass das Distributivgesetz für die Subtraktion $x(y-z) = (xy)-xz$ gilt.
  \end{enumerate}
\end{imp-ex}

Man verwendet die Schreibweise des \qt{Quotienten} $frac{x}{y}=xy^{-1}$ für alle \qt{Zähler} $x \in \real$ under \qt{Nenner} $y \in \real$. Die Inverse $\frac{1}{y}=y^{-1}$ von $y \in \real-\{0\}$ nennen wir den \qt{reziproken Wert} oder den \qt{Kehrwert} von $y$.

\begin{imp-ex} [Rechenregel für Quotienten]
  \phantom{.}
  \begin{enumerate}
    \item $\tfrac{x}{y}=\tfrac{z}{w} \iff xw = yz \quad \forall x,z \in \real \myand \forall y,w \in \real^{\times}$.
    \item $\tfrac{x}{y} \cdot \tfrac{z}{w} = \tfrac{xz}{yw} \quad \forall x,z \in \real \myand \forall y,w \in \real^{\times}$
    \item $\dfrac{\tfrac{x}{y}}{\tfrac{z}{w}} = \tfrac{xw}{yz} \quad \forall x \in \real \myand \forall y,z,w \in \real^{\times}$.
    \item $\tfrac{x}{y} + \tfrac{z}{w} = \tfrac{xw + yz}{yw} \quad \forall x,z \in \real$ und $\forall y,w \in \real^{\times}$
  \end{enumerate}
\end{imp-ex}

\begin{imp-ex}
  Sei $a^2 :\equiv a \cdot a \quad \forall a \in \real$. Zeigen Sie die Gleichungen.
  \begin{itemize}
    \item $(a+b)^2 = a^2 + 2ab + b^2$
    \item $(a-b)^2 = a^2 -2ab + b^2$
    \item $(a+b)(a-b)=a^2-b^2$
  \end{itemize}

\end{imp-ex}

\subsection{Angeordnete Körper}
\textbf{Axiome (Anordnung).} Die Relation $\leq$ auf $\real$ erfüllt die folgenden vier Axiome. $\forall x,y,z \in \real$:
\begin{enumerate}
  \item Reflexivität: $x \leq x$.
  \item Antisymmetrie: $x \leq y$ und $y \leq x \implies x = y$.
  \item Transitivität: $x \leq y$ und $y \leq z \implies x \leq z$.
  \item Linearität: $x \leq y$ oder $y \leq x$.
  \item $\leq$ und $+$: $x \leq y \implies x+z \leq y+z$
  \item $\leq$ und $\cdot$: $(0 \leq x$ und $0 \leq y)$ $\implies$ $0\leq xy$.
\end{enumerate}

Die Axiome (a) - (c) sind die Axiome einer \qt{(partiellen) Ordnung} und zusammen mit Axiom (c) bilden sie die Axiome einer \qt{linearen} (oder auch \qt{totalen}) \qt{Ordnung}.

\begin{mydef-non}
  Weiter definieren wir $x < y$ durch $(x \leq y$ und $x \neq y)$ und sagen \qt{$x$ ist (echt) kleiner als $y$}.
\end{mydef-non}

\begin{mydef-non}
  $x > y$ wird definiert als $y < x$.
\end{mydef-non}

\textbf{Folgerungen.} Das Hinzufügen der Anordnungsaxiome hat folgende Konsequenzen. $\forall x,y,z,w \in \real:$
\begin{enumerate}
  \item Trichotomie: Es gilt entweder $x < y, x=y$ oder $x > y$.
  \item $x < y$ und $y \leq z$ $\implies$ $x < z$.
  \item $x \leq y$ und $z \leq w$ $\implies$ $x+z \leq y+w$.
  \item $y \leq z \iff 0 \leq z-y$.
  \item $x \geq 0 \iff -x \leq 0$.
  \item $x^2 \geq 0$ und $x^2>0$, falls $x \neq 0$.
  \item $0 < 1$.
  \item $0 \leq x$ und $y \leq z$ $\implies xz \leq xz$.
  \item $x \leq 0$ und $y \geq z$ $\implies xy \leq xz$.
  \item $0 < x \leq y$ $\implies$ $0 < x^{-1} \leq y^{-1}$.
  \item $0 \leq x \leq y$ und $0 \leq z \leq w$ $\implies$ $0 \leq xz \leq yw$.
  \item $x + y \leq x + z$ $\implies$ $y \leq z$. (Man darf $x$ streichen)
  \item $xy \leq xz$ $\implies$ $y \leq z$. (Man darf $x$ streichen)
\end{enumerate}

\subsection{Das Vollständigkeitsaxiom}

\[
  \begin{aligned}
    &\forall X,Y \subseteq \real: \Big( \left( X \neq \varnothing \myand Y = \varnothing \myand \forall x \in X \; \forall y \in Y: x \leq y \right) \\
    &\qquad \implies \left(\exists c \in \real \; \forall x \in X \; \forall y \in Y: x \leq c \leq y \right) \Big)
  \end{aligned}
\]

\subsection{Eine erste Anwendung der Vollständigkeit}

\begin{imp-ex}[Existenz der Wurzelfunktion]
  \phantom{.}\\
  $\forall x,y,a \in \real_{\geq0}$
\begin{enumerate}
  \item $x<y \iff x^2 < y^2$.
  \item Eindeutigkeit: $\exists! c \in \real_{\geq 0}: c^2 = a$.
  \item Für eine reele Zahl $a>0$ erfüllen die Teilmengen \[
  X :\equiv \{x \in \real_{\geq 0} \, \mid \, x^2 < a \}, \quad Y :\equiv \{y \in \real_{\geq 0} \, \mid \, y^2 > a \}
  \]

  die Vorraussetzungen des Vollsändigkeitsaxiom.

\begin{mydef-non}
    Wir bezeichnen für jedes $a \geq 0$ die durch $c^2 = a$ und $c \geq 0$ eindeutig bestimmte reelle Zahl als $c=\sqrt{a}$ und sprechen von der \qt{Wurzel von} $a$.
\end{mydef-non}

  \item Wachsend: $x < y \iff \sqrt{x} < \sqrt{y}$.
  \item Bijektion: Die Würzelfunktion ist von $\real_{\geq 0}$ nach $\real_{\geq 0}$ bijektiv.
  \item Multiplikativität: $\sqrt{xy} = \sqrt{x}\sqrt{y}$.
  \item Zwei Lösungen: Für jedes $a>0$ gibt es genau zwei Lösungen der Gleichung $x^2 = a$ mit $x \in \real$.
\end{enumerate}
\end{imp-ex}
  \section{Intervalle und der Absolutbetrag}

\mce{41}
\begin{mydef}[Intervalle]
  Seien $a,b \in \real$. Dann wird das
  \begin{itemize}
    \item \emph{abgeschlossene Intervall} $[a,b]$ durch
    $[a,b] :\equiv \{ x \in \real \mid a \leq x \leq b \}$, das
    \item \emph{offene Intervall} $[a,b)$ durch
    $[a,b) :\equiv \{x \in \real \mid a \leq x < b \}$, und das
    \item \emph{(rechts) halboffne Intervall} $(a,b]$ durch
    $(a,b] :\equiv \{ x \in \real \mid a < x \leq b \}$
  \end{itemize}
  definiert. Wenn das Intervall nicht-leer ist, dann wird $a$ der \qt{linke Endpunkt}, $b$ der \qt{rechte Endpunkt}, und $b-a$ die \qt{Länge des Intervalls} genannt. Die Intervalle $(a,b]$, $[a,b)$, $(a,b)$ sind für genau dann nicht-leer, wenn $a<b$ ist. Das Intervall $[a,b]$ ist genau dann nicht-leer, wenn $a\leq b$ ist. Die genannten Intervalle werden auch \qt{beschränkte Intervalle} genannt.
\end{mydef}

\begin{mydef}[Unbeschränkte Intervalle]
  Für $a,b \in \real$ definieren wir die \qt{unbeschränkten abgeschossenen Intervalle}
  \[
    \begin{aligned}
      [a, \infty) &:\equiv \real_{\geq a} :\equiv \{x \in \real \mid a \leq x \} \\
      (-\infty, b) &:\equiv \real_{\leq b} :\equiv \{ x \in \real \mid x \leq b \}
    \end{aligned}
  \]

  und die \qt{unbeschränkten offenen Intervalle}
  \[
    \begin{aligned}
      (a, \infty) &:\equiv \real_{>a} :\equiv \{ x \in \real \mid a < x \} \\
      (-\infty, b) &:\equiv \real_{<b} :\equiv \{ x \in \real \mid x < b \} \\
      (-\infty, \infty) &:\equiv \real
    \end{aligned}
  \]
\end{mydef}

\begin{mydef}[Umgebung eines Punktes] Sei $x \in \real$. Eine Menge, die ein offenes Intervall enthält, in dem $x$ liegt, wird auch eine \qt{Umgebung} oder \qt{Nachbarschaft} von $x$ genannt. Für ein $\delta > 0$ wird das offene Intervall $(x-\delta, x + \delta)$ die \qt{$\delta$-Umgebung} von $x$ genannt.

Beispielsweise wäre also $\rat \cup [-1,1]$ eine Umgebung von $0 \in \real$, weil sie das offene Intervall $(-1,1)$ und den Punkt $0$ enthält. Die Umgebung selbst kann demnach auch abgeschlossen sein.

Falls ein $y \in \real$ in einer $\delta$-Umgebung von $x$ ist, so sagt man auch, dass $y$ \qt{$\delta$-nahe} an $x$ ist.

Definition durch Absolutbetrag: Für ein $\delta > 0$ und ein $x \in \real$ ist die $\delta$-Umgebung von $x$ durch $\{y \in \real \mid \abs{x-y} < \delta \}$ gegeben. $\abs{x-y} = \abs{y-x}$ kann als \qt{Abstand} von $x$ zu $y$ interpretiert werden.

\end{mydef}

\begin{ex}[Verhalten von Intervallen unter Durchscnitt und Vereinigung]
  \phantom{.}
  \begin{enumerate}
    \item Zeigen Sie, dass ein endlicher Schnitt $\cap _{k=1}^{n}I_k$ von Intervallen $I_1, \dots, I_n$ wieder ein Intervall ist, wenn die leere Menge auch als Intervall zugelassen ist. Kann man die Endpunkte eines nicht-leeren Durchschnitts mittels der Endpunkte der ursprünglichen Intervalle beschreiben?
    \item Wann ist die Vereinigung von zwei Intervallen wieder ein Intervall? Was geschieht in diesem Fall, wenn man zwei Intervalle des selben Typs (offen, abgeschlossen, links halboffen, rechs halboffen) vereinigt?
  \end{enumerate}
\end{ex}

\subsection{Der Absolutbetrag auf den reellen Zahlen}
\begin{mydef}[Absolutbetrag] Der \qt{Absolutbetrag} ist due Funktion
  \[| \cdot | : \real \to \real, \quad x \mapsto |x| :\equiv
  \begin{cases}
    x & \text{ falls } x \geq 0 \\
    -x & \text{ falls } x \leq 0
  \end{cases}\]
\end{mydef}

\textbf{Folgerungen.} $\forall x, y \in \real$ gilt
\begin{enumerate}
  \item Es ist $|x| \geq 0$ und $|x| = 0 \iff x=0$. Dies folgt aus der Trichtomie der Reellen Zahlen:\\
  Für $x=0$ gilt $|x| = 0$, für $x>0$ gilt $|x|=x > 0$, und für $x<0$ folgt $|x| = -x > 0$.
  \item $|-x| = |x|$
  \item $|xy| = |x||y|$. Beweis durch Fallunterscheidung.
  \item Für $x\neq 0$ gilt $\left| \tfrac{1}{x} \right | = \tfrac{1}{|x|}$. Beweis: Aus (c) folgt $\left| \tfrac{1}{x} \right | {|x|} = \left| \tfrac{1}{x} x \right |  = 1$.
  \item $\abs{x} \leq y$ $\iff$ $-y \leq x  \leq y$.
  \begin{prf}
    \qt{$\Rightarrow$ Richtung}: Angenommen $\abs{x} \leq y$. Falls $x \geq 0$, dann gilt $-y \leq 0 \leq x = \abs{x} \leq y$. Falls $x < 0$, dann ist $-y \leq -\abs{x} = x < 0 \leq y$ und damit wiederum $-y \leq x \leq y$. \\
    \qt{$\Leftarrow$ Rightung:} Wir bemerken, dass $-y \leq x \leq y$ auch $-y \leq -x \leq y$ und somit auf jeden Fall $\abs{x} \leq y$ impliziert.
  \end{prf}
  \item $\abs{x} < y \iff -y < x < y$.
  \item \emph{Dreiecksungleichung:} $\abs{x+y} \leq \abs{x} + \abs{y}.$
  \begin{prf}
    Wir addieren die Ungleichungen $-\abs{x} \leq x \leq \abs{x}$ und $-\abs{y} \leq y \leq \abs{y}$ und erhalten
    \[
      -(\abs{x} + \abs{y}) \leq x+y \leq \abs{x} + \abs{y}
    \]

    Woraus nach Eigenschaft (e) die Gleichung $\abs{x+y} \leq \abs{\abs{x} + \abs{y}} = \abs{x} + \abs{y}$ folgt.
  \end{prf}
  \item \emph{Umgekehrte Dreiecksungleichung:} $\abs{\abs{x} - \abs{y}} \leq \abs{x-y}$.
\end{enumerate}

\begin{ex}
  Wann gilt Gleichheit in der Dreicecksungleichung oder umgekehrten Dreiecksungleichung?
\end{ex}

\begin{mydef-non}[Vorzeichen/Signum]
  $\forall x \in \real:$ $x = \operatorname{sgn}(x) \abs{x}$, weobei $\operatorname{sgn}(x)$ das \qt{Vorzeichen} (oder \qt{Signum}) von $x$ ist, welches durch
\[
  \operatorname{sgn}: \real \to \{-1,0,1\}, \quad x \mapsto \begin{cases}
    1  & \text{ falls } x > 0 \\
    0  & \text{ falls } x = 0 \\
    -1 & \text{ falls } x < 0
  \end{cases}
\]

definiert ist.
\end{mydef-non}

\begin{ex}[Absolutbetrag und Quadratwurzel] $\forall x \in \real: x^2 = \abs{x}^2$ und $\sqrt{x^2} = \abs{x}$.
\end{ex}

\begin{mydef}[Offene und abgeschlossene Teilmengen]
  Eine Teilmenge $U \subseteq \real$ heisst \qt{offen} (in $\real$), wenn für jedes $x \in U$ ein $\epsilon >0$ exisitiert mit
  \[
    \{ y \in \real \mid \abs{y-x} < \epsilon \} = (x-\epsilon, x + \epsilon) \subseteq U.
  \]

  Eine Teilmenge $A \subseteq \real$ heisst \qt{abgeschlossen} (in $\real$), wenn ihr Komplement $\real \backslash A$ offen ist.
\end{mydef}

\begin{ex}[Offene Intervalle]
  Zeigen Sie, dass eine Teilmene $U \subseteq \real$ genau dann offen ist, wenn $\forall x \in U$ ein offenenes Intervall $I$ mit $x \in I$ und $I \subseteq U$ existiert. Schliessen Sie, dass die offenen (respektive abgeschlossenen) Intervalle auch im Sinne der obigen Definition offen (respektive abgeschlossen) sind.
\end{ex}

\begin{ex}
  Welche der folgenden Teilmengen von $\real$ sind jeweils offen, abgeschlossen oder weder noch?
  \begin{itemize}
    \item Die Teilmengen $\varnothing, \nat, \integer, \real$.
    \item Die Teilmengen $[0,1), (0,1]$ und $(0,1) \cup (2,3)$
  \end{itemize}
\end{ex}

\section{Der Absolutbetrag auf den komplexen Zahlen}

\begin{mydef}[Absolutbetrag]
  Der \qt{Absolutbetrag} auf $\compl$ ist gegeben durch
  \[
    \abs{z} = \abs{x + yi} :\equiv \sqrt{x^2 + y^2} = \sqrt{z \overline z} $ für $z=x+yi \in \compl.
  \]

  und stimmt mit dem Absolutbetrag auf $\real$ überein, da für $x \in \real: \sqrt{x \overline x} = \sqrt{x^2} = \abs{x}$.
\end{mydef}

\textbf{Eigenschaften des Absolutbetrags auf $\compl$.}
$\forall z,w \in \compl$ gilt
\begin{enumerate}
  \item \emph{Definitheit:} $\abs{z} > 0$ und $\abs{z}=0$ nur dann, wenn $z = 0$.
  \item \emph{Multiplikativität:} $\abs{zw} = \abs{z}\abs{w}$.
  \item \emph{Dreiecksungleichung:} $\abs{z+w} \leq \abs{z} + \abs{w}$.
  \item \emph{Umgekehrte Dreiecksungleichung:} $\abs{\abs{z}-\abs{w}} \leq \abs{z-w}$.
\end{enumerate}

\mce{52}
\begin{mydef}[Offene Bälle]
  Der \qt{offene Ball} mit Radius $r > 0$ um einen Punkt $z \in \compl$ ist die Menge
  \[
    B_r :\equiv \{ w \in \compl \mid \abs{z-w} <r \}.
  \]
\end{mydef}

\begin{imp-ex}[Durchschnitt von offenen Bällen] Seien $z_1, z_2 \in \compl, r_1 > 0$ und $r_2 > 0$. Für jeden Punkt $z \in B_{r_1}(z_1) \cup B_{r_2}(z_2)$ existiert ein Radius $r>0$, so dass
\[
  B_r(z) \subseteq B_{r_1}(z_1) \cup B_{r_2}(z_2).
\]
\end{imp-ex}

\begin{mydef}[Offene und abgeschlossene Teilmenge von $\compl$] Eine Teilmenge $U \subseteq \compl$ heisst \qt{offen} (in $\compl$), wenn zu jedem Punkt in $U$ ein offener Ball um diesen Punkt existiert, der in $U$ enthalten ist.
\[
  \forall z \in U \subseteq \compl \; \exists r>0: B_r(z) \subseteq U.
\]

Eine Teilmenge $A \subseteq \compl$ heisst \qt{abgeschlossen} (in $\compl$), falls ihr Komplement $\compl \backslash A$ offen ist.
\end{mydef}


  \section{Maximum und Supremum}

\subsection{Maximum und Minimum}

\mce{56}
\begin{mydef}[Maximum] Für eine Teilmenge $X \subseteq \real$ sagen wir, dass
  \[
    x_0 :\equiv \operatorname{max} (X) \in \real.
  \]

das \qt{Maximum} von $X$ ist, falls $\forall x \in X: x \leq x_0$ gilt.
\end{mydef}

\begin{mydef}[Minimum]
  Analog sagen wir für eine Teilmenge $X \subseteq \real$, dass
  \[
  x_0 :\equiv \operatorname{min} (X) \in \real.
  \]

  das \qt{Minumum} von $X$ ist, falls $\forall x \in X: x \geq x_0$ gilt.
\end{mydef}

\subsection{Supremum und Infimums}

\begin{mydef}[Beschränktheit und Schranken]
  $X \subseteq \real$ heisst \qt{von oben beschränkt}, falls es ein $s\in \real$ mit $x \leq s \quad \forall x \in X$ gibt. Ein solches $s$ nennet man in diesem Fall eine \qt{obere Schranke} von $X$.

  $X$ heisst \qt{nach unten beschränkt}, falls es ein falls es ein $i\in \real$ mit $x \geq i \quad \forall x \in X$ gibt. Ein solches $i$ nennt man analog eine \qt{untere Schranke} von $X$.

  $X$ heisst \qt{beschränkt}, falls sie von oben und unten beschränkt ist.
\end{mydef}

\begin{thm}[Supremum]
  \label{thm: supremum}
  Sei $X \subseteq \real$ eine von oben beschränkte, nicht-leere Teilmenge. Dann gibt es eine \qt{kleinste obere Schranke} von $X$, die auch das \qt{Supremum} $\operatorname{sup}(X)$ von $X$ genannt wird. $s_0 = \operatorname{sup}(X)$ erfüllt folgende Eigenschaften:
  \begin{enumerate}
    \item \emph{$s_0$ ist eine obere Schranke:} $\forall x \in X: x \leq s_0$.
    \item \emph{$s_0$ ist kleiner gleich jeder oberen Schranke:} $\forall s \in \real: \big((\forall x \in X: x \leq s) \implies s_0 \leq s\big)$
    \item \emph{Kleinere Zahlen sind keine oberen Schranken:} $\forall \epsilon > 0 \; \exists x \in X: x > s_0-\epsilon.$
  \end{enumerate}

  Eigenschaften (b) und (c) sind äquivalent. Falls das Maximum $x_0 = \max(X)$ existiert, dann ist $\max(X) = \sup(X)$. Denn aus $x_0 \in X$ folgt $x_0 \leq s$ für jede obere Schranke $s$ von $X$. Falls $\sup(X) \in X$, dann ist $\sup(X) = \max(X)$, da das Supremum eine obere Schranke ist.
\end{thm}

\mce{61}
\begin{ex}[Existenz des Infimums]
  Machen Sie das gleiche wie in \ref{thm: supremum}. Gehen Sie entweder analog vor, oder das Supremum der  Teilmenge
  \[
    -X :\equiv \{ -x \mid x \in X \}
  \]

  für eine von unten beschränkte, nicht-leere Teilmenge $X \subseteq \real$ betrachten.
\end{ex}

\begin{mydef-non}[Notation]
  Die Notation lässt sich verallgemeinern. Sei $x \in \real$ und $A,B \subseteq \real$. Wir definieren:
  \[
  \begin{aligned}
    -X &:\equiv \{ -x \mid x \in X \} \\
    x+A &:\equiv \{ x+a \mid a \in A \} \\
    A + B &:\equiv \{a + b \mid a \in A, b \in B \} \\
    xA &:\equiv \{xa \mid a \in A \} \\
    AB &:\equiv \{ab \mid a \in A, b \in B \}
  \end{aligned}
  \]

  Es gilt beispielsweise $x+A = \{x\}+A, xA=\{x\}A$ oder auch $[a,b]+[c,d] = [a+c, b+d] \quad \forall a,b,c,d \in \real$ mit $a \leq b$ und $c \leq d$.
\end{mydef-non}

\begin{thm}[Supremum unter Streckung]
  Sei $A \subseteq \real$, $A \neq \varnothing$ und sei $c > 0$. Dann ist $cA$ von oben beschränkt und es gilt $\sup (cA) = c  \sup (A).$
\end{thm}
\begin{prf}
  Sei $s:\equiv\sup(A)$ und $c>0$. Dann gilt $a \leq s \quad \forall a \in A$ wie auch $ca \leq cs$. Da jedes Element von $cA$ von der Form $ca$ für $a \in A$ ist, erhalten wir, dass $cs$ eine obere Schranke von $cA$ ist. $cA$ ist somit von oben beschränkt.

  Sei $\epsilon > 0$. $\implies \exists a \in A: a > s-\frac{\epsilon}{c}$. Somit gilt auch $ca > cs-\epsilon$. Dies charakterisiert das Supremum von $cA$, weil jedes Element von der Form $ca$ ist und wir erhalten $\sup(cA) = cs = c\sup(A)$.
\end{prf}

\begin{thm}[Supremum unter Summen]
  Seien $A,B \subset \real$, zwei nicht-leere, von oben beschränkte Teilmengen von $\real$. Dann ist $A+B$ von oben beschränkt und es gilt
  \[
    \sup(A+B) = \sup(A) + \sup(B).
  \]
\end{thm}
\begin{prf}
  Wir definieren $s_A :\equiv \sup(A)$ und $s_B :\equiv \sup(B)$. Dann gilt $a \leq s_A$ und $b \leq s_B$ $\quad \forall a \in A, b \in B$. Dies impliziert
  \[
    a+b \leq s_A + s_B \quad \forall a \in A, b\in B.
  \]

  Somit ist $s_A + s_B$ eine obere Schranke von $A+B$, weil alle Elemente von $A+B$ von der Form $a+b$ sind.

  Sei $\epsilon > 0$. Dann existiert ein $a \in A$ mit $a > s_A - \frac{\epsilon}{2}$ und ein $b \in B$ mit $b > s_B - \frac{\epsilon}{2}$. Dies impliziert
  \[
    a+b > (s_A + s_B) - \epsilon,
  \]

  und somit
  \[
    \sup(A+B) = s_A + s_B = \sup(A) + \sup(B).
  \]

  \vspace{-\baselineskip}
\end{prf}

\subsection{Uneigentliche Werte, Suprema und Infima}


  \section{Erste Konsequenzen der Vollständigkeit}

\subsection{Das Archimedische Prinzip}

\begin{thm}[Das Archimedische Prinzip]
  \label{thm: das archimedische prinzip}
  \phantom{.} \\
  Es gelten folgende Aussagen:
  \begin{enumerate}
    \item Jede nicht-leere, von oben beschränkte Teilmenge von $\integer$ hat ein Maximum.
    \item $\forall x \in \real \; \exists! n \in \integer$: $n \leq x < n+1$. Dies wird als der eindeutig bestimmte \qt{ganzzahlige Anteil} $\lfloor x \rfloor$ einer reellen Zahl $x \in \real$ definiert. Wir erhalten also die Funktion $x \in \real \mapsto \lfloor x \rfloor \in \integer$, die auch \qt{Abrundungsfunktion} genannt wird.
    \item $\forall \epsilon > 0 \; \exists n \in \nat: \frac{1}{n} < \epsilon$.
  \end{enumerate}
\end{thm}

Der \qt{gebrochene Anteil} (oder auch \qt{Nachkommaanteil}) ist $\{ x\} :\equiv x - \lfloor x \rfloor \in [0,1)$. Wir erhalten eine Funktion $x \in \real \mapsto \{ x\} \in [0, 1)$ mit $x=\lfloor x \rfloor + \{ x\} \quad \forall x \in \real$.

\begin{prf}
  Sei $E \subset \integer$, $E \neq \varnothing$ eine von oben beschränkte Teilmenge. Da auch $E \subseteq \real$, folgt aus \ref{thm: supremum} $\exists s_0 = \sup (E)$. Nach \ref{thm: supremum} (c) $\exists n_0 \in E: s_0 \geq n_0 > s-1$. Wobei hier $1$ die Rolle von $\epsilon$ übernommen hat. Es folgt $s_0 < n_0+1$. Somit gilt $\forall m \in E: m \leq s_0 < n_0+1$ und $m \leq n_0$. Daher ist $n_0$ das Maximum von $E$ und Aussage (a) ist bewiesen.

  \textbf{Zu (b):} Sei $x \in \real, x \geq 0$. Dann ist $E :\equiv \{n \in \integer \mid n\leq x\}$ eine von oben beschränkte, nicht leere Teilmenge, weil auf jeden Falls $0 \in E$. Nach (a) hat $E$ ein Maximum, das heisst, es gibt ein maximales $n_0 \in \integer$ mit $n_0 \leq x$. Daraus folgt $x < n_0+1$, weil falls $x \geq n_0+1$ gelten würde, $n_0$ nicht das maximale Elemennt wäre. Somit gilt $n_0 \leq x < n_0+1$ wie in (b).

  Falls $x<0$ ist, dann können wir obigen Fall auf $-x$ anwenden und finden ein $l \in \integer$ mit
  \[
    l \leq -x < l +1
  \]

  Falls $x=-l$ wählen wir $k :\equiv l+1$ weil dann die Ungleichung $l = -x < l+1$ galt. Ansonsten wählen wir $k :\equiv l$ weil dann die Ungleichung $l < -x < l+1$ gegolten hat und wir erhalten somit für dieses gewählte $k \in \{ l, l+1\} \subseteq \integer$ die Gleichung
  \[
    k-1 < -x \leq k
  \]
  Für $n :\equiv -k \in \integer$ erhalten wir
  \[
  \begin{aligned}
    -n-1 < -x \leq -n \\
    n+1 > x \geq n
  \end{aligned}
  \]

  Somit ist die Existenz einer solchen natürlichen Zahl $n$ auch für diesen Fall bewiesen.

  \textbf{Eindeutigkeit:} Für den Beweis der Eindeutigkeit nehmen wir an, dass für $n_1, n_2 \in \integer$ die Ungleichungen $n_1 \leq x < n_1 + 1$ und $n_2 \leq x < n_2+1$ gelten. Daraus folgt $n_1 \leq x < n_2 + 1$ und auch $n_2 \leq x < n_1 + 1$ und damit $n_1 \leq n_2$ und $n_2 \leq n_1$. Desshalb ist $n_1 = n_2$.

  \textbf{Zu (c):} Sei $\epsilon > 0$. Dann gilt auch $\frac{1}{\epsilon} > 0$. Nach Teil (b) gibt es ein $n \in \nat$ mit $n \leq \frac{1}{\epsilon} < n$. Für dieses $n$ gilt aber auch $\frac{1}{n} < \epsilon$, wie in (c) behauptet.
\end{prf}

\begin{ex}[Supremum von Bildmengen]
  Sei $A\subseteq\real$ eine nichtleere Teilmenge von $\real$. Zeigen Sie, dass im Allgemeinen $\sup(\lfloor A \rfloor ) = \lfloor \sup(A) \rfloor$ nicht gilt. Hierbei ist $\lfloor A \rfloor$ das Bild von $A$ unter der Abrundungsfunktion.
\end{ex}

\mce{70}
\begin{thm}[Dichtheit von $\rat$ in $\real$]
  \label{thm: dichtheit von Q in R}
  Zwischen zwei reellen Zahlen $a,b \in \real$ mit $a < b$ gibt es  ein $r \in \rat$ mit $a < r < b$.
\end{thm}
\begin{prf}
  Nach \ref{thm: das archimedische prinzip} (c) $\exists m \in \integer$:
  \[
    \frac{1}{m} < b-a.
  \]

  Ebenso gibt es nach \ref{thm: das archimedische prinzip} (b) ein $n \in \integer$:
  \[
  \begin{aligned}
    &n-1 \leq ma < n \mytext{oder äquivalenterweise }\\
    &\tfrac{n}{m} - \tfrac{1}{m} = \tfrac{n-1}{m} \leq a < \tfrac{n}{m}.
  \end{aligned}
  \]

  Dies führt zu
  \[
  a < \underbrace{\tfrac{n}{m}}_{r} \leq a + \tfrac{1}{m} < a + b- a = b
  \]

  womit die Behauptung bewiesen ist, wenn wir $r:\equiv \frac{n}{m}$ wählen.
\end{prf}

Dies bedeutet auch für jede Umgebung $I$, dass $I \cap \rat \neq \varnothing$

\mce{71}
\begin{ex}[Jede reelle Zahl ist ein Supremum einer Menge von rationalen Zahlen] Zeigen Sie, dass für jedes $x \in \real$ das Supremum von $\{ r \in \rat \mid r < x\}$ gerade $x$ ist.
\end{ex}

\subsection{Häufungspunkte einer Menge}
\mce{73}
\begin{mydef}[Häufungspunkte von Mengen]
  Sei $A \subseteq \real$ und $x_0 \in \real$. Man nennt $x_0$ einen \qt{Häufungspunkt der Menge} $A$, falls $\forall \epsilon > 0 \; \exists a \in A: 0 < \abs{a-x_0} < \epsilon$.

  In andern Worten: Es gibt für einen Häufungspunkt $x_0$ in jeder Umgebung abgesehen von $x_0$ Punkte in $A$. Hierbei muss $x_0$ nicht unbedingt in $A$ liegen. Die Menge $A$ kommt also ihren Häufungspunkten \qt{von aussen} beliebig nahe.
\end{mydef}

\begin{ex}[Endliche Mengen haben keine Häufungspunkte] Zeigen Sie, dass eine endliche Teilmenge $A \subseteq\real$ keine Häufungspunkte hat.
\end{ex}

\begin{thm}[Existenz von Häufungspunkten]
  Sei $A \subseteq \real$ eine beschränkte unendliche Teilmenge. Dann existiert ein Häufungspunkt von $A$ in $\real$.
\end{thm}
\begin{prf}
  Angenommen $m, M \in \real$ sind die beiden Schranken von $A$, so dass $A \subseteq [m,M]$. Wir definieren
  \[
    X :\equiv \{ x \in \real \mid \abs{A \cap (-\infty, x] } < \infty \}
  \]

  Dann ist $m \in X$ da $\abs{A \cap (-\infty,m]} \leq 1$.

  Des Weiteren gilt $x < M \quad \forall x \in X$ also $m \notin X$, denn für ein hypothetisches $x \geq M$ wäre $A \cap (-\infty,x] = A \cap (-\infty, M] = A$ eine unendliche Menge mit $\abs{A \cap (-\infty,x]} = \abs{A \cap (-\infty, M]} = \abs{A}=\infty$. Wäre also zum Beispiel $A$ von der Gestalt $A=[b,c]$ oder $A = (a,b)$ für $b,c \in \real$ dann wäre $X=\{b \}$ eine Menge mit nur einem Element $b$.
  Daher ist $X$ eine beschränkte, nicht-leere Teilmenge von $\real$, womit das Supremum $x_0 :\equiv \sup(X)$ nach \ref{thm: supremum} existiert. Also $X$ kann auch aus Elementen bestehn, die nicht in $A$ enthalten sind.

  Sei nun $\epsilon>0$. Dann existiert ein $x\in X$ mit $x > x_0 - \epsilon$, was zeigt, dass $A\cap (-\infty, x_0 -\epsilon]$ eine endliche Menge ist, da
  \[
    A\cap (-\infty, x_0 -\epsilon] \subseteq A \cap (-\infty, x] \subseteq X.
  \]

  Des Weitern gilt $x_0 + \epsilon \notin X$. Damit gilt
  $
    \abs{A \cap [-\infty, x_0 + \epsilon]} = \infty.
  $

  Es folgt, dass
  \[
    A \cap (x_0 - \epsilon, x_0 + \epsilon] =
    \underbrace{(A \cap (-\infty, x_0 + \epsilon])}_{\text{unendlich}} \, \backslash \,  \underbrace{(A \cap (-\infty, x_0 - \epsilon])}_{\text{endlich}}
  \]

  eine unendliche Menge ist und abgesehen von möglicherweise $x_0$ und
  $x_0 + \epsilon$ noch weitere Punkte besitzen muss. Da $\epsilon > 0$ beliebig war, sehen wir, dass $x_0$ ein Häufungspunkt der Menge $A$ ist.
\end{prf}

\begin{ex}[Alternative Charakterisierung von Häufungspunkten]
  Sei $A \subseteq \real$ und $x_0 \in \real$. Zeigen Sie, dass $x_0$ genau dann ein Häufungspunkt der Menge $A$ ist, wenn $\forall \epsilon > 0$ der Durchschnitt von $A$ mit der $\epsilon$-Umgebung $(x_0 - \epsilon, x_0 + \epsilon)$ unendlich viele Punkte enthält.
\end{ex}

\subsection{Intervallschachtelungsprinzip}

Der durchschnitt von nicht-leereren, ineainder geschachtelten Intervallen $I_1 \subseteq I_2 \cdots $ mit $I_k \subseteq \real$, die kleiner werden, kann durchaus leer sein. Zum Beispiel gilt
\[
  \bigcap_{n=1}^{\infty} [n, \infty) = \varnothing,
  \qquad
  \bigcap_{n=1}^{\infty} \left(0, \frac{1}{n}\right) = \varnothing
\]

\begin{thm}[Intervallschachtelungsprinzip]
  \label{thm: Intervallschachtelungsprinzip}
  Sei $\forall n \in \nat$ ein (nicht-leeres, abgeschlossenes, beschränktes) Intervall
  \[
    I_n :\equiv [a_n, b_n]
  \]

  gegeben, so dass $\forall n,m \in \nat, m \leq n$ die Inklusion $I_m \subseteq I_n$ oder äquivalenterweise die Ungleichungen
  \[
  a_m \leq a_n \leq b_n \leq b_m
  \]

  gelten.
  Dann ist der Durchschnitt
  \[
    \bigcap_{n=1}^{\infty}I_n :\equiv \left[ \sup\{ a_n \mid n \in \nat \}, \inf \{ b_n \mid n \in \nat \}\right]
  \]

  nicht-leer.
\end{thm}\mce{77}
\begin{prf}
  Es gilt $\forall l,m,n \in \nat$, $l \leq m \leq n$:
  \[
    a_l \leq a_m \leq a_n \leq b_n \leq b_m \leq b_l.
  \]

  Dies zeigt, dass $b_m$ eine obere Schranke von $\{a_k \mid k \in \nat \}$ ist, woraus
  \[
    \widetilde{a} :\equiv \sup \{ a_k \mid k \in \nat \} \leq b_m \in \{ b_k \mid k \in \nat \}
  \]

  folgt. Somit ist auch $\{ b_k \mid k \in \nat \}$ von unten beschränkt. Da $m \in \nat$ beliebig ist, können wir die Gleichung oben für $\widetilde{b} :\equiv \inf \{ b_k \mid k \in \nat \}$ etwas umgestalten zu
  \[
  \widetilde{a} = \sup \{ a_k \mid k \in \nat \} \leq \inf \{ b_k \mid k \in \nat \} = \widetilde{b}.
  \]

  Für $x \in \real$ gilt nun die Abfolge von Äquivalenzen
  \[
  \begin{aligned}
    x \in \bigcap_{n=1}^{\infty}[a_n, b_n]
    & \iff \forall n \in \nat: a_n \leq x \leq b_n \\
    & \iff (\forall n \in \nat: a_n \leq x) \wedge (\forall n \in \nat: x \leq b_n) \\
    & \iff \widetilde{a} \leq x \wedge x \leq \widetilde{b},
  \end{aligned}
  \]

  womit $\bigcap_{n=1}^{\infty}[a_n, b_n]=[\widetilde{a}, \widetilde{b}]=\left[ \sup\{ a_k \mid k \in \nat \}, \inf \{ b_k \mid k \in \nat \}\right]$ gilt, was zu beweisen war.
\end{prf}

\mce{79}
\begin{ex}[Charakterisierung von Intervallen]
  \phantom{.}
  \begin{enumerate}
    \item Eine Teilmenge $I \subseteq \real$ ist ein Intervall $\iff$ $\forall x,y \in I, z \in \real: (x \leq z \leq z) \implies z \in I$.
    \item Daraus folgt, dass ein beliebiger Schnitt $\bigcap_{I \in \mathcal{I}} I$ von Intervallen $I \in \mathcal{I}$ ein Intervall ist.
  \end{enumerate}
\end{ex}


\begin{ex}[Zusammenziehende Intervalle]
  Sei zusätzlich wie in \ref{thm: Intervallschachtelungsprinzip} gegeben dass $\inf\{b_n-a_n \mid n \in \nat \} = 0$. Das heisst, dass die Intervalle immer kürzer werden. Dann besteht $\bigcap_{n=1}^{\infty}[a_n, b_n]$ aus nur einem Punkt.
\end{ex}



	\chapter{Funktionen und die reellen Zahlen}
  \section{Summen und Produkte}
\begin{mydef-non}
  Sei $n\in \nat$ und $a_1, \ddd, a_n \in \compl$ oder $a_1, \ddd, a_n$ Elemente eines Vektorraums $V$. Dann definieren wir das Produkt und die Summe als
\[
  \sum_{j=1}^{n} a_j
  :\equiv
  \begin{cases}
    a_1=\sum_{j=1}^{1}a_j & \mytext{falls} n=1 \mytext{und} \\
    \sum_{j=1}^{k+1}a_j = \left(\sum_{j=1}^{k} a_j \right) + a_{k+1} & \mytext{falls} n = k+1 > 1
  \end{cases}
\]

\[
\prod_{j=1}^{n} a_j
:\equiv
\begin{cases}
  a_1=\prod_{j=1}^{1}a_j & \mytext{falls} n=1 \mytext{und} \\
  \prod_{j=1}^{k+1}a_j = \left(\prod_{j=1}^{k} a_j \right) \cdot a_{k+1} & \mytext{falls} n = k+1 > 1
\end{cases}
\]
\end{mydef-non}

Formal gesehen ist $j \in \{1, \ddd, n\} \mapsto a_j \in V$ eine Funktion, die oft durch eine konkrete Formel gegeben sein wird.
Der einfachste Fall einer Funktion $j \mapsto a_j$ ist der Fall der konstanten Funktion $a_j = z$ für ein $z \in \compl$ und $\forall j \in \{1, \ddd, n\}$. In diesem Fall ergibt sich die Summe zu
\[
\sum_{j=1}^{n}z = nz \quad \forall n \in
 \nat.
\]

\begin{mydef-non}
  Im Falle des Produkts erhalten wir aber die Definition der \qt{Potenzfunktion} für $z \in \compl$
\[
z^n :\equiv \prod_{j=1}^{n}z \quad \forall n \in \nat. $ Rekursiv gilt dann $ z^1 =z $ und $ z^{n+1} = z^n z
\]\end{mydef-non}

Man erweitert die Definition durch $z^0=1 \quad \forall z \in \compl$ und $z^{-n} = (z^n)^{-1} \quad \forall z \in \compl ^ \times, n \in \nat$.

\begin{mydef-non}
  Allgemeiner ist die Summe $\sum_{i=m}^{n} a_j$ für $m,n \in \nat$ rekursiv durch
\[
  \sum_{j=m}^{n}
  :\equiv
  \begin{cases}
    0 & \mytext{falls} m>n, \\
    a_m & \mytext{falls} m = n \mytext{und} \\
    \left(\sum_{j=m}^{n-1}a_j\right) + a_n & \mytext{falls} m < n
  \end{cases}
\]

definiert. Man bezeichnet die $a_j$'s als die \qt{Summanden} und $j$ als den \qt{Index} der Summe.

Das Produkt $\prod_{i=m}^{n} a_j$ wird für $m,n \in \nat$ rekursiv durch
\[
\prod_{j=m}^{n}
:\equiv
\begin{cases}
  0 & \mytext{falls} m>n, \\
  a_m & \mytext{falls} m = n \mytext{und} \\
  \left(\prod_{j=m}^{n-1}a_j\right) \cdot a_n & \mytext{falls} m < n
\end{cases}
\]

Man bezeichnet die $a_j$'s als die \qt{Faktoren} und $j$ als den \qt{Index} des Produkts. Der Index $j$ hat ausserhalb der Summe und es Produkts keinerlei Bedeutung und ist sozusagen eine lokale Variable.
\end{mydef-non}

Es gilt
\[
  \sum_{j=m}^{n} a_j = \sum_{k=m}^{n} a_k = \sum_{l=m}^{n} a_l \mytext{sowie}
  \prod_{j=m}^{n} a_j = \prod_{k=m}^{n} a_k = \prod_{l=m}^{n} a_l.
\]

\begin{ex}[Indexverschiebung]
  Es gilt
  \[
  \sum_{j=m}^{n} a_j = \sum_{k=m-1}^{n-1}a_{k+1} = \sum_{l=m+1}^{n+1}a_{l-1}
  \mytext{sowie}
  \prod_{j=m}^{n} a_j = \prod_{k=m-1}^{n-1}a_{k+1} = \prod_{l=m+1}^{n+1}a_{l-1}
  \]
\end{ex}

\begin{ex}[Potenzregeln]
  Beweisen Sie
  \begin{itemize}
    \item $(zw)^m = z^m w^m$
    \item $z^{m+n} = z^m z^n$ und
    \item $(z^m)^n = z^{mn}$
  \end{itemize}

  zuerst $\forall z,w \in \compl$ und $m,n \in \nat$ mit vollständiger Induktion und dann $\forall z,w \in \compl^\times$ und $m,n \in \compl$.
\end{ex}

\subsection{Rechenregeln für die Summe}
\begin{thm-non}
  Für $c \in \real$ oder $c \in \compl, m,n \in \nat$, $m\neq n$ gilt
\[
\sum_{k=m}^{n}(a_k + b_k) = \sum_{k=m}^{n}a_k + \sum_{k=m}^{n} b_k
\mytext{und}
\sum_{k=m}^{n}(c a_k) = c \sum_{k=m}^{n}a_k,
\]

wobei $a_1, \ddd, a_n, b_1, \ddd, b_n$ in einem reellen oder kopmlexen Vekktorraum $V$ liegen.
\end{thm-non}

\begin{thm-non}[Teleskopsumme]
  \[
  \begin{aligned}
    \sum_{k=m}^{n}(a_{k+1}-a_k) &= (a_{m+1}-a_m) + (a_{m+2}-a_{m+1}) + (a_{m+3} + a_{m+2}) \\
    & \quad + \cdots + (a_{n-1}-a_{n-2})+(a_n-a_{n-1}) + (a_{n+1}-a_n) \\
    & =a_{n+1} - a_m
  \end{aligned}
  \]
\end{thm-non}
\begin{prf}
  Formaler argumentiert
  \[
  \begin{aligned}
    \sum_{k=m}^{n}(a_{k+1} - a_k)
    &= \sum_{k=m}^{n} a_{k+1} - \sum_{k=m}^{n} a_k = \sum_{j=m+1}^{n+1} a_j - \sum_{k=m}^{n} a_k \\
    & = \left( a_{n+1} + \sum_{j=m+1}^{n} a_j \right) - \left(a_m + \sum_{k=m+1}^{n} a_k \right) = a_{n+1} - a_m
  \end{aligned}
  \]
\end{prf}

\begin{ex}[Abel-Summation] Seien $a_1, \ddd, a_n, b_1, \ddd, b_n \in \compl$. Wir setzten $A_k :\equiv \sum_{j=1}^{k}a_j$ für $k \in \nat$ mit $k \leq n$.
Zeigen Sie die Abel-Summationsformel
\[
  \sum_{k=1}^{n}a_k b_k = A_n b_n + \sum_{k=1}^{n-1} A_k (b_k - b_{k+1}).
\]

Verwenden Sie dazu die Gleichung $a_k = A_k - A_{k-1}  \quad \forall k \in \nat$ mit $k \leq n$. Wenden Sie die Formel auf $\sum_{k=1}^{2n} \frac{(-1)^k}{k}$ an.
\end{ex}

\begin{imp-ex}[Verallgemeinerte Dreiecksungleichung]
  Zeigen Sie, dass für alle Zahlen $a_1, \ddd, a_n \in \compl$ die Ungleichung
  \[
    \abs{\sum_{i=1}^{n}a_i} \leq \sum_{i=1}^{n} \abs{a_i}
  \]
  gilt.
\end{imp-ex}

\mce{5}
\begin{thm}[Bernoulli'sche Ungleichung]
  $\forall a \in \real, a \geq -1$ und $\forall n \in \nat_0$:
  \[(1+a)^n \geq 1 + na. \]
\end{thm}
\begin{prf}
  Per Induktion. Für $n=0$ haben wir $(1+a)^0 = 1 = 1 + 0a$. Sei nun also $n \geq 1$ und $a \geq 1$ und nehmen wir an die Gleichung stimmt für $n$. Dann haben wir für $n+1$:
  \[
  \begin{aligned}
    (1+a)^{n+1}
    &= (1+a)^n (1+a) \\
    &\geq (1+na)(1+a) \quad \mytext{weil $a$ grösser oder gleich $-1$ ist} \\
    &= 1+na+a+na^2 \\
    &= 1+ (n+1)a + na^2 \\
    &\geq 1 + (n+1)a
  \end{aligned}
  \]
  \vspace*{-\baselineskip}
\end{prf}

\subsection{Rechenregeln für das Produkt}
Für $m,n \in \nat, m\leq n$ und $a_m, \ddd, a_n, b_m, \ddd, b_n \in \compl$ gilt
\[
  \prod_{k=m}^{n} (a_k b_k) = \left(\prod_{k=m}^{n} a_k\right) \left(\prod_{k=m}^{n} b_k\right)
  \mytext{und}
  \prod_{k=m}^{n} (ca_k)=c^{n-m+1}\left(\prod_{k=m}^{n}a_k\right)
\]

  \chapter{Das Riemann-Integral}


  \chapter{Metrische Räume, Folgen und Stetigkeit}
	\input{content/5A-normierte-vektorraeume}
	\input{content/5B-metrische-raeume}
	\section{Folgen und Konvergenz}

% Definition 5.20
\setcounter{thm}{19}
\begin{mydef}[Folge]
	Eine \qt{Folge} in $X$ ist eine Abbildung $a: \nat \to X$. Man schreibt
	\begin{equation}
    a_n :\equiv a(n) $ als das $n$-te \qt{Folgenglied}, und verwendet folgende Abkürzungen$
	\end{equation}
  \begin{equation}
  	a: \nat \to X \equiv (a_1, a_2, \ldots) \equiv (a_n)_{n\in\nat} \equiv(a_n)_{n=1}^{\infty} \equiv (a_n)_n.
  \end{equation}
  
	Die Menge der Folgen wird auch als $X^\nat$ bezeichnet.
	
	Eine Folge heisst \qt{konstant}, falls $\forall m,n \in \nat: a_n=a_m$ und \qt{schliesslich konstant}, falls 
	\begin{equation}
		\exists M\in \nat: \; \forall m,n \in \nat \; : m,n \geq \nat \implies a_n = a_m
	\end{equation}
\end{mydef}

% Definition 5.21
\begin{mydef}[Konvergenz]
	Sei $(X,d)$ ein metrischer Raum und $(a_n)_n$ eine Folge in $X$.
	
	$(a_n)_n$ \qt{konvergiert} oder \qt{strebt} gegen einen Punkt $A\in X$, falls 
	\begin{equation}
		\forall \epsilon > 0 \; \exists N \in \nat \; \forall n \geq N: (a_n, A) < \epsilon.
	\end{equation}
	
	In diesem Fall nenne wir den Punkt $A$ den \qt{Grenzwert} der Folge und schreiben auch
	\begin{equation}
		\limninfty a_n = A
	\end{equation}
	
	Weiter ist eine Folge in $X$ \qt{konvergent}, falls sie einen Grenzwert besitzt, und \qt{divegent}, falls nicht.
\end{mydef}


\begin{thm}
	Sei $(X,d)$ ein metrischer Raum. Jede konvergente Folge in $X$ besitzt einen eindeutigen Grenzwert.
\end{thm}
\begin{proof} Beweis durch Widerspruch. Seien $A_1, A_2$ zwei verschiedene Grenzwerte einer konvergenten Folge $(a_n)_n$. Sei $\epsilon :\equiv \frac{d(A_1, A_2)}{2} > 0$. Aus der Annahme folgt
	\begin{equation}
		\begin{aligned}
			\exists N_1 \in \nat: d(a_n, A_1) &< \epsilon \quad \forall n \geq N_1 \und \\
			\exists N_2 \in \nat: d(a_n, A_2) &< \epsilon \quad \forall n \geq N_2.
		\end{aligned}
	\end{equation}
	
	Daraus folgt
	\begin{equation}
		 N :\equiv \max \{ N_1, N_2 \} : d(a_n, A_1) < \epsilon \und d(a_n, A_2) < \epsilon \quad \forall n \geq N.
	\end{equation}
	
	Nach der Dreiecksungleichung gilt
	\begin{equation}
		d(A_1, A_2) \leq d(A_1, a_N) + d(a_N, A_2) \; \boxed{<} \;  2\epsilon = d(A_1, A_2)
	\end{equation}
	
	Was ein Widerspruch darstellt. Es kann nicht zwei Grenzwerte geben.
\end{proof}

\emph{Erinnerung:} Für einen metrischen Raum $(X,d)$ und 
$\epsilon > 0$ ist der \qt{$\epsilon$-Ball} oder auch die \qt{$\epsilon$-Umgebung} um $x_0\in X$ durch
\begin{equation}
	B_{\epsilon} (x_0) :\equiv \{ x\in X \mid d(x, x_0) <\epsilon \}
\end{equation}

gegeben. Eine allgemeine Umgebung wird wie folgt definiert.

% Definition 5.24
\setcounter{thm}{23}
\begin{mydef}[Umgebung]
	Eine \qt{Umgebung} $U$ von $x_0 \in X$ ist Teilmenge $U \subseteq X$, die eine $\epsilon$-Umgebung von $x_0$ für ein $\epsilon > 0$ enthält.
\end{mydef}

% Lemma 5.25
\begin{thm}[Indexverschiebung]
	$\forall l \in \nat_0$: $(a_n)_n$ ist konvergent $\iff$ $(a_{n+l})_n$ ist konvergent.
	
	In diesem Fall gilt
	\begin{equation}
		\limninfty a_n = \lim_{n \to \infty} a_{n+l}.
	\end{equation}
	
	Dies gilt für jede Folge $(a_n)_n$ und $(a_{n+l})_n$ in einem metrischen Raum.
\end{thm}




  \chapter{Grenzwerte reeller Folgen und Funktionen}
 	.


  \chapter{Reihen, Funktionsfolgen und Potenzreihen}
  \section{Reihen}

\begin{mydef}
	Sei $(a_k)_k \in \nat \times \compl$. Man nennt $a_k$ das \qt{$k$-te Glied} oder den \qt{$k$-ten Summanden} der \qt{(unendlichen) Reihe} 
  \begin{equation}
		\sum_{k=1}^{\infty} a_k. $ Man nennt $
	\end{equation}
	\begin{equation}
		s_n=\sum_{k=1}^{n} a_k 
	\end{equation}
	
	die \qt{$n$-te Partialsumme} der obigen Reihe. Wir nennen die Reihe \qt{konvergent}, falls der Grenzwert
	\begin{equation}
		\sum_{k=1}^\infty a_k :\equiv \lim_{n \to \infty} \sum_{k=1}^{n} a_k = \lim_{n \to \infty} s_n \in \compl
	\end{equation}
	
	existiert. Wir nennen dies den \qt{Wert der Reihe}. Ansonsten nennen wir die Reihe \qt{divergent}.
\end{mydef}

\begin{thm}
	Falls $\sum_{k=1}^{\infty} a_k$ konvergiert, dann ist $(a_n)_n$ eine Nullfolge.
\end{thm}

\setcounter{thm}{5}
\begin{thm}[Linearität]
	Seinen $\sum_{k=1}^{\infty} a_k$ und $\sum_{k=1}^{\infty} b_k$ konvergente Reihen und $\alpha \in \compl$. Dann sind die Reihen  $\sum_{k=1}^{\infty} (a_k + b_k)$ und $\sum_{k=1}^{\infty} \alpha a_k$ konvergent und es gilt
	\begin{equation}
		\sum_{k=1}^{\infty} (a_k + b_k) = \sum_{k=1}^{\infty} a_k + \sum_{k=1}^{\infty} b_k \und \sum_{k=1}^{\infty} (\alpha a_k) = \alpha \sum_{k=1}^{\infty} a_k
	\end{equation}
\end{thm}

\setcounter{thm}{7}
\begin{thm}[Indexverschiebung für Reihen]
	Sei $\sumkinfty a_k$ eine Reihe. Es gilt $\forall M \in \nat$:
	
	Die Reihe $\sum_{k=M}^{\infty} a_k = \sumlinfty a_{l+M-1}$ ist konvergent $\iff$ die Reihe $\sumkinfty a_k$ ist konvergent.
	
	In diesem Fall gilt
	\begin{equation}
		\sumkinfty a_k = \sum_{k=1}^{M-1} a_k + \sum_{k=M}^{\infty} a_k.
	\end{equation}
\end{thm}

\begin{thm}[Zusammenfassen von benachbarten Gliedern]
	Sei $\sumninfty a_n$ eine konvergente Reihe und $(n_k)_k \in \nat \times \nat$ eine streng monoton wachsende Folge (natürlicher Zahlen).
	Sei 
	\begin{equation}
		\begin{aligned}
			A_1 &: \equiv a_1 + \cdots + a_{n_1} \\
			A_k &: \equiv a_{n_{k-1}} + \cdots + a_{n_k} \fuer k \geq 2
		\end{aligned}
	\end{equation}
	Dann gilt
	\begin{equation}
		\sumkinfty A_k = \sumkinfty a_n
	\end{equation}
\end{thm}

\subsection{Reihen mit nicht-negativen Gliedern}

\begin{thm}[Monotone Partialsummen]
	Für eine Reihe $\sumkinfty a_k$ mit $a_k \geq 0$ bilden die Partialsummen $s_n=\sumkn a_k$ eine monoton wachsende Folge. Falls diese Folge $(s_n)_n$ beschränkt ist, dann Konvergiert die Reihe. Ansonsten gilt
	\begin{equation}
		\sumkinfty a_k = \limninfty s_n = \infty.
	\end{equation}	
\end{thm}

\begin{thm}[Vergleichssatz]
	
	Seien $\sumkinfty a_k$ und $\sumkinfty b_k$ zwei Reihen mit der Eigenschaft 
	\begin{equation}
		0\leq a_k \leq b_k. $ Dann gilt $
	\end{equation}
	\begin{equation}
	 	\underbrace{\sumkinfty a_k}_{\text{Minorante von $\sumkinfty b_k$}}  \leq \underbrace{\sumkinfty b_k}_{\text{Majorante von $\sumkinfty a_k$}} 
	\end{equation} 
	
	und insbesondere gelten die Implikationen
	\begin{equation}
		\begin{aligned}
			&\sumkinfty b_k
			&\text{konvergent} 
			&\implies 
			&\sumkinfty a_k 
			&\mytext{konvergent} 
			&\text{(Majorantenkriterium)} \\
			&\sumkinfty a_k
			&\text{divergent} 
			&\implies 
			&\sumkinfty b_k 
			&\mytext{divergent} 
			&\text{(Minorantenkriterium)}
		\end{aligned}
	\end{equation}
	
	Diese beiden Implikationen treffen auch dann zu, wenn $0\leq a_n \leq b_n$ nur für alle hinreichend grossen $n \in \nat$ gilt.
\end{thm}

\setcounter{thm}{15}
\begin{thm}[Verdichtung]
	\label{thm:verdichtung}
	Eine Reihe $\sumkinfty a_k$ mit $a_1 \geq a_2 \cdots \geq 0$ (nicht-negativen, monoton abnehmenden Gliedern) ist konvergent $\iff$ $\sumkinfty 2^k a_{2^k}$ ist konvergent.
\end{thm}
\begin{proof}
	Es gilt $\forall n\in \nat$
	\begin{equation}
		\begin{aligned}
			a_2    &\leq a_2 \leq a_1 \\
			2a_4   &\leq a_3+a_4 \leq 2a_2 \\
			2^2a_8 &\leq a_5+a_6+a_7+a_8 \leq 2^2a_4 \\
			       &\quad \vdots \\
			2^{n} a_{2^{n+1}} & \leq a_{(2^n)+1} + \ldots + a_{2^{n+1}} \leq 2^n a_{2^n}
		\end{aligned}
	\end{equation}
	\begin{equation}
		\implies \sumknplone 2^{k-1} a_{2^l} \leq \sum_{l=2}^{2^{n+1}} a_l \leq \sumkzeroton 2^{k}a_{2^k}
	\end{equation}
	
	Der Rest folgt durch \autoref{thm:verdichtung} und durch den Grenzübergang $n \to \infty$.
\end{proof}


\subsection{Bedingte Konvergenz}
\begin{mydef}
	Eine Reihe $\sumninfty a_n$ mit $a_n \in \compl$ \qt{konvergiert absolut}, falls die Reihe $\sumkinfty |a_n|$ konvergiert. \\
Die Reihe $\sumninfty a_n$ ist \qt{bedingt konvergent}, falls sie konvergiert, aber nicht absolut konvergiert.
\end{mydef}

\setcounter{thm}{20}
\begin{thm}[Riemannscher Umordnungssatz]
	Sei $\sumninfty a_n$ eine bedingt konvergente Reihe mit reelen Gliedern. Dann gibt es zu jedem $A\in \real$ eine bijektive Funktion (eine Umordnung) $\varphi: \nat \to \nat$, so dass die Reihe $\sumninfty a_{\varphi(n)}$ bedingt konvergiert und $\sumninfty a_{\varphi(n)} = A$ ist. Weiters gibt es eine Umordnung der Reihe, die divergiert.	
\end{thm}


\subsection{Alternierende Reihen}

Für eine Folge $(a_n)_n$ positiver Zahlen bezeichnen wir die Reihe $\sumkinfty (-1)^{n+1}a_n$ als \qt{alternierende Reihe}.

\begin{thm}[Leibniz-Kriterium]
	Gegeben sei eine monoton fallende Folge $(a_n)_n$ positiver (reeler?) Zahlen, die gegen Null konvergiert. Dann konvergiert die zugehörige alternierende Reihe $\sumkinfty (-1)^{k+1} a_k$ und es gilt, dass
	\begin{equation}
		\label{eq: fehlerabschaetzung}
		\left | \sumkonetilll (-1)^{k+1} a_k - \sumkinfty (-1)^{k+1} a_k \right | \leq a_{l+1}. \fueralle l\in \nat
	\end{equation}
	
	Weiters ist
	\begin{equation}
		\sum_{k=1}^{2n} (-1)^{k+1} a_k \leq \sum_{k=1}^{\infty} (-1)^{k+1} a_k \leq \sum_{k=1}^{2n-1} (-1)^{k+1}a_k
	\end{equation}
	
	Abschätzungen des Typs (\autoref{eq: fehlerabschaetzung}) werden meist auch Fehlerabschätzungen oder Fehlerschranken bezeichnet. (Intuitiv: Approximation durch Summe bis zu einem gewissen Glied anstatt der Wert der Reihe)
\end{thm}


\subsection{Das Cauchy-Kriterium}

\begin{thm}[Cauchy-Kriterium] Die Reihe $\sumkinfty a_k$ konvergiert $\iff$ $\forall \epsilon > 0 \; \exists N \in \nat$, so dass
	\begin{equation}
		\left | \sum_{k=m}^{n} a_k \right | < \epsilon \fuer n \geq m \geq N
	\end{equation}
\end{thm}
\begin{proof}
	Dies folgt aus dem Cauchy-Kriterium für Folgen angewendet auf die Folge der Partialsummen $s_n=\sumkinfty a_k$, da 
	\begin{equation}
				s_n - s_{m-1} = \sum_{k=1}^n a_k - \sum_{k=1}^{m-1} a_k = \sum_{k=m}^n a_k \fuer n \geq m
	\end{equation}
\end{proof}


  \section{Absolute Konvergenz}

% Proposition 7.28
\setcounter{thm}{27}
\begin{thm}[Absolute Konvergenz]
	\label{thm:absolute-konvergenz}
	Eine absolut konvergente Reihe $\sumninfty a_n$ ist auch konvergent und es gilt die verallgemeinerte Dreiecksungleichung
	\begin{equation}
		\left | \sumninfty a_n \right | = \sumninfty | a_n |
	\end{equation}
\end{thm}

\subsection{Hinreichende Kriterien für absolute Konvergenz}

\begin{thm}[Majorantenkriterium von Weierstrass]
	Sei $(a_n)_n \in \nat \times \compl$ und $(b_n)_n \in \nat \times \real$ mit $|a_n| < b_n$ für alle hinreichend grossen $n \in \nat$. Falls $\sumkinfty b_m$ konvergiert, dann ist ist $\sumkinfty a_k$ absolut konvergent.
\end{thm}

\begin{thm}[Cauchy-Wurzelkriterium]
	
	Sei $(a_n)_n$ eine Folge komplexer Zahlen und
	\begin{equation}
		\alpha = \limsup_{n \to \infty} \sqrt[n]{|a_n|} \in \real \cup \infty
	\end{equation}
	
	Dann gilt
	\begin{equation}
		\begin{aligned}
    	\alpha < 1 &\implies \sumkinfty a_n \mytext{ist absolut konvergent} \\
    	\alpha > 1 &\implies \sumkinfty a_n \mytext{ist divergent und $(a_n)_n$ ist keine Nullfolge}
		\end{aligned}
	\end{equation}
\end{thm}

\begin{thm}[D'Alemberts Quotientenkriterium]
Sei $(a_n)_n \in \nat \times \compl$ mit $a_n \neq 0 \; \forall n \in \nat$, so dass
\begin{equation}
	\alpha = \limninfty \frac{|a_{n+1}|}{|a_n|}
\end{equation}

existiert. Dann gilt
\begin{equation}
	\begin{aligned}
		\alpha < 1 &\implies \sumkinfty a_n \mytext{ist absolut konvergent} \\
		\alpha > 1 &\implies \sumkinfty a_n \mytext{ist divergent und $\limninfty a_n \neq 0$}
	\end{aligned}
\end{equation}
\end{thm}

% Wichtige Übung 7.33. Beweisen Sie Korollar 7.32

\subsection{Umordnen von Reihen}

% Satz 7.35 
\setcounter{thm}{34}
\begin{thm}
	Sei $\sumninfty a_n \in \nat \times \compl$ absolut konvergent. Sei $\varphi: \nat \to \nat$ eine Bijektion. Dann ist $\sumninfty a_{\varphi(k)}$ ebenso absolut konvergent und es gilt
	\begin{equation}
		\sumninfty a_n = \sumninfty a_{\varphi(n)}
	\end{equation}
\end{thm}

\subsection{Produkte} 

\begin{thm}
	
	Seien $\sumninfty a_n$ mit $a_n \in \nat \times \compl$ und $\sumninfty b_n$ mit $b_n \in \nat \times \compl$ zwei absolut konvergente Reihen. Sei $\varphi \to \nat \times \nat$ eine biijektive Abbildung. Dann ist
	\begin{equation}
		\sumninfty a_{\varphi(n)_1} b_{\varphi(n)_2} 
	\end{equation}
	
	eine absolut konvergente Reihe, wobei $\varphi (n) = (\varphi(n)_1, \varphi(n)_2)$. Weiters gilt
	\begin{equation}
		\sumninfty a_{\varphi(n)_1} b_{\varphi(n)_2} = \left( \sumninfty a_n \right) \left( \sumninfty b_n \right)
	\end{equation}
	
	Informell ausgedrückt kann man schreiben
	\begin{equation}
		\left( \sumninfty a_n \right) \left( \sumninfty b_n \right) =
	  \sumninfty \left( \sumninfty b_n \right) a_m =
	  \sum_{(m,n) \in \nat^2} a_m b_n
	\end{equation}
\end{thm}


% Korollar 7.37 (Cauchy-Produkt)
\setcounter{thm}{36}
\begin{thm}
	Falls $\sumninfty a_n$ und $\sumninfty b_n$ absolut konvergente Reihen mit komlexen Gliedern sind, dann gilt
	\begin{equation}
		\sum_{n=0}^{\infty} \left ( \sum_{k=0}^{n} a_{n-k} b_k \right )
		=\left( \sumninfty a_n \right) \left( \sumninfty b_n \right)
	\end{equation}
\end{thm}


	\section{Konvergenz von Funktionsfolgen}
\subsection{Punktweise Konvergenz}
\begin{mydef}[Funktionsfolgen und punktweise Kovergenz]
	Eine reellwertige oder komlexwertige ``Funktionsfolge" ist eine Folge $(f_n)_n$ von Funktionen $f_n: X \to \real$ (oder $f_n: X \to \compl$). 
	
	Wir sagen dass eine Funktion ``punktweise" gegen eine Funktion $f: X \to \real$ (oder $f: X \to \compl$) ``konvergiert", falls $f_n(x) \to f(x)$ für $n \to \infty$ $\; \forall x\in X$.
	
	Wir bezeichnen die Funktion $f$ als den ``punktweisen Grenzwert" (oder auch ``Grenzfunktion" oder ``Limes") der Funktionsfolge $(f_n)_n$
	\begin{equation}
		\forall x \in X \; \forall \epsilon > 0 \; \exists M \in \nat \; \forall n \in \nat : (n > M \implies |f_n(x)-f(x)| < \epsilon) 
	\end{equation}
	gegeben.

\end{mydef}

\subsection{Gleichmässige Konvergenz}

\begin{mydef}[Gleichmässige Konvergenz]
	Wir sagen, $f_n$ ``strebt gleichmässig" gegen $f$ für $n \to \infty$, oder dass $f$ der ``gleichmässige Grenzwert" der Funktionenfolge $(f_n)_n$ ist, falles es zu jedem $\epsilon > 0$ ein $M \in \nat$ gibt, so dass $\forall n > M$ und alle $x\in X$ die Abschätzung
	\begin{equation}
		|f_n(x)-f(x)|<\epsilon
	\end{equation}
	
	gilt. In der Prädikatenlogik ist gleichmässige Konvergenz durch
	\begin{equation}
	  \forall \epsilon > 0 \; \exists M \in \nat \; \forall n\in \nat : \big(n\geq M \implies ( \forall x \in X: |f_n(x) - f(x)| < \epsilon ) \big)
	\end{equation}
	gegeben.
	
\end{mydef}

% Satz 7.48
\setcounter{thm}{47}
\begin{thm}[Gleichmässige Konvergenz und Stetigkeit]
	\label{thm:gleichmaessige-konvergenz-und-stetigkeit} 
	Sei $D\subseteq \compl$ und $f_n : D\to \compl$ eine Funktionsfolge stetiger Funktionen. Falls $(f_n)_n$ gleichmässig gegen $f:d \to \compl$ konvergiert, dann ist $f$ ebenso stetig.
\end{thm}

% Satz 7.49
\begin{thm}
	Sei $[a,b]$ ein kompaktes Intervall und $f_n: [a,b] \to \real$ eine Funktionsfolge Riemann-integrierbarer Funktionen. Falls $(f_n)_n$ gleichmässig gegen $f:[a,b] \to \real$ konvergiert, dann ist $f$ Riemann-integrierbar und
	\begin{thm}
		\begin{equation}
			\limninfty \int_{a}^{b} f_n \dx = \int_{a}^{b} \limninfty f_n \dx = \int_{a}^{b} f \dx
		\end{equation}
	\end{thm}
\end{thm}
	\section{Potenzreihen}

% Definition 7.54
\setcounter{thm}{53}
\begin{mydef}[Potenzreihe]
	$\forall n \in \nat_0$ sei $a_n \in \compl$. Dann ist der formale Ausdruck
	\begin{equation}
		\sumninfty a_n z^n
	\end{equation}
	
	eine ``Potenzreihe" in der Variable $z$
\end{mydef}

\subsection{Konvergenzradius}

\begin{mydef}[Konvergenzradius]
	Der entsprechende ``Konvergenzradius" wird durch
	\begin{equation}
		R=\frac{1}{\limsup_{n \to \infty} \sqrt[n]{|a_n|}}
	\end{equation}
	
	definiert. Wir setzen $\frac{1}{+\infty} = 0$ und hier $\frac{1}{0} = +\infty$
\end{mydef}

\begin{thm}[Über den Konvergenzradius]
	Sei $\sumninfty a_n z^n$ eine Potenzreihe und $R$ ihr Konvergenzradius. Dann konvergiert die Reihe für alle $z \in \compl$ mit $|z| < R$ absolut und divergiert für alle $z \in \compl$ mit $|z| > R$. \\
	Weiters konvergiert die Funktionenfolge $\sum_{j=0}^{n} a_j z^j$ gleichmässig gegen $\sumninfty a_n z^n$ auf jeder Kreisscheibe der Form $B_S(0) = \{ z \in \compl \mid |z| < S\}$ für jedes $S\in (0, R)$. Insbesondere definiert die Potenzreihe die stetige Abbildung 
	\begin{equation}
		z \in B_r (0) \mapsto \sum_{n=0}^{\infty} a_n z^n \in \compl
	\end{equation}
	
\end{thm}

\subsection{Addition und Multiplikation}

\begin{thm}[Summe und Produkte]
	Seien $\sumninfty a_n z^n$ und $\sumninfty b_n z^n$ zwei Potenzreihen mit Konvergenzradius $R_a$ respektive $R_b$. Dann gilt $\forall z \in \compl$ mit $|z| < \min \{R_a, R_b \}$
	\begin{equation}
		\begin{aligned}
			\sumninfty a_n z^n  \sumninfty b_n z^n  &= \sumninfty (a_n + b_n) z^n \\
			\left( \sumninfty a_n z^n \right) \left( \sumninfty b_n z^n \right)
			 & = \sumninfty \left( \sum_{k=0}^{n} a_{n-k} b_k z^n \right) 
		\end{aligned}
	\end{equation}
	
	Insbesondere ist der Konvergenzradius der Potenzreihen auf der recheten Seite mindestens $\min \{ R_a, R_b \}$
\end{thm}

\end{document}
